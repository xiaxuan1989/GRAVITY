\chapter{集合论与范畴论}\label{appx:cat}
\section{集合}
读者或已熟悉主要的集合、映射与关系知识,包括差集、交并补运算律、卡氏积(Cartisian product)、指标集、可数性、等价类、划分、幂集、陪域、逆像、双射等概念。我们对部分概念进行复习,并讲述读者或许不熟悉但会用到的知识。
若对本书未讲到的概念有所遗忘,可查阅任意分析学、代数学教材。
\subsection{ZFC 公理}

\textbf{Russell 佯谬}考虑了所有集合构成的“集合” $\textbf{V}$,利用\textbf{自我指涉}的思想,构造其子“集合” $\{x \in \textbf{V} : x \notin x\}$ 可导出矛盾。现代集合论处理此问题的主流方法是限制概括原理 $\left\{x : x \text{ 满足性质 } P\right\}$ 的应用范围,并在一阶逻辑与合适的公理系统下进行演绎。“集合” $\textbf{V}$ 实非集合,在一般的公理体系中称作\textbf{类}(class)。凡集合都是类,但还存在非集合的类,称作\textbf{真类}。

下面采用广泛使用的\textbf{Zermelo–Fraenkel 公理集合论}, 并承认\textbf{选择公理}(axiom of choice),简称 \textbf{ZFC 系统}。它包括一阶逻辑和若干公理。
\textbf{一阶逻辑}具有常见的 $\wedge$, $\neg$, $\implies$, $\forall$, $\exists$ 等逻辑联词与量词、括号 $(\;)$、变元 $x, y, z, \ldots$、二元谓词 $=,\in$ 等。可判定一个公式是否合乎语言的规则,例如括号的左右配对,合规者称为合式公式。
若 $x \in y$ 则称 $x$ 为 $y$ 的元素。ZFC 处理的所有对象 $x,y$ 等都应该理解为集合,集合的元素本身也是集合。

本书在逻辑层面诉诸常识, 不采取一阶逻辑的严格形式, 仅以数学工作者日用的素朴语言勾勒 ZFC 的公理, 后续的形式推导一并省去. 
\begin{enumerate}[\bfseries {A}.1]
	\item \textbf{外延公理}: 若两集合有相同元素, 则两者相等.

		空集为 $\mt:=\{u \in X: u \neq u\}$,$X$ 为任意集合。一个近乎无聊的应用是证明 $\mt$ 无关 $X$ 的选取,前提是集合确实存在,这点既可以独立成另一条“存在公理”,也可以划入稍后的无穷公理。
	\item \textbf{配对公理}: 对任意 $x,y$ 存在集合 $\{x,y\}$,其元素恰好是 $x$ 与 $y$。
	
		有序对 $(x,y)$ 可用配对公理来实作:$(x,y) := \{\{x\}, \{x,y\} \}$,由此定义积集 $X \times Y := \{(x,y) : x \in X,\; y \in Y \}$;准此要领可定义有限多集合的积 $X \times Y \times \cdots$。映射 $f: X \to Y$ 等同于其\textbf{图形} $\Gamma_f \subset X \times Y$:$\forall x \in X$,$\Gamma_f \cap (\{x\} \times Y)$ 是独点集,记 $\{f(x)\}$。
	\item \textbf{分离公理模式}: 设 $P$ 为关于集合的一个性质,$P(u)$ 表示集合 $u$ 满足性质 $P$,则对任意集合 $X$ 存在 $Y = \{u \in X : P(u) \}$。
	\item \textbf{并集公理}:  任意集合 $X$ 存在相应的并集 $\bigcup X := \{u : \exists v \in X \; \text{ 使得 }\; u \in v\}$。
	
		公理中的 $X$ 通常视作一族集合,$\bigcup X$ 也相应地写成 $\bigcup_{v \in X} v$。
	\item \textbf{幂集公理}: 任意集合 $X$ 的全体子集构成一集合 $2^X := \{u : u \subset X \}$,称为幂集。
	\item \textbf{无穷公理}: 存在无穷集。更详细地,存在归纳集 $x$ 使 $\mt\in x$,且 $\forall y\in x$,$y \cup \{y\} \in x$,表作
		$\exists x \bigl[ (\mt \in x) \;\wedge \;\forall y \in x \left(y \cup \{y\} \in x \right)  \bigr]$。
        预设归纳集存在的用意在于萃取它的子集 $\left\{\mt, \{\mt\}, \{  \mt, \{\mt\}\}, \{  \mt, \{\mt\}, \{  \mt, \{\mt\}\}\}, \ldots \right\}$,将立即看到这正是自然数集 $\N$。
	\item \textbf{替换公理模式}: 设映射 $F$ 的定义域为 $X$,则存在集合 $\Im F = \{ F(x) : x \in X\}$,称作值域。
	
		分离和替换公理模式里的性质 $P$ 和映射 $F$ 并非集合论意义下的寻常定义, 故无循环论证之虞, 它们实际是由一阶逻辑的合式公式定义的: 以映射 $x \mapsto y$ 为例, 这可以诠释为一阶逻辑中带有两个自由变元 $x,y$ 的合式公式 $\varphi$, 适合于 $\forall x\; \exists! y\; \varphi(x,y)$. 由于对每个 $P$ 或 $F$ 都产生一条相应的公理, 它们被称作公理模式.
	\item \textbf{正则公理}: 任意非空集都含有一个对从属关系 $\in$ 极小的元素.
	
		正则公理的一个重要推论是对于任意集合 $x$, 不存在无穷的从属链 $x \ni x_1 \ni x_2 \ni \cdots$, 特别地 $x \notin x$. 由之可建立集合的层垒谱系, 见 \ref{sec:Grot-universe}.
	\item \textbf{选择公理}: 设集合 $X$ 的每个元素皆非空 , 则存在函数 $g: X \to \bigcup X$ 使得 $\forall x \in X, \; g(x) \in x$ (称作选择函数)。

		选择公理中的 $X$ 应该设想为一族非空集, 选择函数 $g(x) \in x$ 意味着从每个集合 $x \in X$ 中挑出一个元素.
\end{enumerate}

由此可进一步构造自然数集 $\N$、整数集 $\Z$、有理数集 $\Q$ 和实数集 $\R$ 等,我们在高等数学、数学分析中已有接触。

\begin{theorem}
    置 $f:X\to Y$,任意 $A\subset Y$ 的逆像满足 $f^{-1}[A]=f^{-1}[A\cap\Im f]$。
\end{theorem}

\begin{proof}
    回顾一下,$f^{-1}[A]:=\{x\in X:f(x)\in A\}$。由于必然 $f(x)\in\Im f$,则 $f(x)\in A\iff f(x)\in A\cap \Im f$。
\end{proof}

\begin{theorem}
    $f[U] \subset V \iff U\subset f^{-1}[V]$。
\end{theorem}

以下简述商集的概念. 对于任意 $n \geq 1$, 集合 $X$ 上的 $n$ 元关系按定义是 $X^n = \underbracket{X \times \cdots \times X}_{n\; \text{项}}$ 的一个子集 $R$. 多数场合考虑的是二元关系, 此时以 $xRy$ 表示 $(x,y) \in R$. 满足以下性质的(二元)关系 $\sim$ 称作\emph{等价关系}:反身性、对称性、传递性


对于 $X$ 上的等价关系 $\sim$ 和任意 $x \in X$, 含 $x$ 的等价类记为 $[x] := \{y \in X: y \sim x \}$. \emph{商集}\index{shang@商 (quotient)}定义为全体等价类, 亦即 $X/\sim\; := \{[x] : x \in X \}$. 如此得到 $X$ 的无交并分解 (又称 $X$ 的划分) $X = \bigcup_{[x] \in X/\sim} [x]$, 而且 $x \sim y$ 当且仅当 $x,y$ 属于划分中的同一个子集. 易见 $X$ 上的等价关系一一对应于 $X$ 的划分.

我们将采用习见的集合论符号, 例如 $X \subsetneq Y$ 代表 $X \subset Y$ 且 $X \neq Y$, 以 $X \sqcup Y$ 代表 $X$ 和 $Y$ 的无交并, $X^Y$ \index[sym1]{$X^Y$}代表所有函数 $f: Y \to X$ 构成的集合, 特别地 $2^X = P(X)$\index[sym1]{$2^X$}, 这里 $2$ 代表恰有两个元素的集合. 我们将用 $\{\text{pt}\}$ 表示仅有一个元素的集合. 进一步, 对于以某集合 $I$ 为指标的一族集合 $(X_i)_{i \in I}$ 可以定义
\[ \begin{array}{ll}
	\text{并 } \; \bigcup_{i \in I} X_i, & \text{交} \; \bigcap_{i \in I} X_i, \quad (I \neq \emptyset), \\
	\text{无交并 } \; \bigsqcup_{i \in I} X_i, & \text{积} \; \prod_{i \in I} X_i.
\end{array}\]
对应于 $I=\emptyset$ 的并与无交并都是 $\emptyset$, 而积是 $\{\emptyset\}$. 空交的合理定义只能是全体集合构成的类 $\textbf{V}$, 这在 ZFC 系统内是犯规的.

de Morgan 律对任意基数的索引集都成立

公理集合论是当代数学的正统基础. 它的表述范围之广, 形式化程度之高, 以至于数学本身也堪成为数学研究的对象, 诸如何谓证明, 一个命题能否被证明, 公理系统的一致性(无矛盾)等等都能被赋予明晰的数学内容, 这类研究统称元数学\index{yuanshuxue@元数学 (metamathematics)}. 数学实践中平素基于自然语言的定义和论证等等, ``原则上''都能改写为 ZFC 或其他公理集合论里的形式语言, 从而为其陈述和验证赋予一套坚实的基础. 幸抑不幸, 这种基础长期以来仅仅是一种姿态: 即便对看似单纯的集合论定理, 其形式表述往往都极尽繁琐, 遑论靠人工来验证乃至于领略. 然而随着符号逻辑, 计算机理论与相关技术的持续进展, 形势逐渐在起变化. 如四色定理, 素数定理, Kepler 猜想等结果现在已有了形式的验证, 而形式化方法的应用还不限于数学. 集合论之外的思想 (如类型论) 对此似乎是必要的, 至少是起了极大的简化效果. 可以确定的是: 随着理论与技术携手并进, 人们对数学基础与数学实践两方面的认知还会不断被改写.

\subsection{宇宙}\label{sec:Grot-universe}


\section{范畴}
请读者大致回顾至今学习的\textit{所有知识}。我们遇到过各种各样的事物,它们之间又存在着各种联系:从整数之间的大小比较,到集合之间的映射,再到线性空间之间的同构。1940 年代,MacLane 为研究代数拓扑、线性代数中自然同构的问题,试图抽象这一思想,将各个东西及其间的联系,放到一个整体的学说中研究。这些“东西”称为\textbf{对象}(object),它们之间的关系用箭头(arrow)表示,称为\textbf{态射}(morphism)。
指定一组对象构成的集合及其间所有态射之集,就给出了一个\textbf{范畴}(category)。
范畴论将我们熟知的知识进行进一步的抽象封装。
我们不关心一个集合里具体有什么,而是把每个集合看作一个对象;不关心究竟把某个元素映射至何处,而是把映射单纯地看成箭头,研究集合上任意映射之集。
如此,全体集合构成\textbf{集合范畴} $\cate{Set}$。
对象、态射构成的“集合”可以是一般的类。
我们有了一种更为普适、简洁的视角去看待很多理论。

不仅如此,由于这些理论中的态射都具有相似性质(比如复合函数的结合性),这意味着这些理论实质上有非常紧密的联系待发掘。利用范畴论,就很容易找到不同理论之间的对应,比如欧氏几何和用代数表达图形的解析方法。甚至,可将某个理论的概念套进范畴的模版里,进一步发展原有的理论。
在范畴论里,\textit{关系就是一切}。许多情况下,一个数学对象完全由它与所有其他对象的关系决定。换言之,当且仅当两个对象以同样方式与范畴中的每个对象相关时,两个对象本质上是不可区分的。这是著名的\textbf{米田引理}(Yoneda lemma)的一个推论。这同我们的日常经验相符,你大可通过观察一个人的社交关系来确定其性格;若你发现两个社交媒体账号,其关注和动态高度一致,则有理由推断此二账号同属一人。于是,范畴论可以成为构建一个数学理论的蓝图,被称为\textit{数学的数学}。

剩下的关键在于对态射需设置何种要求。一般希望态射的复合具有结合性,且存在到自身的态射。为保证足够的抽象和普适性,这可以是全部的要求了:
\begin{definition}
    设两个类 $\mathrm{Ob},\mathrm{Mor}$,之间配上一对映射
     $\begin{tikzcd} \mathrm{Mor} \arrow[yshift=-0.5ex, r, "t"'] \arrow[yshift=0.5ex, r, "s"] & \mathrm{Ob} \end{tikzcd}$。
    将 $\mathrm{Ob}$ 的元素称作\textbf{对象},$\mathrm{Mor}$ 的元素称为\textbf{态射}。对某个态射 $f\in\mathrm{Mor}$,映射 $s,t$ 的像分别指明其\textbf{来源}(source)和\textbf{目标}(target),也即域和陪域。
    对 $X, Y \in \mathrm{Ob}$,记 $\hom(X, Y) := s^{-1}\{X\} \cap t^{-1}\{Y\}\subset\mathrm{Mor}$,称为 $\hom$-集,其元素称为从 $X$ 到 $Y$ 的态射。
    
    态射还满足如下要求。对任意对象 $X$ 都存在\textbf{恒等}(identity)\textbf{态射} $\id_X \in \hom(X, X)$。可证同个对象的任意两个恒等态射是相等的,故其实恒等态射是存在且唯一的。对任意 $X, Y, Z \in \mathrm{Ob}$,给定态射间的\textbf{复合}(composition)
        \begin{align*}
            \circ : \hom(Y, Z) \times \hom(X, Y) & \longrightarrow \hom(X, Z), \\
            (f, g) & \longmapsto f \circ g,
        \end{align*}
        不致混淆时常将 $f \circ g$ 简记为 $fg$,满足
        \begin{itemize}
            \item 结合律,即对任意 $h, g, f \in \mathrm{Mor}$,若复合 $f(gh)$ 和 $(fg)h$ 都有定义,则 $f (g h) = (f g) h$。
            故两边可以同写为 $f \circ g \circ h$ 或 $fgh$;
            \item 对任意 $f \in \hom(X,Y)$,有 $f \circ \id_X = f = \id_Y \circ f$。
        \end{itemize}
    $(\mathrm{Ob},\mathrm{Mor})$ 称为一个\textbf{范畴}。常记 $\mathcal C=(\mathrm{Ob}(\mathcal C),\mathrm{Mor}(\mathcal C))$。
\end{definition}

\begin{remark}
    $\mathrm{Ob}(\cate{Set})$ 是真类。对象及其态射常记作 $f:X\to Y$ 或 $X\xrightarrow{f} Y$ 的形式。
\end{remark}

图表加箭头是讨论范畴的方便语言。用箭头表达态射,用箭头的头尾衔接表示复合映射。最常用的是\textbf{交换图}(commutative diagram),“交换”意指箭头的复合殊途同归,如
    \[ \begin{tikzcd}
        X \arrow[r, "f"] \arrow[rd, "h"'] & Y \arrow[d, "g"] \\
        & Z
    \end{tikzcd} \qquad \begin{tikzcd}
        A \arrow[r, "u"] \arrow[d, "x"'] & B \arrow[d, "v"] \\
        C \arrow[r, "y"'] & D
    \end{tikzcd} \]
的交换性分别等价于 $g f = h$ 和 $v u = y x$。态射的名称若自明或不重要,则常从图中略去。

\begin{definition}
    对 $X \xrightarrow{f} Y$,若存在 $Y \xrightarrow{g} X$ 使得 $f g = \id_Y,g f = \id_X$,则称 $f$ 是(对象)\textbf{同构}(isomorphism),而 $g$ 称为 $f$ 的\textbf{逆},从恒等态射的性质易见逆若存在则唯一,记 $f^{-1}$。称 $X,Y$ \textbf{互为同构},记作 $X\cong Y$。从 $X$ 到 $Y$ 的同构集记作 $\mathrm{Isom}(X, Y)$。

    $\operatorname{End} X := \hom(X, X)$,$\operatorname{Aut} X := \mathrm{Isom}(X, X)$ 分别称作 $X$ 的\textbf{自同态集}(set of endomorphisms)和\textbf{自同构集}(set of aut)。这些集合在二元运算 $\circ$ 下封闭,或用代数的语言来说:$\operatorname{Aut} X$ 是群,$\operatorname{End} X$ 是幺半群。若忘记其定义可见下节。
\end{definition}

% 自同构英语

给定一个范畴,对象不变但反转所有箭头,则反转后得到的东西仍然是一个范畴,这种对称性也称作\textbf{对偶原理}:
\begin{definition}
    范畴 $\mathcal{C}$ 的\textbf{反范畴}(opposite category)或\textbf{对偶范畴} $\mathcal{C}^\mathrm{op}$ 定义如下:
    \begin{itemize}
        \item $\mathrm{Ob}(\mathcal{C}^{\mathrm{op}}) = \mathrm{Ob}(\mathcal{C})$;
        \item 对任意对象 $X, Y$,$\hom_{\mathcal{C}^{\mathrm{op}}}(Y, X) := \hom_{\mathcal{C}}(X, Y)$;
        \item 恒等态射同 $\mathcal{C}$:$\id^\mathrm{op}_X:=\id_X$;
        \item 对任意态射 $g \in \hom_{\mathcal{C}^{\mathrm{op}}}(Z, Y), f \in \hom_{\mathcal{C}^{\mathrm{op}}}(Y, X)$,$f \circ^\mathrm{op} g:=gf$。交换图为
        \[
\begin{tikzcd}
X \arrow[r,"f"] \arrow[dr,"g f"'] & Y \arrow[d,"g"] \\
& Z
\end{tikzcd}
\im
\begin{tikzcd}
X & Y \arrow[l,"f"'] \\
& Z \arrow[u,"g"'] \arrow[ul,"f\circ^\mathrm{op} g"].
\end{tikzcd}
\]
    \end{itemize}
    容易验证 $\mathcal{C}^\mathrm{op}$ 满足范畴定义,且 $(\mathcal{C}^\mathrm{op})^{\mathrm{op}} = \mathcal{C}$。
\end{definition}
在处理许多范畴论性质时,善用对偶原理能省事不少。

一个范畴往往代表一套理论。研究范畴间的关系可进一步抽象出如下概念:
\begin{definition}
    设 $\mathcal{C}', \mathcal{C}$ 为范畴。映射 $F: \mathcal{C}' \to \mathcal{C}$ 称为\textbf{函子}(functor),包括:
    \begin{enumerate}
        \item 对象间的映射 $F: \mathrm{Ob}(\mathcal C')\to\mathrm{Ob}(\mathcal C)$.
        \item 态射间的映射 $F: \mathrm{Mor}(\mathcal C')\to \mathrm{Mor}(\mathcal C)$, 使得
            \begin{itemize}
                \item $F$ 与来源和目标映射相交换 (即 $sF=Fs$, $tF=Ft$),或等价地,对每个 $X, Y \in \mathrm{Ob}(\mathcal{C}')$ 皆有映射 $F: \hom_{\mathcal{C}'}(X, Y) \to \hom_{\mathcal{C}}(F(X), F(Y))$。换言之,$f\in \hom_{\mathcal{C}'}(X, Y)\Rightarrow F(f)\in \hom_{\mathcal{C}}(F(X), F(Y))$;
                \item $F(g\circ^{\prime} f) = F(g) \circ F(f)$,$F(\id_X) = \id_{F(X)}$。
            \end{itemize}
    \end{enumerate}
    对于 $F: \mathcal{C}_1 \to \mathcal{C}_2,G: \mathcal{C}_2 \to \mathcal{C}_3$,\textbf{复合函子} $G \circ F: \mathcal{C}_1 \to \mathcal{C}_3$ 的定义显然是取相应复合映射:$\mathrm{Ob}(\mathcal{C}_1) \xrightarrow{F} \mathrm{Ob}(\mathcal{C}_2) \xrightarrow{G} \mathrm{Ob}(\mathcal{C}_3)$ 和 $\mathrm{Mor}(\mathcal{C}_1) \xrightarrow{F} \mathrm{Mor}(\mathcal{C}_2) \xrightarrow{G} \mathrm{Mor}(\mathcal{C}_3)$。
\end{definition}



\begin{remark}
	从 $\mathcal{C}'$ 到 $\mathcal{C}$ 和从 $(\mathcal{C}')^\mathrm{op}$ 到 $\mathcal{C}^\mathrm{op}$ 的函子是一回事。为资区分,对于函子 $F: \mathcal{C}' \to \mathcal{C}$,反范畴间的相应函子记为 $F^\mathrm{op}: (\mathcal{C}')^\mathrm{op} \to \mathcal{C}^\mathrm{op}$。
\end{remark}

函子又称\textbf{协变函子},图解为
\[
\begin{tikzcd}
X \arrow[r,"f"] \arrow[dr,"g\circ^{\prime} f"'] & Y \arrow[d,"g"] \\
& Z
\end{tikzcd}
\im
\begin{tikzcd}
F(X) \arrow[r,"F(f)"] \arrow[dr,"F(g\circ^{\prime} f)"'] & F(Y) \arrow[d,"F(g)"] \\
& F(Z).
\end{tikzcd}
\]
$C'$ 的\textbf{逆变函子}是 $(\mathcal{C}')^\text{op}$ 的函子 $F: (\mathcal{C}')^\text{op} \to \mathcal{C}$,图解为
\[
\begin{tikzcd}
X \arrow[r,"f"] \arrow[dr,"g\circ^\prime f"'] & Y \arrow[d,"g"] \\
& Z
\end{tikzcd}
\im
\begin{tikzcd}
X & Y \arrow[l,"f"'] \\
& Z \arrow[u,"g"'] \arrow[ul,"f\circ^\mathrm{op} g"]
\end{tikzcd}
\im
\begin{tikzcd}
F(X) & F(Y) \arrow[l,"F(f)"'] \\
& F(Z). \arrow[u,"F(g)"'] \arrow[ul,"F(g\circ^\prime f)"]
\end{tikzcd}
\]
也就是将函子定义中态射的性质改为 $F(Y)\xrightarrow{F(f)} F(X),F(g\circ^\prime f) := F(f) \circ F(g)$。
取逆映射进行复合可知,若 $f$ 是同构则 $F(f)$ 也是。

研究函子间的态射可进一步抽象:
\begin{definition}
    函子 $F, G: \mathcal{C}' \to \mathcal{C}$ 之间的\textbf{自然变换}(natural transformation) $\theta$ 是一族态射 $\{\theta_X\}$,其中 $\theta_X \in \hom_{\mathcal{C}}(F(X), G(X))$ 且 $X \in \mathrm{Ob}(\mathcal{C}')$,使得下图对所有 $\mathcal{C}'$ 中的态射 $f: X \to Y$ 交换:
    \begin{equation}\begin{tikzcd}
        F(X) \arrow[r, "\theta_X"] \ar[d, "F(f)"'] & G(X) \arrow[d, "G(f)"] \\
        F(Y) \arrow[r, "\theta_Y"] & G(Y) .
    \end{tikzcd}\end{equation}
    $F(X)$ 可简写为 $FX$。
    自然变换写作 $\theta: F \Rightarrow G$,或图解为
    \[ \begin{tikzcd}
        \mathcal{C}' \arrow[bend left=50, r, "F", ""' name=U] \arrow[bend right=50, r, "G"', "" name=D] & \arrow[Rightarrow, to path=(U) -- (D) \tikztonodes, "\theta"] \mathcal{C}.
    \end{tikzcd} \]
    上述带有双箭头的图表有时也被称为 \textbf{2-胞腔}(2-cell)。这样 $\theta$ 可称为从函子 $F$ 到 $G$ 的态射。
\end{definition}


实用中经常会省略严格的范畴论框架,只说态射 $\theta_X: FX \to GX$ 对变元 $X$ 是\textbf{自然的}或\textbf{典范的}(canonical),或称满足\textbf{函子性}。对于\textbf{自然同构}。实践中经常把自然同构直接写成等号,即 $F X= G X$。

任意函子 $F$ 到自身有恒等态射 $\id_F : F \Rightarrow F$。对于 $\theta: F_1 \Rightarrow F_2$,若 $\psi: F_2 \Rightarrow F_1$ 满足 $\psi \circ \theta = \id_{F_1}$,$\theta \circ \psi = \id_{F_2}$,则称 $\psi$ 是 $\theta$ 的逆。可逆的自然变换称为\textbf{函子同构}。由定义直接看出 $\theta$ 的逆若存在则是唯一的,记作 $\theta^{-1}$,它无非是在范畴中\textit{逐点}取逆:$(\theta^{-1})_X := (\theta_X)^{-1}: F_2 X \to F_1 X$。同理可见 $\theta$ 可逆当且仅当任意 $\theta_X$ 可逆。故函子 $F_1,F_2$ 同构等价于对任意变元 $X$ 有 $F_1 X= F_2 X$。

更深层次的态射纳入到了\textbf{高维范畴论}的领域。

\begin{definition}
    从 $X$ 到 $Y$ 的全体映射构成\textbf{映射集}或\textbf{指数对象} $Y^X:=\hom_{\cate{Set}}(X,Y)$。这个记法是注意到 $\hom_{\cate{Set}}(X,Y)$ 的势可以写成 $|Y|^{|X|}$。
\end{definition}

\begin{definition}
hom-函子    
\end{definition}



\begin{definition}
映射 $f:X\to Y$ 诱导出\textbf{拉回}(pull-back)映射 $f^*:\mathbb F^Y\to\mathbb F^X$,它将任意 $Y$ 上函数 $\lambda$ 反向拉回到 $X$ 上(这里将 $f^*(\lambda)$ 简写为 $f^*\lambda$):
\eq{
    f^*\lambda:=\lambda\circ f.
}
\textbf{推前}(push-forward) $f_* v:=v\circ f^*$ 将任意 $X$ 上泛函 $v$ 沿 $f$ 推到 $Y$ 上。

集合 $X$ 的\textbf{对偶}是指 $X^*:=\mathbb F^X$,故 $f\in\hom(X,Y)$ 诱导出拉回 $f^*\in\hom(Y^*, X^*)$。
容易验证 $\id_X^*=\id_{X^*}$ 且 $(f\circ g)^*=g^*\circ f^*$。因此可用 $X\mapsto X^*,f\mapsto f^*$ 构成(逆变)函子 $(\oo)^*: \cate{Set}^\mathrm{op}\to \cate{Set}$,称为\textbf{对偶函子}。
\end{definition}

\begin{remark}
    不可逆的 $f$ 只能对泛函定义推前。
    $f$ 可逆时才能直接对函数 $\lambda:X\to\mathbb F$ 定义 $f_*\lambda:=\lambda\circ f^{-1}$,也即 $f_*:=(f^{-1})^*$,这样也有 $f^*=(f^{-1})_*$。
\end{remark}


