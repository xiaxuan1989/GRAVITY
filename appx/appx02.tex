\chapter{点集拓扑}\label{appx:topo}

18 世纪,Euler 等人发现简单多面体的顶点数 $v$、楞数 $e$、面数 $f$ 总是满足 $v-e+f=2$。
具体地,简单多面体可连续变化成球面。用投影来模拟这个过程:多面体内应存在一种位置,在此处放一盏灯,各顶点和棱都投影在一个外部球面上,这些棱的投影曲线彼此不穿过。比如凸多面体和某些凹多面体,但像厚球壳那样带腔、像甜甜圈那样带洞的多面体是不行的。
这种变化还可形象地视作捏橡皮膜(rubber-sheet),拓扑学因而又称为\textbf{橡皮几何}。的确,自然界存在直觉上形状相似的几何体,为抽象出共性,可试着“揉”它但不撕裂或粘帖,观察是否能得到另一几何体,如从正方体到球体。这种连续变化间的等价性称作\textbf{同胚}(homeomorphism)。$v-e+f$就是一个多面体在同胚下保持不变的量,即\textbf{拓扑不变量}。
进而我们能将各几何体分门别类,注意力集中于诸如“是否有内外”“是否有洞”“绳上打了几个结”这类问题上的,从而架空距离、面积这些传统几何概念。一个典型例子是,对于理想电路,无论接线长度、方向如何,只要节点不变,电路网络就是等价的。

\begin{figure}[ht]
    \centering
    \includegraphics[width=.8\textwidth]{fig/appx/Mobius.png}
    \caption{M\"obius 环}\label{mobius}
\end{figure}

必须指出,用橡皮形变来比喻同胚仍有其局限性。一条纸带不经扭曲,可直接首尾连接得到手环,它具有两个面。扭转半周再相连将得到 M\"obius 环,它只有一个面。扭转一周再相连则又恢复为两个面。这个新环与最初的手环同胚,但却不能通过“揉捏”的方式恢复。如果我们注意到曲面的维度比背景空间的低,而只想关注曲面的内禀性质,则两种手环将视为同⼀空间在 $\R^3$ 内的不同表⽰。所谓的“揉捏”法,实质是 $\R^3$ 的⾃同胚。单看⼆者固然可建⽴同胚,但这种同胚⽆法扩张到整个 $\R^3$ 上。换言之,不存在从 $\R^3$ 的自同胚把两种手环映射在一起,这就解释了问题。这警示我们应更为抽象地看待拓扑。

正如上述例子一样,19 世纪末,人们欲抛掉坐标系而寻求一般集合的拓扑理论。为此要回到同胚的特征:拓扑变化的连续性借助于实数上的 $\epsilon$-$\delta$ 语言,故我们应给出一般集合的连续表述。这一时期可以称为拓扑的分析化,孕育出了\textbf{点集拓扑}。可以说此即最普适的拓扑学。构造这一理论花了很长时间,但我们可简述其关键:极限定义用到了实数集上的小于(序结构)和减法(度量结构),这相当于规定哪类子集是某点邻域、开集和闭集。
因此,拓扑学所研究的对象可以看作脱胎于距离的一种更广义的空间。将这些关键概念抽离出来后,我们就能在拓扑框架下重新得到分析学的定义、命题。现代主流拓扑学正是从这些分析概念出发,一步步走向具象的橡皮几何。
更为深刻的话题将涉及\textbf{代数拓扑}。这一专题将在关于\textbf{上同调}(cohomology)的部分中得以处理,在讨论拓扑分类、积分理论时 de Rham 上同调比较好用。

\section{拓扑空间}

$2^X$ 的子集称为\textbf{子集族}(collection of subsets),全体子集族构成 $2^{2^X}$。欲对 $X$ 中每个点挑选⼀族⼦集作为邻域(neighbourhood),就是给定映射 $\topo N:X\to 2^{2^X}$ 且 $\topo N(x)\ne \mt$。
直观上,邻域只需含有 $x$ 即可,但由于大小任意,可无限接近于 $x$,故有“邻”称。
若 $A\in \topo N(x)$,则 $x$ 为 $A$ 的内点(interior point),而 $A$ 中全体内点之集 $A^\circ$ 就是其内部(interior)。

希望从分析学中抽取邻域的若干性质,但不能太多,因为要普适到能包括不同于通常定义的情形。新表述不能依赖距离概念。关键在于,注意属于、包含等集合术语,已能描绘距离所表达的含义:$x$ 在其任意邻域内;$x$ 的任意两邻域之交也是 $x$ 的邻域;$x$ 的任意邻域的内部也是其邻域;包含 $x$ 某邻域的集合也是其邻域。
有了邻域,其它概念都好办了:在某全集 $X$ 下,开集(open set)是满足 $A=A^\circ$ 的集合,闭集(closed set)是其补集为开的集合。因此 $\mt,X$ 是既开又闭的。根据 $\epsilon$-$\delta$ 定义,$f:X\to Y$ 在 $x\in X$ 连续,即 $\forall\epsilon>0$ 都 $\exists\delta>0$,使得 $x$ 的 $\delta$ 邻域的像含于 $f(x)$ 的 $\epsilon$ 邻域。

往往直接考察 $f$ 整体的连续性,但邻域毕竟是就某 $x\in X$ 而言的,可预料从开集出发语言将更优雅。非空开集总是某点的某邻域,而邻域的内部必开,故 $f$ 的连续性表述不必出现邻域概念。有两点考虑:$f$ 不一定是满射,但空集可作为 $Y\backslash\Im f$ 的逆像;$f$ 不一定是单射,满足 $f(z)=f(x)$ 的 $z$ 可以很多,但 $f$ 的连续性要求处处连续,因此 $f(x)$ 任意邻域的逆像一定是所有 $z$ 的某邻域之并。综上,$f$ 连续即 $Y$ 中任意开集之逆是 $X$ 的开集。

由于包含邻域之集也是邻域,因此任意开集之并为开;而两邻域之交为邻域,经由归纳法,只能保证有限个开集之交为开;$\mt,X$ 是开的,$x\in X$ 的邻域则是能包含某个含有 $x$ 的开集的集合。这几条性质显然能导出邻域体系,故为等价表述。

\begin{definition}
$\topo{T}\subset 2^X$ 称为 $X$ 的一个\textbf{拓扑}(topology),若:
\begin{itemize}
    \item $\mt,X \in \topo{T}$;
    \item $\forall \sigma\subset \topo{T}$,$\bigcup_{U\in\sigma} U\in \topo{T}$;
    \item $U,V\in\topo{T}\im U\cap V\in\topo{T}$。或等价地对有限集 $\{A_{i}\} \subset \topo{T}$,$\bigcap_{i} A_{i} \in \topo{T}$。
\end{itemize}
$\topo{T}$ 的元素称为 $X$ 在 $\topo{T}$ 下的\textbf{开子集},简称\textbf{开集}。
$(X,\topo T)$ 称为 $X$ 关于拓扑 $\topo T$ 的\textbf{拓扑空间}(topological space)。拓扑空间的元素称为\textbf{点}(point)。
\end{definition}

\begin{definition}
    对$x\in X,U\subset X$,若 $\exists O\in\topo{T}$ 使 $x\in O\subset U$,则 $U$ 是 $x$ 的一个\textbf{邻域}。$x$ 的全体邻域构成 $\topo N(x)$。邻域为开则称\textbf{开邻域},构成 $\topo N(x)\cap\topo T$。
\end{definition}

\begin{theorem}
    $\exists U \in\topo N(x)$ 如何 $\iff \exists$ 开集 $O\ni x$ 如何 $\iff \exists V \in\topo N(x)\cap\topo T$ 如何。
\end{theorem}
\begin{proof}
    $\exists U\in\topo N(x)\im \exists O\in\topo T$ 使 $x\in O\subset U\im \exists$ 开集 $O\ni x\im\exists O\in\topo T$ 使 $x\in O\subset O\im \exists V=O \in\topo N(x)\cap\topo T\im\exists U=V\in\topo N(x)\cap\topo T\subset\topo N(x)$。
\end{proof}
\begin{remark}
    凡命题中出现“点存在邻域…”均可换成“点存在开邻域”“包含该点的开集”。
\end{remark}

%% \setminus or \backslash
\begin{definition}
    $A\subset X$ 称为\textbf{闭集},若 $X\backslash A\in\topo{T}$。
\end{definition}

$\topo T$ 的名称若自明或不重要,则拓扑空间常略写为$X$。讨论开集时也常省略全集。De Morgan 律指出:并集之补等于补集之交,因此闭集也能给出拓扑的定义。容易得出等价表述为:$\mt,X$ 是闭的;有限闭集之并为闭;任意闭集之交为闭。

拓扑的选取有很多。若 $\topo T_1\subset\topo T_2$,则称 $\topo T_1$ \textbf{较粗}(coarser)或 $\topo T_2$ \textbf{较细}(finer)。直观上,较细拓扑容纳更多开集,能区分更多点。\textbf{凝聚}(indiscrete)\textbf{拓扑} $\topo T=\{\mt,X\}$ 最粗。\textbf{离散}(discrete)\textbf{拓扑} $\topo T= 2^X$ 最细。
分析学中,$\mt$ 和能表为开区间之并的集合都是 $\R$ 上开集,构成 $\R$ 的\textbf{通常}(usual)\textbf{拓扑} $\topo T_u$。视 $x,y\in\R$ 的距离为 $|x-y|$,则开区间实质是以 $\frac{x+y}{2}$ 为心的一维球。$(\R,\topo T_u)$ 称为\textbf{实直线}。通常拓扑当然比离散、凝聚拓扑更直观,后者一般用于构造反例。

\begin{definition}
    对 $(X,\topo T_X),(Y,\topo T_Y)$,$f: X \to Y$ \textbf{连续},若 $\forall A\in\topo T_Y$,$f^{-1}[A]\in \topo T_X$。以拓扑空间为对象、连续映射为态射可构建\textbf{拓扑空间范畴} $\cate{Top}$。
\end{definition}

\begin{definition}
    双向连续的双射称作\textbf{同胚}。显然同胚时拓扑同势,即开集结构等同。
\end{definition}

\begin{remark}
    同胚的逆也必须连续。
    考虑复平面和 $f(x)=\e^{\i x},x\in [0,2\pi)$。则 $\Im f$ 为单位圆周 $|z|=1$。它是连续双射,但逆不连续。而直观上圆周与区间不能同胚。
\end{remark}

\begin{definition}
    映射 $f:X\to Y$ \textbf{保持}某种性质,就是若 $A\subset X$ 具有这种性质,则 $f[A]$ 也具有。\textbf{开映射}就是保持开的映射。闭映射定义同理。
\end{definition}

\begin{theorem}
    同胚是开映射。
\end{theorem}

\begin{proof}
取任意开集 $U$,同胚 $f:U\to f[U]$。因 $f^{-1}$ 连续,开集 $f^{-1}[f[U]] = U$ 的原像 $f[U]$ 也开。同理同胚也是闭映射。
\end{proof}

\begin{definition}
置 $(X,\topo T)$ 和 $A\subset X$。
\begin{itemize}
    \item \textbf{内部}为开集 $A^\circ:=\bigcup_{O\in\topo T\cap 2^A} O=\{x: \text{$\exists$ 开集 $O\subset A$ 使 $x\in O$}\}$,也即含于 $A$ 的最大开集。
    \item \textbf{闭包}(closure)为闭集 $\bar{A}:=\bigcap_{X\backslash B\in\topo T,A\subset B} B=\{x: \text{$\forall$ 闭集 $B\supset A$ 都 $x\in B$}\}$,也即包含 $A$ 的最小闭集。
    \item $\partial A=\bar{A}\backslash A^\circ$ 称为 $A$ 的\textbf{边界}(boundary)。
\end{itemize}
\end{definition}

\begin{remark}
    显然 $A^\circ\subset A\subset \bar A$。$A^\circ\cup\partial A=\bar A$,$A^\circ\cap\partial A=\mt$。
    
    $\partial A=\mt\iff A^\circ=A=\bar A\iff A$ 既开又闭。
\end{remark}

\begin{theorem}
    $X\backslash A^\circ=\overline{X\backslash A}$,$(X\backslash A)^\circ=X\backslash \bar A$。
\end{theorem}
\begin{proof}
    $A^\circ:=\bigcup_{X\backslash F\in\topo T,X\backslash F\subset A} X\backslash F=X\backslash\bigcap_{X\backslash F\in\topo T,X\backslash A\subset F} F=X\backslash\overline{X\backslash A}$。
    替换即可。
\end{proof}

\begin{theorem}
    $A\in\topo T\eqto A=A^\circ\eqto\forall x\in A,A\in\topo N(x)\eqto\forall x\in A,\exists U\in\topo N(x)$ 使 $U\subset A$。
\end{theorem}

\begin{proof}
    $A\in\topo T,A\subset A\im A\subset A^\circ\iff A=A^\circ$;反之 $A=A^\circ\in\topo T$。

    $A=A^\circ\iff\forall x\in A,\exists O\in\topo T$ 使 $x\in O\subset A\iff\forall x\in A, A\in\topo N(x)\im \forall x\in A,\exists U=A\in\topo N(x)$ 使 $U\subset A\im \forall x\in A,\exists O\in\topo T$ 使 $ x\in O\subset U\subset A \im A=A^\circ$。
\end{proof}

\begin{theorem}
    $A$ 闭 $\iff A=\bar A$
\end{theorem}
\begin{proof}
    $A$ 闭 $\iff X\backslash A=(X\backslash A)^\circ=X\backslash\bar A\iff A=\bar A$。
\end{proof}

\begin{theorem}
    \textbf{内点}:$x\in A^\circ\iff\exists$ 开集 $O\ni x$ 使 $O\subset A^\circ\subset A$;
    \textbf{外点}:$x\notin \bar A\iff\exists$ 开集 $O\ni x$ 使 $O\subset X\backslash\bar A\subset X\backslash A$;
    \textbf{边界点}:$x\in\partial A\iff\forall$ 开集 $O\ni x$ 与 $A,X\backslash A$ 都交。
\end{theorem}
\begin{proof}
    注意 $A^\circ\in\topo T,X\backslash \bar A = (X\backslash A)^\circ\in\topo T$ 即可。并注意 $O\cap A=\mt\iff O\subset X\backslash A$。
\end{proof}

\begin{remark}
    立即可得 $x\in\bar A\iff\forall$ 开集 $O\ni x,O\cap A\ne\mt$。
\end{remark}

\begin{definition}
置 $A\subset X$。\textbf{导集}(derived set)为 $A':=\{x \in X:x\in\overline{A\backslash\{x\}}\}$,其元素称为\textbf{聚点}(accumulation point)或\textbf{极限点}。
$\bar{A}=X$ 时称 $A$ 在 $X$ 中\textbf{稠密}(dense)。
\end{definition}
\begin{remark}
    $x\in A'\eqto \forall U\in\topo N(x),(U\cap A)\backslash\{x\}=U\cap (A\backslash\{x\})\ne\mt$。$U$ 可等价换为任意含 $x$ 开集。
\end{remark}
\begin{eg}
    考虑实直线,$0,1/2,1$ 都是 $[0,1)$ 的聚点。
\end{eg}

\begin{theorem}
    $\bar A=A\cup A'$。  
\end{theorem}

\begin{proof}
    任给 $x\in A'$。$\forall U\in\topo N(x),(U\cap A)\backslash\{x\}\ne\mt\Rightarrow \forall U\in\topo N(x),U\cap A\ne\mt\eqto x\in\bar A$。说明 $A'\subset \bar A$,此 $A\cup A'\subset\bar A$;
    任给 $x\in\bar A\backslash A'$。$x\in\bar A,x\notin A'\eqto \exists U\in\topo N(x),(U\cap A)\backslash\{x\}=\mt$ 且 $U\cap A\ne\mt\Rightarrow U\cap A=\{x\}\Rightarrow x\in A$。说明 $\bar A\backslash A'\subset A\iff \bar A\subset A\cup A'$。
\end{proof}

\begin{theorem}
    $A$ 闭 $\iff A'\subset A$。
\end{theorem}
\begin{proof}
    $A$ 闭 $\iff A=\bar A\iff A=A\cup A'\iff A'\subset A$。
\end{proof}

\begin{definition}
    设拓扑空间 $X$ 上的映射 $f: X \to Y$,$Y$ 是一个带有零元 $0$ 的集合($\R,\C$ 或某个拓扑群、向量空间)。那么:
    \begin{itemize}
        \item 零集:$Z(f):=f^{-1}\{0\}=\{x\in X: f(x)=0\}$;
        \item 非零集(cozero set):$\operatorname{coz}f:=X\backslash Z(f)=\{x\in X: f(x)\ne 0\}$;
        \item \textbf{支集}(support):$\supp f:=\overline{\operatorname{coz}f}
  = X\backslash (Z(f))^\circ$。
    \end{itemize}
    更方便地,要求 $Y$ 是拓扑空间且 $\{0\}$ 闭($Y=\R,\C$ 当然满足),从而可谈及连续 $f$。这样 $Z(f)=f^{-1}\{0\}$ 闭,$\operatorname{coz} f$ 开,$\supp f$ 闭。
\end{definition}

\begin{theorem}
     $x\notin \supp f\iff\exists$ 开集 $U\ni x$ 使 $f[U]=\{0\}$。
\end{theorem}
\begin{proof}
    首先 $U\subset Z(f)\iff f[U] \subset \{0\}$。存在元素 $x\in U$,故 $\mt\ne U\subset Z(f)\iff f[U]=\{0\}$,而 $x\in X\backslash\supp f = (Z(f))^\circ \iff \exists$ 开集 $U\ni x$ 使 $U\subset Z(f)$。
\end{proof}
\begin{theorem}
    $\supp f \subset O\subsetneq X\im f[X\backslash O]=\{0\}$。反之需要 $O$ 闭的条件。
\end{theorem}
\begin{remark}
    $\supp f \subset O$ 称为 $f$ 支撑在 $O$ 上。
\end{remark}

\begin{proof}
    $\mt\ne X\backslash O\subset (Z(f))^\circ\subset Z(f)$。
    反之 $\mt \ne X\backslash O\subset Z(f)\xRightarrow{X\backslash O\in\topo T} X\backslash O\subset (Z(f))^\circ$。
\end{proof}

\begin{definition}
    $S:\N\to X,n\mapsto x_n$ 的值域 $\Im S=\left\{x_n\right\}$ 称为\textbf{序列}或\textbf{点列}。
    若 $\exists x \in X$,$\forall U\in\topo N(x),\exists N\in\N$,使 $n>N\im x_n\in U$,则称 $\{x_n\}$ 有\textbf{极限}或\textbf{收敛}于 $x$。
\end{definition}

\begin{remark}
    点列聚点的任意邻域都含点列的无限个点。极限是聚点,但聚点不一定是极限。
\end{remark}

\begin{definition}
    在 $(X,\topo T)$ 下,可定义 $A \subset X$ 的\textbf{相对}(relative)\textbf{拓扑}或\textbf{诱导}(induced)\textbf{拓扑} $\topo T_r=\{O \cap A: O\in\topo T\}$。赋予诱导拓扑的子集称为原集的\textbf{拓扑子空间}(topological subspace)。
\end{definition}

\begin{eg}
    对于实直线和 $A=[0,2]$,在 $\topo T_u$ 的诱导下 $B=(1,2]$ 是 $A$ 中的开集。
\end{eg}

\begin{definition}
    给定 $(X,\topo T_X),(Y,\topo T_Y)$ 和 $Z=X \times Y$。
    定义
    \[\topo T_Z=\{O\subset Z:A\in\topo T_X,B\in\topo T_Y,\text{$O$ 可表为形如 $A\times B$ 的集合之并}\},\]
    $(Z,\topo T_Z)$ 称为 $X,Y$ 的\textbf{乘积拓扑空间}(product topological space)。
\end{definition}

这与 $\R$ 的通常拓扑类似,都是通过囊括空集和“任意并”操作来生成拓扑。这一方法可总结如下。
\begin{definition}
    置 $(X,\topo T)$。
    $\topo C\subset 2^X$ 称为 $A\subset X$ 的\textbf{覆盖}(cover),若 $A\subset\bigcup_{O\in\topo C}O$($A=X$ 时显然取等)。$\topo C\subset\topo T$ 时称\textbf{开覆盖}。
    $\topo S\subset\topo C$ 是 $\topo C$ 的\textbf{子覆盖}(subcover),若它也是覆盖。
\end{definition}

\begin{definition}
    取 $X$ 的覆盖 $\topo{B}$。若它保持“有限交”性质:对任意 $U_1,U_2\in\topo{B}$,$\exists\topo{F}\subset \topo{B}$ 使 $\bigcup_{U\in\topo{F}}U=U_1\cap U_2$,就称$\topo{B}$ 为\textbf{拓扑基}。可以证明
$\topo{T}=\left\{\bigcup_{U\in\topo{F}}U:\topo{F}\subset\topo{B}\right\}$ 构成拓扑,称为 $\topo{B}$ 的\textbf{生成拓扑}。\textbf{拓扑子基}的元素所有可能的有限交构成拓扑基。
\end{definition}
\begin{theorem}
    $\topo{B}\subset\topo T$,因而是开覆盖。
\end{theorem}

拓扑基是挑出⼀部分开集来代表所有开集,类似地也有一种挑出⼀部分邻域来代表所有邻域的办法。

\begin{definition}
    $x\in X$ 的\textbf{邻域基}或\textbf{局部基} $\topo B_x\subset\topo N(x)$ 满足 $\forall O\in\topo N(x)$,$\exists B\in\topo B_x$ 使 $B\subset O$。
\end{definition}

许多空间往往都难以直接想象或处理,我们可考虑将它分解为简单直观的部分,再重组回去,就能描述复杂的拓扑空间。

\begin{definition}
    设 $\tilde X=\{[x]:x\in X\}$ 是 $(X,\topo T)$ 在某个等价关系下的分割或商集(quotient set)。
    对其规定如下拓扑:映射 $\pi: X \to \tilde X,x\mapsto [x]$ 称为\textbf{自然映射}或\textbf{典范投影},取 $\tilde{\topo T}=\{U \subset Y:\pi^{-1}(U)\in\topo T\}$,称为\textbf{商拓扑},$(\tilde X,\tilde{\topo T})$ 称为\textbf{商空间}。
\end{definition}
可见,$\pi$ 就好比用来粘合拓扑空间的胶水。
可以证明,设 $\tilde{\topo T}^{\prime}$ 为 $\tilde X$ 上另一个拓扑,且使 $\pi$ 连续,则 $\tilde{\topo T}^{\prime} \subset \tilde{\topo T}$。

\begin{definition}
    拓扑空间\textbf{局部地}具有某个性质,意指任意点都存在邻域具有这种性质。
\end{definition}

\section{可数与分离}
本节目的是给出物理上足够有用的空间,使拓扑性质足够好,不至于太怪异。

在拓扑学中针对一堆研究对象谈它们的可数性时,并不一定是指这些对象的全体可数,因这个要求不是很容易达到。很多时候可数性会定义成:从这堆对象中可以选出可数多个代表来“刻画出”其他所有对象。邻域/拓扑基就是从邻域/开集里面选出来的“代表团”。
和以后要讲的连通性、紧致性不同,直接检查可数性来证明两个空间不同胚的例子并不多见,可数性对于一些数学技巧(特别是数学归纳法)是否能应用至关重要,所以点集拓扑学中有好几个著名定理,如果去掉可数性的前提条件就证明不出来了。用邻域基、拓扑基可定义出两种\textbf{可数定理}。

\begin{definition}\textbf{第一可数}空间的任意点存在可数邻域基,\textbf{第二可数}空间存在可数拓扑基,各记 $C_1,C_2$。字母 C 来⾃于英语“countable”。
\end{definition}

\begin{theorem}
    $C_2$ $\im$ \textbf{Lindel\"of}:任意开覆盖都存在可数子覆盖。
\end{theorem}

\begin{proof}
    对任意开覆盖 $\topo U$,对每个基元素 $B$(来自可数基)如果它被某个 $U\in\topo U$ 包含,就挑一个这样的 $U_B$。因为基是可数的,所以被挑出来的 $U_B$ 至多可数;再用“每个点都落在某个基元素里”推出这些 $U_B$ 仍覆盖全空间,于是得到可数子覆盖。

取$C_2$ $X$ 的一个可数基 $\topo B=\{B_n\}_{n\in\mathbb N}$。令 $\topo U$ 为 $X$ 的任意开覆盖。
对每个 $n$,若存在 $U\in\topo U$ 使得 $B_n\subset U$,则任选一个这样的开集记为 $U_n\in\topo U$;
若不存在这样的 $U$,则忽略该 $n$(不选取任何东西)。

记所有被选中的开集所成的集合为
\[
\topo U':=\{U_n:\ B_n\subset U_n\text{ 且 }U_n\in\topo U\}.
\]
由于 $\topo B$ 可数,故 $\topo U'$ 至多可数。

下面证明 $\topo U'$ 覆盖 $X$。任取 $x\in X$,由 $\topo U$ 覆盖,存在 $U\in\topo U$ 使 $x\in U$。
又由于 $\topo B$ 为基,存在某个基元素 $B_n$ 满足
\[
x\in B_n\subset U.
\]
由构造知此时 $B_n$ 必会触发选择,于是存在被选中的 $U_n\in\topo U'$ 使得 $B_n\subset U_n$。
从而 $x\in B_n\subset U_n$,说明 $x$ 落在 $\bigcup\topo U'$ 中。
因此 $\topo U'$ 是 $\topo U$ 的可数子覆盖,证毕。
\end{proof}

\begin{theorem}
    $C_2 \im C_1$。
\end{theorem}

再讨论分离。分离意指是否可用拓扑结构(开集、连续映射)区分空间中的两个原本不应重叠的部分。按照对“不重叠两部分”的理解不同、区分方法的不同,有很多种不同判定方法和条件,统称\textbf{分离公理}(separation axiom)。
常用的列举如表 \ref{tab:separation}。字母 T 来⾃于分离公理的德语“das Trennungsaxiom”。
\begin{table}[ht]
    \centering
    \caption{分离公理}
    \label{tab:separation}
    \begin{tabular}{lll}
        \toprule
        符号 & 别称 & 定义\\
        \midrule
        $T_0$ & Kolmogorov & $\forall x \ne y$, $\exists$ 开集 $U$ 使 $(x \in U, y \notin U)\lor (y \in U, x \notin U)$ \\
        $T_1$ & Fréchet & $\forall x \ne y$, $\exists$ 开集 $U, V$ 使 $(x \in U, y \in V)\land (y \notin U , x \notin V)$ \\
        $T_2$ & Hausdorff & $\forall x \ne y$, $\exists$ 不交开集 $U, V$ 使 $x \in U, y \in V$ \\
        $T_3$ & 正则 $^*$ & $T_1$ 并且 $\forall$ 闭集 $F$ 及 $x \notin F$,$\exists$ 不交开集 $U, V$ 使 $x \in U, F \subset V$ \\
        $T_4$ & 正规 $^*$ & $T_1$ 并且 $\forall$ 不交闭集 $A, B$, $\exists$ 不交开集 $U, V$ 使 $A \subset U, B \subset V$ \\
        $T_5$ & 完全正规 $^*$ & 任意子空间都 $T_4$ \\
        \bottomrule
        \multicolumn{3}{l}{\small $^*$ 正则(regular)、正规(normal)有时专指除开 $T_1$ 的第二个条件。}
    \end{tabular}
\end{table}

\begin{theorem}
    $T_1\iff$ 单点集闭。
\end{theorem}
\begin{proof}
    置 $\{x\}$。$\forall y\ne x$ 即 $y\in X\backslash\{x\}$,$y$ 必有开邻域 $V\notni x$ 即 $V\subset X\backslash\{x\}$,故 $X\backslash\{x\}$ 开。
\end{proof}

\begin{theorem}
    $T_5\im T_4\im T_3\xRightarrow{(T_3\Rightarrow T_1)}T_2 \im T_1\im T_0$。
\end{theorem}

\begin{eg}
    考虑 $\R^2$ 上直线 $y=0,1$。取两条直线之并,取等价关系 $(x,0) \sim(x,1),x>0$ 来构建商拓扑空间,也就是把两条线的 $x>0$ 部分黏合一起。$(0,0),(0,1)$ 的任意开邻域存在黏合部分 $x>0$,也就没有不交开邻域,因而不是 $T_2$ 空间。
\end{eg}

\begin{eg}
    给 $\N_+$ 定义拓扑,开集取 $\mt,\N_+$ 以及 $\{1,2,3,\cdots,n\}$。这个空间既非 $T_2$ 亦非紧致的。
\end{eg}

\begin{theorem}[收缩(shrink)]
    $T_3\iff$ 对任意点 $x\in X$ 及其任意邻域 $O$,存在开集 $U$ 使
$x\in U, \bar U \subset O$。

$T_4\iff$ 对任意闭集 $F\subset X$ 与任意开集 $O\supset F$,存在开集 $U$ 使 $F\subset U, \bar U \subset O$。
\end{theorem}

\section{度量空间}

我们看到,$\R$ 的距离便可生成一个拓扑基。距离概念的推广自然是在任意集合中,给定满足三角不等式、正定性的对称实函数。开区间也可相应地推广。实际上,具有应用意义的拓扑空间大多具有距离概念。

\begin{definition}
    置集合 $X$ 和\textbf{距离}(或\textbf{度量})$d: X \times X \to [0,\infty)$,$\forall x,y,z\in X$ 满足:
    \begin{itemize}
        \item $d(x,y)=d(y,x)$;
        \item $d(x,y) = 0\eqto x=y$;
        \item $d(x,z) \leqslant d(x,y)+d(y,z)$。(三角不等式)
    \end{itemize}
    $B(x,r)=\{y \in X: d(x, y)<r\}$ 称为以 $x \in X$ 为中心、$r$ 为半径的\textbf{开球}(open ball)。
    全体开球 $\{B(x,\epsilon):x\in X,\epsilon >0\}$ 就是一个拓扑基,其生成拓扑 $\topo T_d$ 就是 $(X,d)$ 的\textbf{度量诱导拓扑}。$(X,d,\topo T_d)$ 称为\textbf{度量空间}(metric space)。存在度量 $d$ 生成的拓扑,称\textbf{可度量}。
\end{definition}

\begin{remark}
    原则上可以有负定的距离,就像度规一样,但这种距离一般不用于诱导拓扑。
\end{remark}

\begin{definition}
    \textbf{局部可度量}就是 $\forall x\in X$ 都存在开集 $O\ni x$ 可度量。
\end{definition}

\begin{eg}
    \textbf{离散度量}满足 $d_{0}(x,y)=1$ 若 $x \ne y$。这是一个描述离散拓扑的办法。
\end{eg}

\begin{definition}
    置度量空间 $X$。度量 $d_{1},d_{2}$ 是\textbf{等价度量},若 $\exists a,b>0$,使对 $\forall x,y \in X$,有 $a d_{1}(x, y) \leqslant d_{2}(x, y) \leqslant b d_{1}(x, y)$。
\end{definition}

\begin{remark}
    显然等价度量诱导出相同拓扑。
\end{remark}

\begin{eg}
    置 $x=\left(x^1,\cdots,x^{n}\right),y=\left(y^{1},\cdots,y^{n}\right)\in\R^{n}$。$p$-度量为
\[
d_{p}(x, y):=\left(\sum_{k=1}^{n}\left|x^{k}-y^{k}\right|^{p}\right)^{1 / p},\quad p \geqslant 1.
\]
常取$p=2$,称为\textbf{欧氏度量}。
$\R$ 的开球即开区间。$\R^2$ 的开球称为\textbf{开圆盘}。极限为
\[
d_{\infty}(x,y):=\lim_{p\to\infty}d_{p}(x, y)=\max _{1 \leqslant k \leqslant n}\left\{\left|x^{k}-y^{k}\right|\right\}.
\]
任意 $p$-度量等价,诱导出通常拓扑 $\topo T_u$。当然,亦可视作 $\R$ 通常拓扑的乘积拓扑。
\end{eg}

\begin{theorem}
    可度量 $\im T_5$。
\end{theorem}

我们已悉知 $\epsilon$-$\delta$ 语言可视为连续性在通常拓扑的特例。现在再用度量空间重新叙述。
置 $(X,d_X,\topo T_X),(Y,d_Y,\topo T_Y)$ 和映射 $f:X\to Y$。$f$ 在 $x \in X$ 处对度量诱导拓扑连续,当且仅当 $\forall\epsilon>0,\exists\delta>0$ 使 $B(x,\delta) \subset f^{-1} [B(f(x),\epsilon)]$。
即 $f^{-1} [B(f(x),\epsilon)]$ 是 $x$ 点邻域。
换言之,$d_X(x,x^{\prime})<\delta\Rightarrow d_Y(f(x),f(x^{\prime}))<\epsilon$。
若 $f$ 在任意点连续,即 $\forall x \in X,\epsilon > 0$,$f^{-1} [B(f(x),\epsilon)]\in\topo T_X$,结合“任意并”性质和逆像的保并性,可知 $\forall U\in\topo T_Y$,$f^{-1}[U]\in\topo T_X$,即 $f$ 连续。

\begin{theorem}
    $\R^n$ 开球和 $\R^n$ 同胚。
\end{theorem}
\begin{proof}
    不失一般性,考虑单位开球 $B(0,1)$。置 $f:B(0,1)\to \R^n$ 为
\eq{\label{eq:B-Rn}
f(x)=\frac{x}{1-|x|},
\quad
f^{-1}(y)=\frac{y}{1+|y|}.
}
它是双向连续的双射。
\end{proof}

对于度量空间,点列 $\left\{x_n\right\}$ 收敛于 $x$ 等价于 $\lim_{n\to\infty}d(x, x_n)=0$。
点列称为基本列或 Cauchy 列,即$\forall\epsilon>0,\exists N\in\N$使 $k,l>N\Rightarrow d(x_{k},x_{l})<\epsilon$。
度量空间称为\textbf{完备的}(complete),若其中任意 Cauchy 列都收敛于其中的点。

%

实直线上闭区间完备,但开区间不完备:只需注意 $\{1/n\}_{n=2}^\infty$ 在 $(0,1)$ 上无极限。

$\R^{n}$ 当然完备。
一致连续性即 $\forall \epsilon>0,\exists\delta>0$ 
使 $\forall x_{1},x_{2} \in X$,$d(x_{1},x_{2})<\delta\Rightarrow d(f(x_{1}),f(x_{2}))<\epsilon$。
注意一致连续要更强:对整个空间要求一致的 $\delta$。

\begin{definition}
    度量空间 $X$ 的子集称为\textbf{有界的},若存在开球包含它。否则称\textbf{无界的}。
\end{definition}

\begin{remark}
    完备性、有界性都不是拓扑不变性。例如,$(0,1)$ 与实直线同胚,但前者不完备、有界,后者完备、无界。
\end{remark}

置度量空间 $X$。$f: X \to X$ 称为(严格)\textbf{压缩映射}(contraction),若 $\exists c\in(0,1)$,使 $\forall x,y \in X$ 且 $x\ne y$,有 $d(f(x),f(y)) \leqslant c d(x,y)$。
仿照分析学可类似证明 \textbf{Banach 不动点定理}:若度量空间 $X$ 完备,$f$ 为一个压缩映射,则 $f(x)=x$ 的解存在且唯一。这是非线性泛函分析的一个有用工具。

%%%

\begin{definition}
    \textbf{有限 $\epsilon$-网}(net)是有限集 $\{x_n\}\subset X$,使 $\forall x\in X$,$\exists x_i$ 使 $d(x,x_i)<\epsilon$。换言之,$X\subset\bigcup_{i=1}^n B(x_i,\epsilon)$。
\end{definition}

\begin{definition}
    $(X,d)$ 称为\textbf{全有界的}(totally bounded),若存在任意 $\epsilon>0$ 的有限 $\epsilon$-网。换言之,可被有限多个半径任意小的球覆盖。
\end{definition}

\begin{remark}
    全有界 $\im$ 有界。
\end{remark}

\begin{eg}
    无穷维 Hilbert 空间 $\ell^2$ 是完备的。任意 $\epsilon>0$,$\{e_n\}_{n=1}^\infty$ 构成 $\ell^2$ 的 $\epsilon$-网格,但它是无穷集,不全有界。
\end{eg}

%%%

\section{紧致性}

分析学给出了若干等价的实数定理,其中一条是著名的\textbf{Heine-Borel 定理}:若有界闭集能被某开集族覆盖,则可从中找到有限个开集,构成依旧能覆盖的子族。
19 世纪中叶,Dirichlet 在试图严格化函数连续性时,率先用有限覆盖技术,证明了闭区间上的连续实函数一致连续。
不过他的结果半个世纪后才发表,期间 Heine, Weierstrass 等人各自独立地用类似技术也证明了该结论。Borel 利用这些技术证明了上述的有限覆盖定理。后来 Lebesgue 等人才把它推广到了一般情形。
有界性是与距离密切相关的,但有限覆盖性只涉及开集。可见关于有界闭集的分析学命题,很多都能延伸到点集拓扑。

\begin{definition}
    $A$ 称\textbf{紧致的}(compact)或\textbf{紧的},若其任意开覆盖都存在有限子覆盖。
\end{definition}

\begin{eg}
    实直线上,单点集 $\{x\}\subset X$ 必紧。$\R$、开区间、半开区间非紧。
\end{eg}
\begin{proof}
    对 $\{x\}$ 任取开覆盖 $\topo C$,则 $\exists O\in\topo C$ 使 $x\in O$,$\{O\}\subset\topo C$ 是有限集。
    不妨以 $(0,1]$ 为例,存在开覆盖 $\{(1/n,2):n\in\N\}$ 不具有限子覆盖;其余情形同理。
\end{proof}

\begin{definition}
    若 $K\subset X$ 中任何序列都包含收敛子列,则称 $K$ 是\textbf{相对紧的}。若这些子列还收敛于 $K$ 中,则称\textbf{自列紧的},简称\textbf{列紧}\footnote{列紧在部分书中指相对紧。}。
\end{definition}

%%%
\begin{remark}
    紧 $\im$ 列紧。
    若空间满足可数基或序列空间性质,两者可能一致。
\end{remark}

\begin{theorem}\label{thm:compactness}
    度量空间上,紧 $\iff$ 列紧 $\iff$ 全有界 $\land$ 闭 $\land$ 完备。
\end{theorem}
\begin{proof}
    设 $X$ 是紧的。任意 $x\in X$,$\exists O\in\topo S$ 使 $x\in O$,则 $\topo S$ 是 $x$ 的某邻域的有限子覆盖。任意点列 $\{x_n\}$ 都有极限点 $x$,则 $\exists O\in\topo S$ 使 $x\in O$,因此 $\{x_n\}$ 有收敛子列。
    反之,若 $X$ 是完备的且有界的,则 $\forall \epsilon>0,\exists x_1,\cdots,x_n\in X$
    使 $\bigcup_{i=1}^n B(x_i,\epsilon)$ 覆盖 $X$。任意开覆盖 $\topo C$ 都存在 $\epsilon$-覆盖 $\topo S=\{B(x_i,\epsilon):x_i\in X\}$,则 $\topo S$ 是 $\topo C$ 的有限子覆盖。
    反之,若 $X$ 的任意开覆盖都存在有限子覆盖,则任意点列 $\{x_n\}$ 的极限点 $x$ 都有某开邻域 $O$,使 $\{x_n\}$ 的无限个点都在 $O$ 中,因此 $\{x_n\}$ 有收敛子列。任意有界序列都有收敛子列,因此 $X$ 是完备的。
\end{proof}

\begin{eg}[Bolzano-Weierstrass]
    实直线上,列紧 $\iff$ 有界 $\land$ 闭。
\end{eg}
\begin{proof}
    只需说明有界闭 $\iff$ 全有界,完备性自动满足。

    若 $K\subset \R$ 是紧集。考虑 $K\subset\bigcup_{n\in\N^+}(-n,n)=\R$,紧性使存在 $K\subset \bigcup_{i=1}^m(-n_i,n_i)=(-\max_i{n_i},\max_i{n_i})$,即有界。
\end{proof}

\begin{eg}[Heine-Borel]
    结合 \ref{thm:compactness} 知,实直线上,紧 $\iff$ 有界 $\land$ 闭。
\end{eg}
%%%

\begin{definition}
    \textbf{局部紧}是任意点都存在\textbf{邻域}相对紧(闭包紧)。
\end{definition}

紧致要求只用有限的开集就能覆盖全集,但这太过严苛。
可以尝试弱化,即全局上仍可能需无穷个开集,但要求每个点附近只相交有限个开集。
这样无穷和在每点附近仍可转化为对有限项的拼接,便于操作。

%%%
\begin{definition}
置 $X$ 的覆盖 $\topo V,\topo U$,若 $\forall V\in\topo V$ 都 $\exists U\in\topo U$ 使 $V\subset U$,
则称 $\topo V$ 是 $\topo U$ 的一个\textbf{细化}或\textbf{加细}(refinement)。默认讨论开覆盖,故称\textbf{开加细}。
\end{definition}

\begin{definition}
覆盖 $\topo V$ 称\textbf{局部有限}(locally finite),若 $\forall x\in X$,$\exists$ $x$ 的邻域只与 $\topo V$ 中有限多个集合相交:$|\{V\in\topo V:\ O\cap V\ne\varnothing, O\in\topo N(x)\}|<\aleph_0$。
\end{definition}


\begin{definition}
    \textbf{仿紧}或\textbf{副紧}(paracompact)的拓扑空间中,任意开覆盖都存在局部有限的开加细。
\end{definition}
%%%
\begin{remark}
    更原始的定义是称“任意开覆盖都存在开星加细”:称 $\topo V$ 是覆盖 $\topo U$ 的星加细,对每个 $V\in\topo V$,存在 $U\in\topo U$ 使 $\operatorname{St}(V,\topo V):=\bigcup\{O\in\topo V:\ O\cap V\ne\varnothing\}\subset U$。
    但“局部有限的开加细”最实用。
\end{remark}

\begin{theorem}[Stone]
    可度量 $\im$ 仿紧。
\end{theorem}
\begin{proof}
    下面对一般度量空间 $(X,d)$ 证明副紧性:任取开覆盖 $\topo U$,要构造一个局部有限开细化。

对每个 $x\in X$,选取 $U_x\in\topo U$ 使 $x\in U_x$。
因为 $U_x$ 开,存在 $r(x)>0$ 使得开球 $B(x,2r(x))\subset U_x$。
令
\[
\topo B:=\{B(x,r(x)):\ x\in X\},
\]
则 $\topo B$ 是 $\topo U$ 的开细化(因为 $B(x,r(x))\subset B(x,2r(x))\subset U_x$),但未必局部有限。
为此我们把这些球按半径分层:

对每个 $n\in\mathbb N$,定义
\[
X_n:=\{x\in X:\ 2^{-(n+1)}<r(x)\le 2^{-n}\},
\qquad
\topo B_n:=\{B(x,r(x)):\ x\in X_n\}.
\]
则 $\topo B=\bigcup_{n=1}^\infty \topo B_n$,且每层的半径都有统一下界 $>2^{-(n+1)}$。

现在对固定的 $n$,在集合 $X_n$ 中取一个\emph{极大}的 $2^{-(n+2)}$-分离子集 $S_n$:
\[
d(s,s')\ge 2^{-(n+2)}\quad(s\ne s',\ s,s'\in S_n),
\]
且 $S_n$ 在 $X_n$ 中关于包含关系极大(Zorn 引理或贪心构造均可保证存在)。
极大性推出:对任意 $x\in X_n$,存在 $s\in S_n$ 使得
\[
d(x,s)<2^{-(n+2)}.
\]
否则 $S_n\cup\{x\}$ 仍是分离集,违背极大性。

于是对每个 $n$,族
\[
\topo V_n:=\{B(s,2^{-n}) : s\in S_n\}
\]
覆盖 $X_n$(因为若 $x\in X_n$,取 $s\in S_n$ 满足 $d(x,s)<2^{-(n+2)}<2^{-n}$)。

\emph{关键:$\topo V_n$ 是局部有限的。}
取任意点 $p\in X$,看球 $B(p,2^{-(n+2)})$。
若它与很多个 $B(s,2^{-n})$ 相交,则对每个相交的 $s$,有
\[
d(p,s)\le 2^{-n}+2^{-(n+2)}<2^{-n+1}.
\]
也就是说所有这些 $s$ 都落在有界集合 $B(p,2^{-n+1})$ 内。
另一方面,$S_n$ 是 $2^{-(n+2)}$-分离的,故不同的 $s$ 之间至少相距 $2^{-(n+2)}$。
这迫使在固定半径的球 $B(p,2^{-n+1})$ 中只能容纳\emph{有限}多个这样的 $s$(否则会出现无限多个两两分离点,
从而可构造出无限多互不相交的小球,矛盾于该大球的“有限装填”,这是度量空间中的标准有界分离集有限性论证)。
因此 $B(p,2^{-(n+2)})$ 只能与有限多个 $B(s,2^{-n})$ 相交,局部有限性得证。

最后令
\[
\topo V:=\bigcup_{n=1}^\infty \topo V_n.
\]
则 $\topo V$ 覆盖 $X$(因为 $X=\bigcup_n X_n$,且每个 $X_n$ 被 $\topo V_n$ 覆盖),并且是可数并联的局部有限族:
对任意点 $p$,在每个固定层 $n$ 上只有有限多个集合与某小邻域相交,
再取更小邻域即可保证跨所有层也只与有限多个相交,从而 $\topo V$ 局部有限。

还需验证它细化原覆盖 $\topo U$:对 $B(s,2^{-n})\in\topo V_n$,
由于 $s\in X_n$,有 $r(s)>2^{-(n+1)}$,从而 $2^{-n}\le 2r(s)$,于是
\[
B(s,2^{-n})\subset B(s,2r(s))\subset U_s\in\topo U.
\]
故 $\topo V$ 是 $\topo U$ 的局部有限开细化。
因此任意度量空间副紧。

将 $X$ 取为第二步得到可度量化的 $M$,即得 $M$ 副紧,从而仿紧。证毕。
\end{proof}

\section{度量化定理*}

\begin{theorem}
    $T_2\xRightarrow{\text{仿紧}} T_4$ 
\end{theorem}
\begin{proof}
    
\end{proof}

\begin{theorem}[Urysohn-Lindel\"of]
    $T_3\xLeftrightarrow{C_2} T_4\xLeftrightarrow{C_2}$ 可度量
\end{theorem}


\begin{theorem}[Smirnov]
    $\text{仿紧}\land T_2\land\text{局部可度量}\im$ 可度量
\end{theorem}

\begin{theorem}[Negata-Smirnov]
    
\end{theorem}
\begin{remark}
    这是另⼀个经典研究成果。证明⾮常复杂精巧,实际应⽤中也没有很多可以模仿着去做的机会,故略去。
\end{remark}


\begin{theorem}
    $T_2 \xRightarrow{\text{局部紧}} T_3$
\end{theorem}

\begin{proof}
\textbf{局部正规}

而在局部紧 $T_2$ 空间中,总可以把点与不含该点的闭集用互不相交的开集分开,这正是正则性。

具体地,取任意 $x\in M$ 与闭集 $F\subset M$,且 $x\notin F$。
因为 $F$ 闭,所以 $U_0:=M\setminus F$ 是开且含 $x$。
由局部紧性,存在开集 $U$ 使得
\[
x\in U,\qquad \overline U\ \text{紧},\qquad \overline U\subset U_0.
\]
于是 $\overline U\cap F=\varnothing$。

对每个 $y\in F$,由 $T_2$ 性,存在互不相交开集 $U_y\ni x$ 与 $W_y\ni y$。
注意到 $\overline U$ 是紧集,且 $\{W_y\}_{y\in F}$ 是 $F$ 的开覆盖,
从而也是 $\overline U\cap F=\varnothing$ 的“无关”——我们改用紧性来做如下操作:
对每个 $p\in \overline U$,因 $p\notin F$ 且 $F$ 闭,$T_2$ $\Rightarrow$ 可分离,
故存在开集 $A_p\ni p$ 与开集 $B_p\supset F$ 使得 $A_p\cap (M\setminus B_p)=\varnothing$,
亦即 $A_p\subset B_p$ 且 $B_p$ 仍是包含 $F$ 的开集。
于是 $\{A_p\}_{p\in\overline U}$ 覆盖紧集 $\overline U$,
取有限子覆盖 $A_{p_1},\dots,A_{p_N}$,令
\[
V:=\bigcup_{j=1}^N A_{p_j},\qquad W:=\bigcap_{j=1}^N B_{p_j}.
\]
则 $V$ 为开且含 $\overline U$,$W$ 为开且含 $F$,并且由于每个 $A_{p_j}\subset B_{p_j}$,
有 $V\subset W$,从而 $V\cap (M\setminus W)=\varnothing$。
现在取 $U':=U$(开含 $x$)与 $W':=M\setminus V$(开含 $F$),便得
\[
U'\cap W'=\varnothing.
\]
于是正则。
从而 $T_2 \land \text{局部紧}\iff T_2 \land (T_2 \land \text{局部紧}) \im T_1 \land \text{正则} \iff T_3$。
\end{proof}

\begin{definition}
    置开覆盖 $\topo P=\{O_i\}$。若函数族 $\{\rho_i\}$ 满足
    \begin{itemize}
        \item $0\leqslant\rho_i\leqslant 1$;
        \item $\rho_i$ 支撑在 $O_i$ 上;
        \item $\sum_i \rho_i=1$,
    \end{itemize}
    则称从属于 $\topo P$ 的一个\textbf{单位分解}(partition of unity)。若 $\rho_i$ 连续则称\textbf{连续单位分解}。
\end{definition}

\begin{theorem}
    仿紧 $\land$ $T_2\im$ 任何开覆盖存在从属于它的局部有限的连续单位分解。
\end{theorem}
\begin{proof}
    设开覆盖 $\topo P=\{O_\alpha\}$。
    由仿紧性,$\topo P$ 存在局部有限开加细 $\{U_\alpha\}$ 满足 $U_\alpha\subset O_\alpha$。
    
    (必要时可先取局部有限细化 $\{U_i\}$,再对每个 $U_i$ 选取一个包含它的 $O_{\alpha(i)}$,并重编号。)

    \medskip
    \noindent\textbf{构造从属于 $\{U_i\}$ 的非负光滑函数族 $\{\varphi_i\}$。}
    由于 仿紧 $\land$ $T_2\im T_4$。由 $T_4$ 收缩性,对每个 $i$ 可取开集 $V_i$ 使得
    \[
        \overline{V_i}\subset U_i.
    \]
    由流形上的 bump 函数引理(亦即光滑 Urysohn 引理),存在 $\varphi_i\in C^\infty(M)$ 满足
    \begin{itemize}
        \item $0\le \varphi_i\le 1$;
        \item $\varphi_i\equiv 1$ 于 $\overline{V_i}$;
        \item $\supp(\varphi_i)\subset U_i$。
    \end{itemize}
    特别地,$\varphi_i$ 在 $M\setminus U_i$ 上恒为 $0$,因此也从属于原覆盖:$\supp(\varphi_i)\subset U_i\subset O_i$。

    \medskip
    \noindent\textbf{归一化。}
    令
    \[
        \Phi(x):=\sum_{i=1}^\infty \varphi_i(x).
    \]
    由于 $\{U_i\}$ 局部有限,故对每个 $x\in M$,在某个邻域内仅有限个 $\varphi_i$ 非零,
    因而上述和在该邻域内是有限和,$\Phi$ 是光滑函数。

    下面证明 $\Phi(x)>0$ 对一切 $x$ 成立。
    因为 $\{V_i\}$ 仍覆盖 $M$(可在取 $V_i$ 时保证这一点:对每个 $x\in U_i$ 选一个相对紧邻域并并入某个 $V_j$;或等价地先取 $\{V_i\}$ 为 $\{U_i\}$ 的开覆盖并满足 $\overline{V_i}\subset U_i$),
    所以给定 $x\in M$,存在某个 $k$ 使得 $x\in V_k\subset \overline{V_k}$,从而 $\varphi_k(x)=1$,于是 $\Phi(x)\ge 1$。

    因此可定义
    \eq{
        \rho_i(x):=\frac{\varphi_i(x)}{\Phi(x)}.
    }
    则 $\rho_i\in C^\infty(M)$,且 $0\le\rho_i\le 1$,并且
    \[
        \supp(\rho_i)\subset \supp(\varphi_i)\subset U_i\subset O_i.
    \]
    最后,
    \[
        \sum_{i=1}^\infty \rho_i(x)=\frac{\sum_i\varphi_i(x)}{\Phi(x)}=1,
    \]
    且由于局部有限性,上式在任意点的邻域内都是有限和,因此无收敛性问题。
    故 $\{\rho_i\}$ 构成从属于 $\topo P$ 的单位分解。
\end{proof}


\section{连通性}

\begin{definition}
    拓扑空间 $X$ 称为\textbf{连通的}(connected),若它不能写成两个互不相交的非空开集之并。
    定义等价关系如下:$x \sim y$ 当且仅当 $x,y$ 属于 $X$ 的同一个连通子空间。每个等价类称 $X$ 的一个\textbf{连通分支}。
\end{definition}
\begin{theorem}
    连通的两条有用的等价说法:任何一个从 $X$ 到赋予离散拓扑的两点集合 $\{0,1\}$ 的连续映射都不是满射;连通拓扑空间只有两个既开又闭的子集。
\end{theorem}

\begin{theorem}
    连续映射保持连通。
\end{theorem}
%% 有问题
\begin{proof}
假使 $f[A]$ 不连通,则存在不交非空开集 $U,V\subset Y$ 使
\[
f(A)\subset U\cup V,\quad f(A)\cap U\ne\varnothing,\quad f(A)\cap V\ne\varnothing.
\]
于是
\[
A = f^{-1}(U)\cap A\ \cup\ f^{-1}(V)\cap A
\]
是把 $A$ 分成两个不交的非空开集(因为 $f^{-1}(U), f^{-1}(V)$ 在 $X$ 中开),矛盾。故 $f(A)$ 连通。
\end{proof}

\begin{definition}
    设区间 $I\subset\R$ 和拓扑空间 $X$,连续映射 $C:I\to X$ 称为 $X$ 上的一条(有向)\textbf{曲线}(curve)。
    $\sigma\in I$ 称为曲线的\textbf{参数},$\Im C$ 称为\textbf{路径}(path),不混淆时亦可称曲线。
\end{definition}

\begin{definition}
    $C':I'\to X$ 称为\textbf{重参数化}。严格单调
\end{definition}

\begin{definition}
    拓扑空间 $X$ 称为\textbf{路径连通的}(path-connected)或\textbf{弧连通的}(arcwise-connected),如果 $X$ 中的任意两点都可被一条完全在 $X$ 中的路径连接。类似地有\textbf{局部路径连通}。
\end{definition}

\begin{eg}
    开球路径连通,因为对任意两点 $a,b\in B(x,r)$,存在线段
 $(1-t)a+tb,t\in[0,1]$ 都在 $B(x,r)$ 里。
 由于这里路径为直线,故路径连通甚至升级为了凸性(convex)。
\end{eg}

\begin{theorem}
    连续映射保持路径连通。
\end{theorem}

\begin{proof}
    置 $A\subset X$ 路径连通、$f:X\to Y$ 连续。任取 $y_0,y_1\in f[A]$,则 $\exists a_0,a_1\in A$ 使 $y_0=f(a_0),y_1=f(a_1)$。
    进而存在路径 $\gamma:[0,1]\to A$ 连接 $a_0,a_1$,即 $\gamma(0)=a_0,\gamma(1)=a_1$。复合保持连续性,从而 $f\circ\gamma$ 连接 $y_0,y_1$。故 $f[A]$ 路径连通。
\end{proof}

\begin{eg}
    $O\in\topo T_u$ 一定局部路径连通:$\forall x\in O$,$\exists U:= B(x,r)\subset O$(从而 $B(x,r)=O\cap B(x,r)\in\topo T_O$)使 $x\in U$ 且 $U$ 路径连通。
\end{eg}

\begin{theorem}
    连续开映射保持局部路径连通性。
\end{theorem}
%% 没区分子拓扑
\begin{proof}
置 $A\subset X$ 局部路径连通、$f:X\to Y$ 连续开。 
任取 $y\in f[A]$,$\exists x\in X$ 使 $f(x)=y$。故 $\exists U\in\topo T_X$ 使 $x\in U$ 且 $U$ 路径连通,从而 $\exists V:=f[U]\in\topo T_Y$ 使 $y=f(x)\in V$ 且 $V$ 路径连通。
故 $f[A]$ 局部路径连通。
\end{proof}

\begin{proof}
设 $f:X\to Y$ 连续且为开映射,$A\subset X$ 局部路径连通。记 $B=f(A)$,并赋予子空间拓扑。我们证明 $B$ 局部路径连通。

任取 $y\in B$,取 $x\in A$ 使得 $f(x)=y$。令 $O\subset B$ 为 $y$ 的任意开邻域。由于 $B$ 取子空间拓扑,存在 $Y$ 中开集 $U$ 使得
\[
O=B\cap U.
\]
由连续性知 $f^{-1}(U)$ 在 $X$ 中开,从而
\[
f^{-1}(O)=f^{-1}(B\cap U)=f^{-1}(B)\cap f^{-1}(U)\supset A\cap f^{-1}(U)
\]
是 $X$ 中的集合;并且因为 $x\in A$ 且 $f(x)\in U$,可知
\[
x\in A\cap f^{-1}(U).
\]
注意到 $A\cap f^{-1}(U)$ 在 $A$ 的子空间拓扑中是开集,因此它是 $x$ 在 $A$ 中的一个开邻域。

由于 $A$ 局部路径连通,存在 $A$ 中路径连通的开集 $W$ 满足
\[
x\in W\subset A\cap f^{-1}(U).
\]
于是 $f(W)\subset U$ 且 $f(W)\subset f(A)=B$,从而
\[
f(W)\subset B\cap U=O.
\]

下面说明 $f(W)$ 是 $B$ 中开且路径连通:
\begin{itemize}
\item $W$ 在 $A$ 中开,故存在 $X$ 中开集 $G$ 使 $W=A\cap G$。由于 $f$ 为开映射,$f(G)$ 在 $Y$ 中开,从而
\[
f(W)\subset f(G),\qquad f(W)=B\cap f(G),
\]
因此 $f(W)$ 在 $B$ 中开。
\item $W$ 路径连通且 $f$ 连续,所以连续像 $f(W)$ 路径连通。
\end{itemize}

综上,对任意 $y\in B$ 的任意开邻域 $O$,都能找到 $B$ 中开且路径连通的邻域 $f(W)$ 满足
\[
y\in f(W)\subset O.
\]
故 $B=f(A)$ 局部路径连通。
\end{proof}


\begin{theorem}
    路径连通 $\im$ 连通。
\end{theorem}
%%
\begin{proof}
    置 $X$ 路径连通,任取 $x,y\in X$,存在 $\gamma:[0,1]\to X$ 连接。因 $[0,1]$ 连通,故 $\Im\gamma$ 连通。

假使 $X$ 不连通,则存在不交非空开集 $U,V$ 使 $X=U\cup V$。取 $x\in U,y\in V$,则连通集 $\Im\gamma$ 必须完全落在 $U$ 或完全落在 $V$(否则会被 $U,V$ 分割成两个开集,破坏连通),矛盾。

    从而整个空间不可能被分成两个不交的非空开集。
\end{proof}

\begin{proof}
    可以,最“直接”的证明就是用**反证法**:

设 $X$ 不连通,则存在不交非空开集 $U,V$ 使得
\[
X=U\sqcup V.
\]
取 $x\in U,\ y\in V$。假设 $X$ 路径连通,则存在连续路径 $\gamma:[0,1]\to X$,$\gamma(0)=x,\ \gamma(1)=y$。

令
\[
A=\gamma^{-1}(U),\qquad B=\gamma^{-1}(V).
\]
由于 $\gamma$ 连续且 $U,V$ 开,故 $A,B$ 是 $[0,1]$ 中的开集;又因 $U\cap V=\varnothing$ 且 $U\cup V=X$,得
\[
A\cap B=\varnothing,\qquad A\cup B=[0,1].
\]
并且 $0\in A$(因为 $\gamma(0)=x\in U$),$1\in B$(因为 $\gamma(1)=y\in V$),所以 $A,B$ 都非空。

这说明 $[0,1]$ 被分成两个不交的非空开集 $A,B$,与 $[0,1]$ 连通矛盾。

因此不存在这样的路径 $\gamma$,即 $X$ 不路径连通。
\end{proof}


\begin{theorem}
    连通 $\land$ 局部路径连通 $\im$ 路径连通
\end{theorem}
\begin{proof}
    置 $(M,\topo T)$ 连通 $\land$ 局部路径连通。
任取 $p\in M$,记 $C_p:=\{q\in M:\text{存在路径连接 $p,q$}\}$ 为包含 $p$ 的路径连通分支。任取 $q\in C_p$,由于 $M$ 局部路径连通,存在 $q$ 的一个路径连通的开邻域 $V$。对任意 $r\in V$,可将 $p,q$ 的路径与 $V$ 中 $q,r$ 的路径拼接,得到一条从 $p$ 到 $r$ 的路径,因此 $r\in C_p$,从而 $V\subset C_p$,从而 $q$ 是 $C_p$ 的内点,这说明 $C_p$ 是开集。

因此,$M$ 可以表示为若干个互不相交的开路径连通分支的并。若 $M$ 不路径连通,则至少存在两个不同的路径连通分支,从而 $M$ 可分解为两个不交的非空开集,这与 $M$ 的连通性矛盾。

* 同理,其他路径连通分支也都开;而不同路径连通分支两两不交,且并起来是全体 $X$。
* 若 $X$ 有至少两个路径连通分支 $C_x,C_{x'}$,则
  \[
  X=C_x\sqcup (X\backslash C_x)
  \]
  是两个不交非空开集的分割,违背连通性。

所以 $X$ 只能有一个路径连通分支,即 $X$ 路径连通。证毕。
\end{proof}

一般总能将非连通集视为有限个连通分支之并,不妨只考虑其中一个连通分支。连通集合称为\textbf{单连通的},就是说,其中任意形似橡皮圈的闭合曲线(称\textbf{简单闭线})所围成的任意曲面都含于此集合中。

单连通区域总同胚开球。

%%%

连通区域的表面正是\textbf{闭曲面}。
比如,闭曲面总能将 $\R^n$ 分为有界部分(即所包围区域)和无界部分,因此可区分空间的内外。
一般偏好将法矢取定为朝外,称为\textbf{外指}(outward-pointing)\textbf{法矢},简称\textbf{外法矢}。
单连通的表面就称\textbf{简单闭曲面},简单闭线可视作特殊情形。
物理学总是关注这种良好对象,比如求分段光滑的简单闭曲面的场通量,而实际上分析学的 Gauss 定理的确适用之。
