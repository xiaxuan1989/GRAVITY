\chapter{Lie 导数与对称性}\label{appx:Lie-deri}

一般的光滑映射 $\phi$ 仍可诱导其上张量场的变换。给定标量场 $f:N\to\R$,$p\in M$ 变到 $\phi(p)\in N$ 后标量场的值为 $f(\phi(p))$,而我们可将 $\phi(p)$ 处的 $f$ 值直接搬运回 $p$ 处,即定义新标量场 $\phi^* f:M\to\R$ 满足 $\phi^* f(p):=f(\phi(p))$。
\begin{definition}
    $\phi^*:C^\infty(M)\to C^\infty(N)$ 称为\textbf{拉回}(pull back),使 $\phi^* f:=f\circ\phi$。
\end{definition}

\begin{theorem}
    易知 $\phi^*$ 是线性映射,且 $\phi^*(fg)=(\phi^* f) (\phi^* g)$。
\end{theorem}



类似地,记 $\phi(p)=q$,若 $\phi$ 是同胚,则也可将 $p=\phi^{-1}(q)$ 处的 $f$ 值往前搬至 $q$ 处,即定义 $\phi_* f$ 使 $\phi_* f(q):=f(\phi^{-1}(q))$ 或者 $\phi_* f:=f\circ\phi^{-1}$,故
\eq{
    \phi_*=(\phi^{-1})^*\iff \phi^*=(\phi^{-1})_*.
}
再研究切矢场 $v$。确定一个 $v$ 只需对其作用于任意 $f$ 的结果做出要求。将 $v$ 推前即 $\phi_* v(f):=v(\phi^* f)$。这里未出现 $\phi^{-1}$,可见对一般光滑映射,只能借助切矢场给出:
\begin{definition}
    $\phi_*:T_p M\to T_{\phi(p)}N$ 称为\textbf{推前}(push forward),使 $\phi_* v:=v\circ \phi^*$。
\end{definition}

将 $A$ 拉回即 $\phi^* A(f):=A(\phi_* f)$,或直接用 $\phi^*=(\phi^{-1})_*$。对余切矢场 $\omega$ 只需考虑对切矢场 $A$ 作用,同理 $\phi^* \omega:=\omega\circ\phi_*,\phi_* \omega:=\omega\circ\phi^*$。一般地,对任意张量场,其拉回和推前即
\eq{
    \phi^* T=T(\phi_*(\oo),\cdots;\phi_*(\oo),\cdots),\quad \phi_* T=T(\phi^*(\oo),\cdots;\phi^*(\oo),\cdots).
}



进一步对微分同胚,则称\textbf{单参微分同胚群}。对 $\forall p\in M$,取不同参数 $t$ 的 $\phi_t$ 将其映射至不同点 $\phi_t(p)$,形成一条以 $\phi_0(p)=p$ 为参数零点的光滑曲线 $C_p:\R\to M$,这称为该单参微分同胚群过 $p$ 点的\textbf{轨道}(orbit)。

因此 $M$ 上的单参微分同胚群给出其对应的光滑切矢场。

由于同胚是双射,因此各轨道是不会相交的,因此只要给出各 $p$ 点在其对应轨道的切矢 $\pdv*{t}$,就给出了 $M$ 上的光滑切矢场。



反之是否成立?可以证明,


看来 $\forall \sigma\in\R$ 可借用 $V^\mu$ 的积分曲线定义微分同胚 $\phi_\sigma:M\to M$:设过 $p$ 的积分曲线为 $C_p$(且 $C_p(0)=p$),规定 $\phi_\sigma(p):=C_p(\sigma)$,于是似乎得到了单参微分同胚群。然而存在极端情形,某条积分曲线在参数取某些值时像点不存在,比如可人为挖去背景的某些区域,因此只能说 $V^\mu$ 局部地给出单参微分同胚群。准确地,只要 $p\in M$ 存在,总可找到其开邻域 $U\subset M$,使得 $U,\phi_t[U]$ 同胚且满足$\phi_t\phi_s=\phi_{t+s}$,其中参数范围为零的某开邻域 $I\subset\R$。这样我们就把 $\{\phi_t|t\in I\}$ 称为 $p$ 的\textbf{单参微分同胚局部群}。

一般来说总考虑 $U=M$。

\textbf{参考系}指时空上的一个光滑单位类时矢量场 $\bm Z$。

数学上,如果一条曲线处处与该矢量场相切,就称为该矢量场的一条积分曲线。十分显然,因为一点只对应一条积分曲线,倘若对应多条,矢量场于该点的取值便出现多值。可见积分曲线不会相交。

参考系的每一积分曲线 $\gamma(\tau)$ 都是类时世界线。

所有微分同胚(包括那些不同单参变换的复合,它们不是单参变换)仍将构成群,称为 $M$ 的\textbf{微分同胚群}。一个集合可以是群,也可同时是一个背景,这种群称为 \textbf{Lie 群}。连续变换构成的群就可以用参数代表其坐标系,故实质上 Lie 群,其维数即参数种类,而一个切矢场又给出一种参数,可见微分同胚群是一种无穷维 Lie 群。一个光滑切矢场生成的是微分同胚群的一个\textbf{单参子群},故又被称为微分同胚群的一个\textbf{无穷小生成元}或\textbf{生成元}。

矢量场 $\xi$ 能局部地给出单参微分同胚群 $\{\phi_t\}$。由 $\phi_t$ 可诱导出一种导数,其目的是为了比较邻近的 $T,\phi_{t*} T$ 或等价的 $\phi_t^*T,T$。
\begin{definition}
    张量场沿切矢场 $\xi$ 的 \textbf{Lie 导数}为
\eq{
\Lie_\xi T:=\lim_{t\to 0}\frac{\phi_t^* T-T}{t}.
}
\end{definition}

\begin{eg}
    就标量场 $f$ 而言 $\Lie_\xi f=\xi(f)$。
\end{eg}


定义流形间映射便可谈及微分同胚。微分同胚映射可自然诱导张量丛的映射,故可谈及张量变换的主动、被动观点。所有微分同胚构成微分同胚群。任意光滑矢量场是该群的生成元。显然微分同胚有无穷多生成元,故维数无穷。一个光滑矢量场在群中挑出一个单参子群,称为单参微分同胚局部群(无穷小微分同胚)。若光滑矢量场还有完备性,则给出一个单参微分同胚群。紧致流形上任意矢量场可延拓至完备的。保持度规分量不变的变换对应的微分同胚映射称为等度规映射。诱导此映射的矢量场称为 Killing 矢量。代数定义可从 Lie 导数出发,也可从 Killing 方程出发以跳过 Lie 导数。此定义便告知可用 Killing 矢量场讨论度规(时空)的对称性。

