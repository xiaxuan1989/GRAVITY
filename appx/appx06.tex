\chapter{Lie 导数与对称性}\label{appx:Lie-deri}

一般的光滑映射 $\phi$ 仍可诱导其上张量场的变换。给定标量场 $f:N\to\R$,$p\in M$ 变到 $\phi(p)\in N$ 后标量场的值为 $f(\phi(p))$,而我们可将 $\phi(p)$ 处的 $f$ 值直接搬运回 $p$ 处,即定义新标量场 $\phi^* f:M\to\R$ 满足 $\phi^* f(p):=f(\phi(p))$。
\begin{definition}
    $\phi^*:C^\infty(M)\to C^\infty(N)$ 称为\textbf{拉回}(pull back),使 $\phi^* f:=f\circ\phi$。
\end{definition}

\begin{theorem}
    易知 $\phi^*$ 是线性映射,且 $\phi^*(fg)=(\phi^* f) (\phi^* g)$。
\end{theorem}



类似地,记 $\phi(p)=q$,若 $\phi$ 是同胚,则也可将 $p=\phi^{-1}(q)$ 处的 $f$ 值往前搬至 $q$ 处,即定义 $\phi_* f$ 使 $\phi_* f(q):=f(\phi^{-1}(q))$ 或者 $\phi_* f:=f\circ\phi^{-1}$,故
\eq{
    \phi_*=(\phi^{-1})^*\Longleftrightarrow \phi^*=(\phi^{-1})_*.
}
再研究切矢场 $v$。确定一个 $v$ 只需对其作用于任意 $f$ 的结果做出要求。将 $v$ 推前即 $\phi_* v(f):=v(\phi^* f)$。这里未出现 $\phi^{-1}$,可见对一般光滑映射,只能借助切矢场给出:
\begin{definition}
    $\phi_*:T_p M\to T_{\phi(p)}N$ 称为\textbf{推前}(push forward),使 $\phi_* v:=v\circ \phi^*$。
\end{definition}

将 $A$ 拉回即 $\phi^* A(f):=A(\phi_* f)$,或直接用 $\phi^*=(\phi^{-1})_*$。对余切矢场 $\omega$ 只需考虑对切矢场 $A$ 作用,同理 $\phi^* \omega:=\omega\circ\phi_*,\phi_* \omega:=\omega\circ\phi^*$。一般地,对任意张量场,其拉回和推前即
\eq{
    \phi^* T=T(\phi_*(\oo),\cdots;\phi_*(\oo),\cdots),\quad \phi_* T=T(\phi^*(\oo),\cdots;\phi^*(\oo),\cdots).
}



进一步对微分同胚,则称\textbf{单参微分同胚群}。对 $\forall p\in M$,取不同参数 $t$ 的 $\phi_t$ 将其映射至不同点 $\phi_t(p)$,形成一条以 $\phi_0(p)=p$ 为参数零点的光滑曲线 $C_p:\R\to M$,这称为该单参微分同胚群过 $p$ 点的\textbf{轨道}(orbit)。

由于同胚是双射,因此各轨道是不会相交的,因此只要给出各 $p$ 点在其对应轨道的切矢 $\dv*{t}$,就给出了 $M$ 上的光滑切矢场。

因此 $M$ 上的单参微分同胚群给出其对应的光滑切矢场。

反之是否成立?可以证明,
给定一个光滑切矢场 $V^\mu$,则任意点 $p$ 必有唯一曲线 $\gamma(\sigma)$ 经过,且 $\dv*{\sigma}=V$ 在曲线上处处满足。这种曲线称为\textbf{积分曲线}(integral curve)或\textbf{流线}(streamline)。因为可取任意坐标系 $\{x\}$,则积分曲线即要求其参数式 $x^\mu(\sigma)$ 满足
\eq{
    \dv{x^\mu}{\sigma}=V^\mu,
}
其中 $V^\mu$ 是 $x^\mu$ 的函数,故这是一阶常微分方程,给定初始条件 $x^\mu(0)$ 立即有唯一解。这样,总可给光滑矢量场找到一族流线。

看来 $\forall \sigma\in\R$ 可借用 $V^\mu$ 的积分曲线定义微分同胚 $\phi_\sigma:M\to M$:设过 $p$ 的积分曲线为 $C_p$(且 $C_p(0)=p$),规定 $\phi_\sigma(p):=C_p(\sigma)$,于是似乎得到了单参微分同胚群。然而存在极端情形,某条积分曲线在参数取某些值时像点不存在,比如可人为挖去背景的某些区域,因此只能说 $V^\mu$ 局部地给出单参微分同胚群。准确地,只要 $p\in M$ 存在,总可找到其开邻域 $U\subset M$,使得 $U,\phi_t[U]$ 同胚且满足$\phi_t\phi_s=\phi_{t+s}$,其中参数范围为零的某开邻域 $I\subset\R$。这样我们就把 $\{\phi_t|t\in I\}$ 称为 $p$ 的\textbf{单参微分同胚局部群}。

一般来说总考虑 $U=M$。

\textbf{参考系}指时空上的一个光滑单位类时矢量场 $\bm Z$。

数学上,如果一条曲线处处与该矢量场相切,就称为该矢量场的一条积分曲线。十分显然,因为一点只对应一条积分曲线,倘若对应多条,矢量场于该点的取值便出现多值。可见积分曲线不会相交。

参考系的每一积分曲线 $\gamma(\tau)$ 都是类时世界线。

所有微分同胚(包括那些不同单参变换的复合,它们不是单参变换)仍将构成群,称为 $M$ 的\textbf{微分同胚群}。一个集合可以是群,也可同时是一个背景,这种群称为 \textbf{Lie 群}。连续变换构成的群就可以用参数代表其坐标系,故实质上 Lie 群,其维数即参数种类,而一个切矢场又给出一种参数,可见微分同胚群是一种无穷维 Lie 群。一个光滑切矢场生成的是微分同胚群的一个\textbf{单参子群},故又被称为微分同胚群的一个\textbf{无穷小生成元}或\textbf{生成元}。

矢量场 $\xi$ 能局部地给出单参微分同胚群 $\{\phi_t\}$。由 $\phi_t$ 可诱导出一种导数,其目的是为了比较邻近的 $T,\phi_{t*} T$ 或等价的 $\phi_t^*T,T$。
\begin{definition}
    张量场沿切矢场 $\xi$ 的 \textbf{Lie 导数}为
\eq{
\mathscr L_\xi T:=\lim_{t\to 0}\frac{\phi_t^* T-T}{t}.
}
\end{definition}

\begin{eg}
    就标量场 $f$ 而言 $\mathscr L_\xi f=\xi(f)$。
\end{eg}

则 Killing 矢量场的各流线可视作第 $\hat x^1$ 坐标线,Killing 方程将化为
\eq{
    0=\delta^\lambda_1  \hat g_{\mu\nu,\lambda}+\hat g_{\rho \nu}\hat\del_\mu \delta^\rho
_1 +\hat g_{\mu\sigma}\hat\del_\nu\delta^\sigma_1 =\pdv{\hat g_{\mu\nu}}{\hat x^1}.
}

故直观上,Killing 矢量场的流线表示对称变换的方向,沿着该方向移动度规场仿佛没有变化。



等度量变换所构成的集合称\textbf{等度规群}。

显然 Killing 矢量场的常系数线性组合还是 Killing 矢量场,所以等度规群可对应于全体 Killing 矢量场所构成的线性空间。于是,我们只要找到一组基底即可,即最大的一组线性独立的 Killing 矢量场。可证 $n$ 维空间的独立 Killing 矢量场的总数最多为 $\binom{n+1}{2}=\frac{n(n+1)}2$,细节待。具有这一数目的空间称为\textbf{最大对称空间}。寻找 Killing 矢量场的一般方法是求 Killing 方程的通解,但对于最大对称空间可轻松找到例子,之后验证即可。

除了用 Killing 方程作为 Killing 矢量场的判据,还可以用如下方法。



假设存在一个坐标系 $\{\hat x\}$ 使处处有
\eq{
    \xi=\pdv{\hat x^1}\Longleftrightarrow \hat \xi^\mu=\delta^\mu_1,
}



实质上,只要找到某个坐标系 $\{\hat x\}$ 使得 $\hat g_{\mu\nu}$ 与某个坐标 $\hat x^1$ 无关,则 $\pdv*{\hat x^1}$ 就是一个 Killing 矢量场。


有理由相信简单的 $\mathbb{E}^n, \R^{1,3}$ 等具有最大对称性。先以 $\mathbb{E}^2$ 为例。用某自然坐标系,Killing 方程即 $\del_{(\mu}\xi_{\nu)}=0$,独立解个数为 3。最简单的情况就是平移,即满足 $\del_\mu\xi_\nu=0$ 的常矢量。独立解只需取各轴的坐标基底场即可:
\eq{
    \xi_{1}=\pdv{x},\quad \xi_{2}=\pdv{y}.
}
其流线是各坐标轴的一系列平行线。
凭借在对称性上的直觉,可知第三种将对应旋转,而切换到极坐标有 $\d \l^2=\d r^2+r^2\d\phi^2$,的确与 $\phi$ 无关,因此第三种是
\eq{
    \xi_3=\pdv{\phi}=-y\pdv{x}+x\pdv{y}.
}
读者亦可用 Killing 方程验证之。由于系数与坐标有关,因此 $\xi_3$ 独立于 $\xi_1,\xi_2$。
可预料流线 $\gamma(\phi)$ 是原点的同心圆,但下面还是严格求解之:
\[
    \dv{x}{\phi}=-y,\quad \dv{y}{\phi}=x.
\]
最简单的方法是借助复数。令 $z=x+\i y$,则 $\dv*{z}{\phi}=-y+\i x=\i z$,通解为
\[
    z=A\e^{\i(\phi-\phi_0)}\To x=A\cos(\phi-\phi_0),\quad y=A\sin(\phi-\phi_0).
\]
假设初值条件是 $p=\gamma(0)$,流线上点记 $q=\gamma(\alpha)$,则
\[
    x_q=x_p\cos\alpha-y_p\sin\alpha,\quad y_q=x_p\sin\alpha+y_p\cos\alpha.
\]
坐标观点将 $x_q^\mu$ 视作 $p$ 的新坐标 $x'^\mu_p$,因而将上式视为对任意同一点 $p$ 的新旧坐标 $\{x\},\{x'\}$ 之关系。

对于 $\mathbb E^3$,很明显 6 种独立 Killing 矢量场对应3种平移、3种平面旋转。对于 $(\R^2,\eta)$,旋转变换改为伪转动变换,极坐标对应为
\eq{x=r\cosh\theta,\quad t=r\sinh\theta,}
这样闵氏度规表为 $\d s^2=-r^2\d\theta^2+\d r^2$,称为 \textbf{Rindler 坐标系}。$\theta$ 正是快度。$\pdv*{\theta}$ 的流线为双曲线族。对于 $\R^{1,3}$,10 种独立 Killing 矢量场对应 4 种平移和 6 种平面旋转,其中平面旋转分为3 种空间旋转、3 种时空伪旋转,进而复合出任意 Poincaré 变换。不过,为了将变换的参数和物理上的速率联系在一起,只需根据 3-速定义证明 $u=\tanh\theta$ 即可。



定义流形间映射便可谈及微分同胚。微分同胚映射可自然诱导张量丛的映射,故可谈及张量变换的主动、被动观点。所有微分同胚构成微分同胚群。任意光滑矢量场是该群的生成元。显然微分同胚有无穷多生成元,故维数无穷。一个光滑矢量场在群中挑出一个单参子群,称为单参微分同胚局部群(无穷小微分同胚)。若光滑矢量场还有完备性,则给出一个单参微分同胚群。紧致流形上任意矢量场可延拓至完备的。保持度规分量不变的变换对应的微分同胚映射称为等度规映射。诱导此映射的矢量场称为 Killing 矢量。代数定义可从 Lie 导数出发,也可从 Killing 方程出发以跳过 Lie 导数。此定义便告知可用 Killing 矢量场讨论度规(时空)的对称性。

