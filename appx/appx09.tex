\chapter{纤维丛}\label{appx:bundle}

\section{丛}
丛是更复杂的矢量丛和纤维丛概念的基本底层结构。

\begin{definition}
    {\heiti 丛}(bundle)是一个三元组$(E, B, \pi)$,其中$\pi: E\to B$(或$E\xrightarrow{\pi} B$)是一个连续满射;
    拓扑空间$B$称为{\heiti 底空间},拓扑空间$E$称为{\heiti 总空间}或 {\heiti 丛空间} ,映射$\pi$称为{\heiti 丛投影};
    $\forall b\in B$,空间$\pi^{-1}(b)\in E$称为植于$ b\in B$点的{\heiti 纤维}(fibre)。
\end{definition}
直观地,我们认为丛是纤维$\pi^{-1}(b)$的并集,并被空间$E$的拓扑“粘在一起”。


\begin{definition}
     如果$E'$ 是 $E$ 的子空间,$B'$ 是 $B$ 的子空间,且 $ \pi' = \pi |_{E'}: E'\to B'$;
     那么称丛 $(E',  B',\pi')$ 是 $(E, B,\pi)$ 的{\heiti 子丛}。
\end{definition}

\begin{definition}
    若映射$s: B\to E$使得$\pi \circ s = {\rm id}$成立,则称$s$是丛$(E, B,\pi)$的{\heiti 截面}。
    换句话说,截面$s$把$b\in B$点映射成纤维$\pi^{-1}(b)$中的元素,即$s(b)\in \pi^{-1}(b)$。
\end{definition}
    
%设$(E',B,\pi')$是$(E, B,\pi)$的子丛;再设$s$是$(E, B,\pi)$的截面;
%则$s$是$(E',B,\pi')$的截面的充要条件是:$\forall b\in B$,有$s(b)\in E'$。


给一个最简单的例子,设有拓扑空间$B$和拓扑空间$F$,$B\times F$是它们的笛卡尔积空间;
则$(B\times F, B,\pi)$构成丛,其中$\pi$是$B\times F$到$B$的投影。
笛卡尔积丛$(B \times F, B,\pi)$的每个截面$s$具有$s(b) = \bigl(b,\, f(b)\bigr)$的形式,
其中映射$f: B\to F$是由$s$唯一决定的。
关于截面的这个命题证明过程大致是:
“$f: B\to F$是由$s$唯一决定的”是显然的,因为$s$就是
在每一点处指定了纤维丛中的元素,也就是说唯一决定了$f$。
假设每个映射$s: B\to B\times F$具有$s(b) = \bigl(\tau(b), \, f (b)\bigr)$的形式;
由于$b=\pi \circ s(b) = \tau(b)$,所以$s$为$B\times F$截面的
充要条件是$s(b) = \bigl(b,\, f(b)\bigr)$对于每个$b\in B$成立。

    
%这个命题说,将映射$\pi\circ s: B\to F$赋给积丛$(B \times F, B,\pi)$的每个截面$s$的函数是
%从$(B \times F,B,\pi)$的所有截面集合到映射集合$B\to F$的双射。
%如果$(E,B,\pi)$是积丛$(B\times F,B,\pi)$的子丛,则$(E,B,\pi)$的截面$s$具有$s(b) = \bigl(b, f (b)\bigr)$的形式,
%其中$f: B\to F$是一个映射,使得对于每个表$(b, f (b))\in E$。


Newton 线元在数学上称为\textbf{半度规}。

规范理论