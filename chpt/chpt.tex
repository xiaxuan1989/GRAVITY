
用 $B_{\mu\nu}$ 来分析潮汐的具体形式。$A^\mu=0$ 使 $B_{\mu\nu}=\nabla_\nu Z_{\mu}$。对其求沿线协变导数并注意 $\nabla_{\nu}\nabla_{\kappa} \omega_\mu - \nabla_{\kappa}\nabla_{\nu} \omega_\mu = - R_{\lambda\mu\nu\kappa} \omega^\lambda$:
\eq{
    Z^\kappa \nabla_\kappa \nabla_\nu Z_\mu = Z^\kappa \nabla_\nu \nabla_\kappa Z_\mu + R_{\lambda\mu\nu\kappa} Z^\lambda Z^\kappa = - B^\kappa{}_\nu B_{\mu\kappa} + R_{\lambda\mu\nu\kappa} Z^\lambda Z^\kappa.
}
缩并得到\textbf{Raychaudhuri 方程}
\eq{
    Z^\kappa \nabla_\kappa \theta = g^{\mu\nu} Z^\kappa \nabla_\kappa \nabla_\nu Z_\mu = -\frac{1}{3}\theta^2 - \sigma_{\mu\nu}\sigma^{\mu\nu} + \omega_{\mu\nu}\omega^{\mu\nu} - R_{\mu\nu} Z^\mu Z^\nu.
}
再计算扭转 $\sigma_{\mu\nu} \equiv B_{(\mu\nu)}-\frac{1}{3}\theta\,h_{\mu\nu}$ 的变化


% 分解、基本恒等式与 Raychaudhuri 后续:剪切张量的演化方程


% 投影与分解
\eq{
B_{\mu\nu} := \nabla_\nu Z_\mu
= \tfrac13 \theta\, h_{\mu\nu} + \sigma_{\mu\nu} + \omega_{\mu\nu},\qquad
\theta := B^\mu{}_\mu,\quad
\sigma_{\mu\nu}=\sigma_{\nu\mu},\ \ \sigma^\mu{}_\mu=0,\quad
\omega_{\mu\nu}=-\omega_{\nu\mu}.
}

% B 的沿流线导数(已由上文给出)
\eq{
\dot B_{\mu\nu} := Z^\kappa \nabla_\kappa B_{\mu\nu}
= - B_\mu{}^{\lambda} B_{\lambda\nu} - R_{\lambda\mu\nu\kappa} Z^\lambda Z^\kappa .
}

% 取对称无迹投影即可得到剪切的演化
\eq{
\dot\sigma_{\mu\nu}
:= \left(\dot B_{\alpha\beta}\right)_{\langle\mu\nu\rangle}
= -\left(B_\mu{}^{\lambda} B_{\lambda\nu}\right)_{\langle\mu\nu\rangle}
      -\left(R_{\lambda\mu\nu\kappa} Z^\lambda Z^\kappa\right)_{\langle\mu\nu\rangle}.
}

% 先处理二次项 (代入 B 的分解并在 h-子空间做代数化简)
\eq{
\left(B_\mu{}^{\lambda} B_{\lambda\nu}\right)_{\langle\mu\nu\rangle}
= \tfrac{2}{3}\,\theta\,\sigma_{\mu\nu}
  + \sigma_\mu{}^{\lambda}\sigma_{\lambda\nu}
  + \omega_\mu{}^{\lambda}\omega_{\lambda\nu}
  - \tfrac13 h_{\mu\nu}\!\left(\sigma_{\alpha\beta}\sigma^{\alpha\beta}
                              - \omega_{\alpha\beta}\omega^{\alpha\beta}\right).
}

% 再处理曲率项:分解为 Weyl 的电部分与 Ricci 的 STF 投影
\eq{
\left(R_{\lambda\mu\nu\kappa} Z^\lambda Z^\kappa\right)_{\langle\mu\nu\rangle}
= E_{\mu\nu}
+ \tfrac12\!\left(h_\mu{}^{\alpha} h_\nu{}^{\beta}
      - \tfrac13 h_{\mu\nu} h^{\alpha\beta}\right) R_{\alpha\beta},
\qquad
E_{\mu\nu}:= C_{\mu\alpha\nu\beta} Z^\alpha Z^\beta .
}

% 将以上两式代回得到剪切演化的闭式方程
\eq{
\boxed{
\begin{aligned}
Z^\kappa \nabla_\kappa \sigma_{\mu\nu}
=&\ - \tfrac{2}{3}\,\theta\,\sigma_{\mu\nu}
   - \sigma_\mu{}^{\lambda}\sigma_{\lambda\nu}
   - \omega_\mu{}^{\lambda}\omega_{\lambda\nu}
\\[2pt]
&\ + \tfrac13 h_{\mu\nu}\!\left(\sigma_{\alpha\beta}\sigma^{\alpha\beta}
                               - \omega_{\alpha\beta}\omega^{\alpha\beta}\right)
   - E_{\mu\nu}
   - \tfrac12\!\left(h_\mu{}^{\alpha} h_\nu{}^{\beta}
      - \tfrac13 h_{\mu\nu} h^{\alpha\beta}\right) R_{\alpha\beta}.
\end{aligned}
}
}




\newcommand{\ang}[1]{\left\langle #1 \right\rangle} % 对称无迹部分

% 定义“沿流正交、对称、无迹投影”(尖括号):
\begin{equation}
X_{\ang{\mu\nu}} \equiv
\left(
h_{(\mu}{}^{\alpha} h_{\nu)}{}^{\beta}
-\tfrac{1}{3} h_{\mu\nu} h^{\alpha\beta}
\right) X_{\alpha\beta},
\qquad
\dot{\sigma}_{\ang{\mu\nu}} \equiv
\left(
h_{(\mu}{}^{\alpha} h_{\nu)}{}^{\beta}
-\tfrac{1}{3} h_{\mu\nu} h^{\alpha\beta}
\right) Z^\kappa \nabla_\kappa \sigma_{\alpha\beta}.
\end{equation}

% 取(\ref{eq:dotB-master})的 ST(对称—无迹—正交)部分得到剪切的演化:
% 先给出 Weyl 张量的电部以分离曲率的无迹正交部分
\begin{equation}
E_{\mu\nu} \equiv C_{\mu\alpha\nu\beta} Z^\alpha Z^\beta,
\qquad
\mathcal{R}_{\ang{\mu\nu}} \equiv
\left(
h_{(\mu}{}^{\alpha} h_{\nu)}{}^{\beta}
-\tfrac{1}{3} h_{\mu\nu} h^{\alpha\beta}
\right) R_{\alpha\beta},
\end{equation}
其中 $C_{\mu\alpha\nu\beta}$ 为 Weyl 张量,$\mathcal{R}_{\ang{\mu\nu}}$ 是 ${R}_{\alpha\beta}$ 的“横向对称无迹”部分。

% 经过代数整理,可得剪切(\sigma_{\mu\nu})的 Raychaudhuri 型演化方程:
\begin{align}
\dot{\sigma}_{\ang{\mu\nu}}
&= -\tfrac{2}{3}\,\theta\,\sigma_{\mu\nu}
   - \sigma_{\alpha\ang{\mu}\,}\sigma_{\nu\ang{}}{}^{\alpha}
   - \omega_{\alpha\ang{\mu}\,}\omega_{\nu\ang{}}{}^{\alpha}
   + \tfrac{1}{3} h_{\mu\nu}\left(\sigma_{\alpha\beta}\sigma^{\alpha\beta}
   - \omega_{\alpha\beta}\omega^{\alpha\beta}\right)
\nonumber\\
&\quad
+ E_{\mu\nu}
- \tfrac{1}{2}\,\mathcal{R}_{\ang{\mu\nu}} .
\label{eq:shear-evolution-ricci}
\end{align}

% 若使用 Einstein 方程将 Ricci 项写成物质源的各向异性应力部分,
% 取能量动量张量的 1+3 分解
% T_{\mu\nu}=\rho Z_\mu Z_\nu+p\,h_{\mu\nu}+2q_{(\mu}Z_{\nu)}+\pi_{\mu\nu},
% 其中 \pi_{\mu\nu} 为正交、对称、无迹张量,则
\begin{equation}
\mathcal{R}_{\ang{\mu\nu}}
= 4\pi G\left( \rho + 3p \right)_{\ang{\mu\nu}}
+4\pi G\left(-2\,\pi_{\mu\nu}\right)
\ \Longrightarrow\
-\tfrac{1}{2}\,\mathcal{R}_{\ang{\mu\nu}}
= +2\pi G\,\pi_{\mu\nu} \quad (\text{无热流 }q_\mu=0 \text{ 且取适当单位}).
\end{equation}
因此在物质源表示下,剪切演化式常写为
\begin{equation}
\dot{\sigma}_{\ang{\mu\nu}}
+ \tfrac{2}{3}\theta\,\sigma_{\mu\nu}
+ \sigma_{\alpha\ang{\mu}\,}\sigma_{\nu\ang{}}{}^{\alpha}
+ \omega_{\alpha\ang{\mu}\,}\omega_{\nu\ang{}}{}^{\alpha}
- \tfrac{1}{3} h_{\mu\nu}\left(\sigma_{\alpha\beta}\sigma^{\alpha\beta}
- \omega_{\alpha\beta}\omega^{\alpha\beta}\right)
+ E_{\mu\nu}
- 4\pi G\,\pi_{\mu\nu}
=0 .
\label{eq:shear-evolution-matter}
\end{equation}

% 两个常见极限:
% 1) 真空(或无各向异性应力)且测地、无扭转(\omega_{\mu\nu}=0)的丛:
\begin{align}
\text{(真空且 }\omega_{\mu\nu}=0\text{)}:\qquad
\dot{\sigma}_{\ang{\mu\nu}}
+ \tfrac{2}{3}\theta\,\sigma_{\mu\nu}
+ \sigma_{\alpha\ang{\mu}\,}\sigma_{\nu\ang{}}{}^{\alpha}
+ E_{\mu\nu}=0 .
\end{align}

% 2) 共形平坦(E_{\mu\nu}=0)且各向同性流体(\pi_{\mu\nu}=0):
\begin{align}
\text{(共形平坦且 }\pi_{\mu\nu}=0\text{)}:\qquad
\dot{\sigma}_{\ang{\mu\nu}}
= - \tfrac{2}{3}\theta\,\sigma_{\mu\nu}
  - \sigma_{\alpha\ang{\mu}\,}\sigma_{\nu\ang{}}{}^{\alpha}
  - \omega_{\alpha\ang{\mu}\,}\omega_{\nu\ang{}}{}^{\alpha}
  + \tfrac{1}{3} h_{\mu\nu}\!\left(\sigma_{\alpha\beta}\sigma^{\alpha\beta}
  - \omega_{\alpha\beta}\omega^{\alpha\beta}\right).
\end{align}

% 为完整性,给出挠率(扭转)的演化式(测地丛):
\begin{equation}
\dot{\omega}_{\mu\nu}
\equiv h_{\mu}{}^{\alpha} h_{\nu}{}^{\beta} Z^\kappa \nabla_\kappa \omega_{\alpha\beta}
= -\tfrac{2}{3}\theta\,\omega_{\mu\nu}
- 2\,\sigma_{\alpha[\mu}\,\omega_{\nu]}{}^{\alpha} .
\label{eq:twist-evolution}
\end{equation}

% 上式 (\ref{eq:shear-evolution-ricci})–(\ref{eq:twist-evolution}) 与文中给出的标量 Raychaudhuri 方程一道,
% 给出了测地丛形变(\theta,\sigma_{\mu\nu},\omega_{\mu\nu})的完整动力学。



\eq{
B_{\mu\nu}=\frac{1}{3}\theta h_{\mu\nu}+\sigma_{\mu\nu}+\omega_{\mu\nu},\quad u^\mu=\frac{1}{3}\theta\eta^\mu+\sigma^\mu{}_\nu\eta^\nu+\omega^\mu{}_\nu\eta^\nu.
}



Weyl 张量

曲率张量不因真空方程而一定为零的部分,称为 Weyl 张量。这个张量量度了测地线组的“潮汐”扭曲。所以真空区域里的引力场的“局部强度”在 Newton 极限之下是与宏观的试验物质的潮汐力相关的,而不是与引力的范数相关的.