

用 $B_{\mu\nu}$ 来分析潮汐的具体形式。$A^\mu=0$ 使 $B_{\mu\nu}=Z_{\mu;\nu}$。对其求沿线协变导数:
\eq{
    Z^\lambda \nabla_\lambda (\nabla_\nu Z_\mu) = R^\lambda{}_{\mu\nu\kappa}Z^\mu = \nabla_{\nu}\nabla_{\kappa}Z^\lambda - \nabla_{\kappa}\nabla_{\nu}Z^\lambda
}

Raychaudhuri 方程

Weyl 张量

曲率张量不因真空方程而一定为零的部分,称为 Weyl 张量。这个张量量度了测地线组的“潮汐”扭曲。所以真空区域里的引力场的“局部强度”在 Newton 极限之下是与宏观的试验物质的潮汐力相关的,而不是与引力的范数相关的.