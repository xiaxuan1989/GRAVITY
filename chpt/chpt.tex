
考虑 Lorenz 规范下的无源方程 $\square A_\mu = 0$。
由 Fourier 定理,$[0,L]$ 上任意良好的函数总可写作 $f(x)=\sum c_n \exp(\i \frac{2\pi n}{L}x)$。
推广到 $\R$ 上就是 $f(x)=\int_\R \tilde f(k)\e^{ikx}\d k$,$\R^3$ 就是 $f(\bm x)=\int_\R \tilde f(\bm k)\e^{i\bm k\cdot\bm x}\d[3]k$。
将 $\e^{i\bm k\cdot\bm x}$ 称为基底平面波,因其同相位点之集 $\bm x$ 满足 $\bm k\cdot\bm x$ 为常数,即波前为平面,属于一种平面波;其次,任意 $f(x)$ 可以 $\e^{i\bm k\cdot\bm x}$ 为基底,展开系数为各 Fourier 变换结果 $\tilde f(\bm k)$。

同理,$A_\mu(x)=\int a_\mu(k) \e^{\i\theta} \d[4]k$,其中 $\theta = k^\mu x_\mu \in \R$。所以 $\square A_\mu = \int a_\mu \e^{\i\theta} \square\theta \d[4]k$



$A_\mu = \Re{C_\mu \e^{\i\theta}}$,其中 $C_\mu\in\C$ 称为\textbf{偏振矢量},$\del_\mu C^\mu = 0$。

对于光,虽不能定义固有时和质量概念,但可借量子理论的 de Broglie 关系定义其能量和动量。设频率 $\nu$ 和波矢 $k^i$,则能量为 $E=h\nu$,3-动量为 $p^i = \hbar k^i$。4-动量仍满足 $P^0=E,P^i=p^i$。

实际上,描述光线时一般取任意仿射参数 $\lambda$。比如,考虑一点 $x_{0}$,此处一个指向未来的类光矢量 $N$ 就可代表光的走向。由 $\alpha(\lambda)=x_{0}+\lambda N$ 定义的直线 $\alpha$ 就是自由光子经过 $x_{0}$ 的世界线。

先回顾 3 维语言并考虑 SI。$\bm{E},\bm{B}$ 是真空中波动方程矢值解的事实表明:我们找到了 Maxwell 方程组的 “平面 波”解,其中
\eq{
\bm{E}=\bm{\alpha} \cos \Omega+\bm{\beta} \sin \Omega,
}
式中 $\bm{\alpha},\bm{\beta}$ 是常矢量,而 
\eq{
\Omega=(\omega t-\omega \bm{r} \cdot \bm{e}),\quad \bm{e} \cdot \bm{e}=1,
}
$\omega>0,\bm{\alpha},\bm{\beta}$ 和 $\bm{e}$ 是常量;$\omega$ 是频率,$\bm{e}$ 是给出传播方向的单位矢量($t$ 增加 $\tau$ 且 $\bm r$ 增加 $c \tau \bm{e}$ 保持 $\Omega$ 不变)。这满足波动方程,如果常量不可普遍选取,将不可能找到使真空方程也成立的 $\bm B$。取上式的散度得
\eq{
\div \bm{E}=\frac{\omega}{c}(\bm{e} \cdot \bm{\alpha} \sin \Omega-\bm{e} \cdot \bm{\beta} \cos \Omega)=0,
}
即必须选择 $\bm{\alpha},\bm{\beta}$ 使其垂直于 $\bm{e}$。而另一方面,必须找到 $\bm{B}$ 使
\eq{
\curl \bm{E}=\frac{\omega}{c}(\bm{e} \times \bm{\alpha} \sin \Omega-\bm{e} \times \bm{\beta} \cos \Omega)=-\frac{\partial \bm{B}}{\partial t},
}
一种可能的选择是
\eq{
\bm{B}=\frac{\bm{e} \times \bm{E}}{c}=\frac{1}{c}(\bm{e} \times \bm{\alpha} \cos \Omega+\bm{e} \times \bm{\beta} \sin \Omega).
}
不难看出 $\bm{E},\bm{B}$ 同样满足真空方程。这种方法得到的解称为\textbf{单色平面电磁波}。注意,$\bm{E}$ 和 $\bm{B}$ 垂直于传播方向,在此意义下这样的波是横波,$\bm{E}$ 的定义可更简要地写成
\eq{
\bm{E}=\Re{(\bm{\alpha}+\i\beta) \e^{-\i \Omega}},
}
用 Fourier 分析可证明真空中每个解都是单色平面波的组合。如果 $\bm{\alpha}$ 和 $\bm{\beta}$ 成正比,平面波是\textbf{“平面”偏振}的或\textbf{“线”偏振}的;如果 $\bm{\alpha} \cdot \bm{\alpha}=\bm{\beta} \cdot \bm{\beta},\bm{\alpha} \cdot \bm{\beta}=0$,平面波则是\textbf{“圆”偏振}的。

Maxwell 理论的核心概念是具有确定频率或一定色的光波可用 Maxwell 方程组的单色平面解来表示。


%% 

\textbf{标架}(frame,四维时称 tetrad)一般指非坐标基的任意基底。取四维标架 $\{e_I, I =0,1,2,3\}$。我们主要考虑正交归一的标架,即
\eq{
    g(e_I,e_J)=\eta_{I J},
}
若某处 $e_0$ 类时并指向未来,则该处标架还可以看作一个\textbf{瞬时观者},记作 $(p,\{e_I\})$,其中$e_0=Z$。不强调空间标架时可记作 $(p,Z)$。一般考虑瞬时观者们的空间标架的定向也一致。

\begin{wrapfigure}{l}{.35\textwidth}
    \includegraphics[width=.3\textwidth]{fig/chpt01/frame.pdf}
    \caption{空间右手 3-标架}
\end{wrapfigure}

一个连续标架场就相当于参考系。

综上,无自转观者所要求的无非是 4-标架沿线费移。以后谈及等效原理会详述无自转的重要性,如可以避开科氏惯性力之干扰。


因为参考系由充斥于时空(或其开子集)的无数观者组成,过每一时空点有且仅有一条观者世界线,所以,给定一个参考系后,全时空(或其开子集)就有一个 4-标架场。任何时空点的任何张量都可用该点的4-标架作为基底表出。
 


\begin{definition}[惯性观者]
    自由落体观者是世界线为测地线的无自转观者。
\end{definition}


\begin{figure}[ht]\centering
    \subfloat[\centering Cynthia]{\includegraphics[height=.3\textwidth]{fig/chpt01/eggchair.jpg}}
    \subfloat[\centering Taylor]{\includegraphics[height=.5\textwidth]{fig/chpt01/spinning_chair.pdf}}
    \caption{\small 设 Cynthia、Taylor 两人坐在地面的两把椅子上:Cynthia 坐底座固定的蛋壳椅;Taylor 坐底座静置的办公椅且不停自转。二者世界线都走直线,但 Taylor 不能作为惯性观者。}
\end{figure}


非测地线意味着 3-速在变化,进而一点的瞬时观者可 boost 到邻近点的另一瞬时观者去,这样,此点观者所持的空间矢量便自然能延至下一观者中去。boost 变换下,$e_\mu$ 间当然能保持正交,说明各点瞬时观者或 4-标架场各点取值间差的正是\textit{无穷小(infinitesimal) boost 变换}。
将非测地线看作是一系列瞬时惯性观者的无穷小 boost 变换,轻松使得 $e_i$ 保持其空间性。并且,由于点之间接近,boost 也不包含空间转动(只受 3-速影响),故各 $e_i$ 朝向不变。

$Z,A$ 正交使得 $Z$ 保模长地发生“转动”,因而空间矢量为与之正交而一齐转动。这种转动只涉及到 $Z,A$ 所在(无穷小)平面,此即,boost 变换应是关于 $Z,A$ 平面的伪转动。3 维空间中旋转变换是相关于一个角速度矢量而言的,即
\eq{
\frac{D V_i}{\d\tau}=\varepsilon_{ijk}\omega^j V^k,
}
其同时可看作处于 $\vec\omega$ 所对平面上的旋转(按右手螺旋),这个视角在代数上就是在 3 维空间中取其对偶的 2-形式(当然,先用旋转平面的诱导度规降下来),即 $\Omega=\star\vec\omega$,分量为 $\Omega_{ik}=\omega^j\varepsilon_{jik}=-\omega^j\varepsilon_{ijk}$。这样
\eq{
\frac{D V^i}{\d\tau}=-\Omega^{ij} V_j,
}
这里旋转平面的特点是同 $\vec\omega$ 垂直,这意味着 $\vec\omega$ 正比于平面上任二线性无关矢量之叉乘,则 $\Omega$ 正比于其楔积。将其推广至高维时,旋转平面有其确切含义(找到两个线性无关矢量),而角速度矢量的对偶不再是平面,故表述上采取 2-形式而非矢量叉乘:
\eq{
\frac{D V^\mu}{\d\tau}=\varepsilon_{ijk}\omega^j V^k=-\Omega^{\mu\nu} V_\nu=-(A^\mu Z^\nu-Z^\mu A^\nu)V_\nu,
}

瞬时观测

\section{能动张量与场方程}

现在得到理想流体的能动张量为
\eq{
    T^{\mu\nu}=(\rho+p)U^\mu U^\nu + p g^{\mu\nu}.
}

然而,在相对论中,密度这个物理量微妙得多。虽然广义相对论总讨论宏观乃至宇观视角,物质作为引力场源的确适用于连续介质模型。
下文我们先回到 $\R^{3+1}$。以能量密度为例,设在瞬时静系下,单元 $\Delta V$ 内的质量为 $\Delta m$,因此质量(静能)密度就是 $\rho=\Delta m/\Delta V$。但当该单元相对于某惯性系的 3-速为 $\bm{u}$ 时有 $\Delta E=\gamma \Delta m$,且单元内沿 $\bm u$ 方向的长度会收缩,体积变为 $\Delta V/\gamma$。故能量密度应为 $\rho_{EM}=\gamma\Delta E/\Delta V=\gamma^2\rho$。可见从瞬时静系到其它惯性系,$\rho_{EM}$ 的变换含有两个 $\gamma=\Lambda^0{}_{0}$,因为体积与能量二者都进行了变换。因此能量密度不可能是某个矢量的分量,而应是某个 2 阶张量的分量。这便能体现 $T$ 的必要性。总之,各种密度应视作绝对张量 $T$ 在某参考系的相对分量。

可见 $T$ 应是表示物质场某种密度属性的张量。

无论如何,$T$ 应能全面地描述物质场的各种属性。那么有哪些属性呢?除质量密度 $\rho$ 外,常见的还有\textit{应力(stress)}、动量密度这样一些量,这都是不能单靠质量密度描述的。注意,按一般认知,这似乎应看作动力效应而非物质场属性,但这其实是在用质点模型看待问题,并不正确。

此时当然是诸如密度、应力、温度等的概念更为有用。
不仅如此,应力应当彻底视作物质场属性有很多支撑性证据,比如,以后会知道,压强(正应力)与能量密度之间的关系 $p=w\rho$ 直接决定了辐射、普通物质、宇宙学常数等不同成分对宇宙膨胀速度的影响。
 



直观上讲,张量这个词最初就是由 Cauchy 等人研究连续介质应力理论而提出的。此前提过应力即某微小平面(称界面)所分割相邻部分的作用力。故应力应当依赖于两个方向:一是作用力方向,一是界面方向(用其法矢代表)。某作用力和某面垂直时便是该方向的\textit{正应力},相切时便称\textit{切应力}。某界面上的应力总可分解为此二者。考察某空间点,若应力在此各个方向都是正应力,我们就可降低自由度,即应力只需依赖一个矢量。若在瞬时静系中,各方向正应力还一致(即\textit{各向同性}),那就可直接用标量 $p$ 表示正应力,即\textit{压强(pressure)}。但很显然,在一般情形下应力不是矢量,因为总需借助两个独立矢量,所以应力是 2 阶张量。
因 $\left\{\left(e_i\right)^a\right\}$ 正交归一,不难证明 $T^{i j}=T_{i j}$。设 $\left\{\left(e^\mu\right)_a\right\}$ 为 $\left\{\left(e_\mu\right)^a\right\}$ 的对偶基底,我们 来讨论空间张量 $\hat{T}^{a b} \equiv T^{i j}\left(e_i\right)^a\left(e_j\right)^b$ [或 $\hat{T}_{a b} \equiv T_{i j}\left(e^i\right)_a\left(e^j\right)_b$ ]的物理意义。以 $\Delta S$ 代表与 $\left(e_i\right)^a(i$ 为 $1,2,3$ 中之任一) 垂直的空间单位面元,由正文可知
\[
T^{i j}=T_{i j}=\Delta S \text { 一侧物质对他侧物质的力的 } j \text { 分量,}
\]
因此 $\hat{T}^{a b}$ 应解释为 3 应力张量。另一方面,
\[
T^{i j}=\hat{T}^{a b}\left(e^i\right)_a\left(e^j\right)_b=\left[\hat{T}^{a b}\left(e^i\right)_b\right]\left(e^j\right)_a=\hat{T}^{a b}\left(e^i\right)_b \text { 的 } j \text { 分量。}
\]
式(6-4-2)和(6-4-3)结合给出
\[
\hat{T}^{a b}\left(e^i\right)_b \text { 的 } j \text { 分量 }=\Delta S \text { 一侧物质对他侧物质的力的 } j \text { 分量,}
\]
故
\[
\hat{T}^{a b}\left(e^i\right)_b=\Delta S \text { 一侧物质对他侧物质的作用力。}
\]
而作用力无非是被作用者的 3 动量的变化率,相互作用无非是相互交换 3 动量,所以
\[
\begin{aligned}
\hat{T}^{a b}\left(e^i\right)_b & =\text { 单位时间内沿 }\left(e_i\right)^a \text { 方向穿过与 }\left(e_i\right)^a \text { 直的单位面积的 } 3 \text { 动量 } \\
& =\text { 沿 }\left(e_i\right)^a \text { 方向的 } 3 \text { 动量流密度。}
\end{aligned}
\]
上式中的 $\left(e_i\right)^a$ 可以是任意空间方向的单位矢量,所以上式表明沿任一空间方向的 3 动量流密度都可由 $\hat{T}^{a b}$ 与该方向的单位矢量缩并而得,于是可把 $\hat{T}^{a b} \equiv T^{i j}\left(e_i\right)^a\left(e_j\right)^b$
故 3-应力张量就是 3-动流密度。
可见,我们不仅能从 $\gamma$ 数量上判断张量阶数,还能从所需的矢量个数来判断。比如,此前说能量需要变换一个 $\gamma$,就是由于能量是 4-动量的时间分量,换句话说,$\bm P$ 需要点乘一个矢量,才能定义能量 $E=P^0=-\bm P\cdot \bm e_0$。流密度就是通过某单位面元的流量,因此需要定义表面的矢量。于是定义能流密度需要两个矢量。同理定义动量密度、动流密度亦需要两个矢量。可见这些密度都有希望统一为 2 阶张量,$T$ 的地位再次坐实。如上这样用“输入矢量”观点来看待张量的方式称作\textit{映射语言},实际上,我们在张量积定义上已瞥见这一点。 

总结下来,现有这样一些密度量等待统一:能量密度(质量密度为其瞬时静系情形)、3-动量密度(能流密度)、3-动流密度(应力张量)。注意能量、动量是通过 4-动量统一的,我们自然想到类似于“4-动量密度”的概念。又注意量密度、3 维流密度其实可统一成某种“4 维”流密度,分别作为其时间、空间分量,即量密度是“时间方向”的流密度。可见答案呼之欲出:$T$ 应是 \textit{4-动流密度}。定义一个张量需借助坐标系。取任意的某坐标系,$T$ 的分量就是\textit{4-动量 $\mu$ 分量通过面元 $\d x^\nu=0$ 的流密度},为指标平衡,记作 $T^{\mu\nu}$。

传递性在宇观物质中的地位显赫。$T$ 就是描述各类物理量如何依附于物质,而沿不同坐标传递。换句话说,我们可将物质运动、相互作用等视为能流的体现。根据定义可知,$T^{00}$ 就是能量密度 $\rho_{EM}$;$T^{0i}$ 就是能流密度,$T^{i0}$ 就是 3-动量密度,二者相等,记作 $w^i$;$T^{ij}$ 就是 3-动流密度,即应力张量。若无切应力,则应力均沿面元法矢,此时只有对角元非零,故对角元表示正应力。可见 $T^{ij}$ 中非对角元部分表示切应力。这些量显然是相对的:物质在瞬时静系下也要不断地往未来迁移,(静止)能量密度非零;在相对运动系下,则观者将看到部分能量会流经空间方向。$T$ 在整个时空上绝对,但相对运动系的观者却测出动能的部分(体现为能流密度)。可见,虽的确可定义 \textit{4-动量密度矢量}来统一能量密度、能流密度(3-动量密度),但该矢量却并非是 4-动量那样的绝对矢量,即该矢量本身就与观者有关。取任意瞬时观者 $(p,\bm e_\mu)$,其中 $\bm e_0=\bm Z$,则我们应将 $T$ 往其标架投影。取坐标基与其标架一致,则
\eq{
Z^\mu=\delta_0^\mu,\quad (\bm e_i)^\mu=\delta_i^\mu.
}
其对偶满足
\eq{
    Z_\nu=\eta_{\mu\nu}Z^\mu=\eta_{0\nu},\quad (\bm e^j)_\nu=\delta^j_\nu.
}
其中注意 $\bm Z$ 按 $\eta$ 的对偶与对偶坐标基矢 $\bm e^0$ 差负号。考虑 $T$ 协变形式的分量 $T_{\mu\nu}$,它都对应 $T$ 与 $\{\bm e_\mu\}$ 的缩并,即
\begin{align}
    T_{\mu\nu}Z^\mu Z^\nu&=T_{00}=\eta_{0\mu}\eta_{0\nu}T^{\mu\nu}=T^{00}=\rho_{EM},\\ 
    T_{\mu\nu}Z^\mu(\bm e_i)^\nu&=T_{0i}=\eta_{0\mu}\eta_{i\nu}T^{\mu\nu}=-\delta_{ij}T^{0j}=-w_i,\\
    T_{\mu\nu}(\bm e_i)^\mu(\bm e_j)^\nu&=T_{ij}=\eta_{i\mu}\eta_{j\nu}T^{\mu\nu}=\delta_{ik}\delta_{jl}T^{kl}.
\end{align}
如果 $(p,\bm Z)$ 是瞬时静系,则 $p$ 处质元 3-动量为零,因此 4-动量只剩时间分量 $\Delta m$,则 $T_{\mu\nu}Z^\mu Z^\nu=T_{00}=\rho$ 就是质量密度。可见 $T^{\mu\nu}$ 的定义的确能满足在测地线汇中 $T_{\mu\nu}Z^\mu Z^\nu=\rho$。定义瞬时观者所测 4-动量密度 $\bm W$ 是 $\rho_{EM},w^i$ 的整合,即
\eq{
    \bm W=\rho_{EM}\bm Z+\bm w,
}
这可视作 $\bm W$ 在相应瞬时观者标架下的分解。可作如下改写:
\begin{align*}
    \bm W
    &=T_{\mu\nu}Z^\mu Z^\nu \bm e_0-\delta^{ij}T_{\mu\nu}(\bm e_j)^\mu Z^\nu\bm e_i\\
    &=T^{\mu\nu}Z_\mu Z_\nu \bm e_0-T^{\mu\nu}(\bm e^i)_\mu Z_\nu\bm e_i\\
    &=-T^{0\nu}Z_\nu\bm e_0-T^{i\nu}Z_\nu\bm e_i,
\end{align*}
即
\eq{\label{eq:W=TZ}W^\mu=-T^{\mu\nu}Z_\nu=T^{\mu 0}.}
利用此式可讨论能量守恒。能量等在某时空点的流经方向可归结为每个坐标的正负方向,但对封闭体系而言能量守恒,故其必定从这儿进多少,从这儿就得出多少,这时只需考察四个坐标的正方向即可。

易从 Maxwell 方程推出\textit{连续性(continuity)方程} $\partial_t\mu+\div\bm j=0$ 并导出电荷守恒,类比这一点,能量守恒即要求
\eq{
    \partial_\mu W^\mu=\partial_0\rho_{EM}+\partial_i w^i=0.
}

任意选取坐标系不改变零张量事实。为在 $p$ 附近谈及协变导数,不妨取 $(p,\bm Z)$ 所对的瞬时惯性系,则 \eqref{eq:W=TZ} 式所用 $Z_\nu=\eta_{0\nu}$ 在 $p$ 及附近皆成立,故
\[
    \partial_\mu W^\mu=\partial_\mu T^{\mu 0}=0,
\]
可见能量守恒等价于 $\partial_\mu T^{\mu 0}=0$。除此之外,封闭系统还应有动量、角动量的守恒。同理知动量守恒表为 3-动量时间偏导加上“3-动流密度空间散度”的形式,而注意此即
\eq{
\partial_\mu T^{\mu i}=\partial_0 w^i+\partial_j T^{ij}=0.
}
第二项便是 3-动流密度的空间散度。是的,张量亦可谈及散度,当然,一般要注明是对哪一指标缩并。然而注意 $T^{\mu\nu}$(进而 $T^{ij}$)是对称的,故无关于缩并位置,可视作仅此一种求散方式。角动量是角物理量而非线物理量,因此其守恒的表述有些微妙。实际上,只要注意 $T^{ij}=T^{ji}$ 代表\textit{空间各向同性}(球对称性),这对应于角动量守恒。有关对称性与守恒律的联系,我们在第 \ref{sec:sym-noe} 节再详述。综上,对与外界无相互作用的封闭物质场而言,总要求 $T_{\mu\nu}$ 的 4 维散度为零:
\eq{
\partial^\mu T_{\mu\nu}=0,
}
根据最小替换法则,一般情形下即要求
\eq{
\nabla^\mu T_{\mu\nu}=0,
}
像这样协变微分与张量某个指标缩并,就称该张量(关于该指标)的\textit{协变散度}。然而情况却些许微妙。张量写法上可记作 $\div T=\bm 0$。一般的 $g$(弯曲时空)极有可能不具对称性,进而仍不能导出整体的守恒律。上式体现的是在微分水平上的守恒,说明守恒律一般只在无穷小层次上保留。这件事使我们对这些物理定律的本性又有了新的洞察。确实有学者针对这一点批评了 $\nabla^\mu T_{\mu\nu}=0$ 这种假设(比如,至今很难完全弄清引力场自身的能动张量的自洽定义),但否定 $\nabla^\mu T_{\mu\nu}=0$ 一定意味着引力理论的不同。广相要求 $\nabla^\mu T_{\mu\nu}=0$,因为马上将看到,作为广相核心的场方程是以此为前提的。

考虑质量场集中在一点 $\bm x_p$ 的极限情况,或者说能动张量只在一条世界线上非零的极限情况,此时
\[
T^{\mu\nu}=\rho U^\mu U^\nu,
\]
其中 $\rho=m\delta(\bm x-\bm x_p)$。$\nabla_\mu T^{\mu\nu}=0$ 给出 $0=\rho U^\mu \nabla_\mu U^\nu+ U^\nu (U^\mu\partial_\mu\rho+\rho\nabla_\mu U^\mu)$。同乘 $U_\nu$,第一项便是 $U_\nu A^\nu$,仿照 \eqref{3412.UA=0} 式知二者正交。因而对第二项,一定有
\[
U^\mu\partial_\mu\rho+\rho\nabla_\mu U^\mu=0,
\]
故
\[
\rho U^\mu \nabla_\mu U^\nu=0.
\]
可作对 $\bm x$ 的体积分,由挑选性知在 $\bm x_p$ 的世界线上必有 $m U^\mu \nabla_\mu U^\nu=0$,$m$ 的非零导致轨迹必满足 $U^\mu \nabla_\mu U^\nu=0$,说明守恒流将是 $g$ 下的测地线。在这个极限下,不论能动张量的本性如何,物质的运动都是由 $g$ 的几何学所决定的。这样,所有的对象都以同样的方式“下落”,就如 Galileo 实验中重物的下落与质量无关一样。这些考虑致使等效原理有了具体的体现。

在量子电动力学里,电子传播子的一回圈自能图给出
\[
\Sigma(\slashed p)
= (-ie^2)\!\int\!\frac{d^d k}{(2\pi)^d}\,
\gamma^\mu\frac{\slashed p-\slashed k + m}{(p-k)^2-m^2+i0}\gamma_\mu\,
\frac{1}{k^2+i0},
\]
用维数正规化 $d=4-2\varepsilon$ 计算,在 Feynman 规下可化为(略去与 $\slashed p-m$ 成正比的项):
\[
\Sigma(\slashed p)=\frac{\alpha}{4\pi}\Big[\big(\tfrac{1}{\varepsilon_{\rm UV}}-\gamma_E+\ln4\pi+\ln\frac{\mu^2}{m^2}\big)(-\slashed p+4m)
+\text{有限项}\Big].
\]
为保证物理质量是传播子的极点位置(on-shell 方案),引入质量对消子 $\delta m$ 与场强重整 $Z_2$,令
\[
S_R^{-1}(\slashed p)=\slashed p-m_{\rm phys}-\big(\Sigma(\slashed p)-\delta m\big),
\]
并施加
\[
S_R^{-1}(\slashed p)\big|{\slashed p=m{\rm phys}}=0,\qquad
\frac{d S_R^{-1}}{d\slashed p}\Big|{\slashed p=m{\rm phys}}=1.
\]
解得(on-shell 方案)
\boxed{\;\delta m_{\rm OS}
= m_{\rm phys}\,\frac{3\alpha}{4\pi}
\Big(\frac{1}{\varepsilon_{\rm UV}}-\gamma_E+\ln4\pi+\ln\frac{\mu^2}{m_{\rm phys}^2}+\frac{4}{3}\Big)\;}
(以及相应