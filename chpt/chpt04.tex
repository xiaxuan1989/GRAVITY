\chapter{实验理论}\label{chpt:experiment}

\subsection{对钟}
就某一时空点而言可设置 4-速为 $Z^\mu$ 的观者,称为\textbf{瞬时观者}.若瞬时观者与观测对象在该处的 4-速 $U^\mu$ 相同,则称它是观测对象的\textbf{瞬时固有观者}.这样,设荷电系统的运动由 $U^\mu$ 描述,瞬时共动观者在其中某处测到的电荷密度 $\rho_{e0}$ 就是\textbf{固有电荷密度}.处处设置瞬时共动观者就形成了\textbf{固有系},则固有电荷密度构成一个标量场.


$U_\mu Z^\mu=g_{00} U^0 Z^0$.为保持模方 $-1$,在此坐标系下有 $Z^0=1/\sqrt{-g_{00}}$.

的 3-速差距可通过




其等价定义是
\eq{
    J^\mu=\rho_{e0} U^\mu.
}
 

以瞬时观者为参考.

尺缩效应,有$\rho_{e0}=\rho_e/gamma$,其中 $gamma=U^0=-U_\mu Z^\mu$.

以后会证明,$\bm g$ 可在任意世界线上表为 $\eta_{\alpha\beta}$,这种坐标系准确称为固有坐标系,但若还要在该系下满足 $\partial_\lambda\eta_{\alpha\beta}=0$(进而 $\Gamma^\lambda_{\mu\nu}=0$),则世界线一定是测地线,且固有坐标系无自转.

\eq{\label{U^0=1}
U^0=\frac{1}{\sqrt{-g_{00}}}\approx 1,
}
其中系数是为使 $g_{00}U^0U^0=-1$

观者 $G$ 自带 4-速场,可直接观测 $p\in G$ 处发生的任何事情。比如,设研究对象 $L$ 经过 $p$,$G$ 可测得 $L$ 此处的 4-速 $U^\mu$。但有时整个 4-速场不重要,不必强调观者世界线如何,只需给定事件 $p$ 及其一个指向未来类时单位矢 $Z^\mu$ 作为 4-速。严格来说,观者除用标准钟获得时间感,也能用标准尺获得空间方向感。

非零空间矢量当然类空;反之不然,因为类空矢量是绝对的,但空间矢量是相对观者而言的。总之,我们不仅考虑 $p$ 处 $e_0^\mu=Z^\mu$,还配备另外 3 个按正交右手归一的空间矢量 $e_i^\mu$,共同构成切空间的一组基,用于物理量的观测。$(p,e_I^\mu)$ 称为\textbf{瞬时观者}(instantaneous observer),简记 $(p,Z^\mu)$。

正交归一的基底还称\textbf{标架}\footnote{3-标架的英语是 vielbein。4-标架的英语是 vierbein 或 tetrad。德语里 viel 指多而 vier 指四。}。称 $(p,U^\mu)$ 为 $L$ 在 $p$ 的\textbf{瞬时共动观者}。



设研究对象 $L(\tau)$ 同观者 $G$ 交于 $p=L(\tau_1)$。若二者都做惯性运动,取 $G$ 所在惯性系,则 $U^0=\gamma,U^i=\gamma u^i$。即使 $G$ 任意,由于 $G$ 仍可直接测量时间长度、空间长度,故 $L$ 相对于 $G$ 的 3-速仍能表为 $u^i=\dv*{x^i}{t}={U^i}/{U^0}$。
注意标架已正交归一,任意度规在该系表为 $\eta_{\mu\nu}$,则 $-U^\mu Z_\mu=U^0=\gamma$,得
    \eq{u^i=-\frac{U^i+U^\mu Z_\mu Z^i}{U^\mu Z_\mu}=-\frac{U^i}{U^\mu Z_\mu}-Z^i.}
此式是广义协变的,可见给定 $Z^\mu,U^\nu$ 就能得到 $L$ 对瞬时观者 $G$ 的 3-速。$\gamma=1/{\sqrt{1-u^2}}$ 仍成立。
与 3-速不同,只用一个瞬时观者不足以测量 3-加速,因为要确定 3-加速需要获知另一邻近点的 3-速,这不可直接观测\footnote{除非某邻域内与 $G,\alpha$ 重合,这样 $G$ 必然瞬时共动。但“邻域”具有有限大小,这意味着重合是“可观的”(曲线差距至少是高二阶无穷小),因而 3-加速在邻域内为零。当然,由于“有限大小”可以任意小,我们不妨放宽“直接观测”的限制,这样便不要求严格相交(只要求曲线差距高一阶),仍可用瞬时静止观者观测非零 3-加速。但其它瞬时观者仍然测不了,因为曲线差距仍然同阶。}。通常我们还是尝试构建参考系来测量 3-加速。

\section{等效原理}\label{sec:eq-prin}



引力理论的检验标准当然是科学的基础——实验,为此自然需要一个关于引力实验本身的理论,这样就可从实验结果依据实验理论来评判引力理论。Dicke 从 1960s 开始所从事的实验研究使人们对等效原理的理解逐渐深刻,并意识到应把等效原理摆在引力理论的约束这一地位上。

Galileo 性 $m_I=m_G$ 可准确地称为\textbf{弱}(weak)\textbf{等效原理},简称 WEP。Einstein 将力学实验推至一切物理实验的等效原理则称 \textbf{Einstein 等效原理},简称 EEP。亦有学者称之\textbf{强}(strong)\textbf{等效原理},简称 SEP。
不过,Will\cite{Will18} 选择区分 EEP 和 SEP。他将 EEP 表述为:

    1. \textit{WEP 成立};

    2.\textbf{局域 Lorentz 不变性}:\textit{任何局域非引力实验的结果与观者 4-速无关};
    
    3.\textbf{局域位置不变性}:\textit{任何局域非引力实验的结果与其时空点本身无关}。

\noindent 注意局域一词仍指空间点。之后又在 SEP 的表述中强调\textbf{自引力系统}\footnote{自引力系统(self-gravitating system)指各部分由引力聚集或能产生引力场的系统,大到恒星、行星,小到日常所见的物体皆于考虑之列。}与测试粒子一样遵守 WEP,且将局域 Lorentz 和位置不变性中的非引力实验换为任意实验。可见这种要求似乎更强,它不仅考虑系统所处的外引力场,还考虑系统内物体所激发的自引力场,即引力的主被动方面皆考虑。
它们对引力理论选择的具体影响如何?目前任何引力理论都满足 WEP,因为验证很早就已开始且精度较高,这件近乎铁打的事实使人很难寻找一个不符 WEP 的理论。可论证若 EEP 成立,则引力一定是时空几何效应,严格满足一个度规理论的基本性质:

1.\textit{时空上能定义度规};

2.\textit{自由的测试粒子的世界线满足该度规下的测地线方程};

3.\textit{局域惯性系中非引力物理定律满足狭义相对论}。

\noindent 前文已从 WEP 窥探到 2 的成立。又可证\textbf{Schiff 猜想}(conjecture):凡完备且自洽的满足 WEP 的理论皆必须满足 EEP。当然,引力的度规理论是很多的,除了广义相对论,还有 Brans-Dicke 理论。

 由于目前已知的度规理论除广义相对论均不满足 SEP,故有人提出:若 SEP 成立,广义相对论可能是唯一选择。关于这一命题的讨论目前还不够严格,故只能算猜想。
目前学术界对等效原理的分类未达成共识,Ohanian\cite{Ohanian} 就批评用自引力系统强行区分 WEP 和 EEP。对此无法说得更通俗了,只有在我们学会后续知识后才能有更深理解。

上述所有论证细节和具体实验均放入

WEP 目前最精确的检验是 21 世纪于太空中完成的,得到铝、铂间 $m_G/m_I$ 的差异小于 $10^{-14}$,且数据在地球上空各处基本一致;关于 EEP,对量子非引力实验的描述需用波函数,而波函数至少散布在一个区域内,这样强引力场的潮汐效应会显现在波函数中,但可选择对弱引力场中的小区域波函数进行实验,已有数据显示这种量子系统可满足;SEP 目前最精确的检验利用了地月测距(lunar-laser-ranging),测得两天体间 $m_G/m_I$ 的差异不大于 $5.5\times 10^{-13}$,引力结合能的贡献差异不大于 $1.3\times 10^{-3}$。


\section{几何}


\section{后 Newton 近似}

\section{引力波}

\section{宇宙学}