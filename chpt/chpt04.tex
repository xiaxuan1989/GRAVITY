\chapter{实验理论}\label{chpt:experiment}

\section{等效原理}\label{sec:eq-prin}



引力理论的检验标准当然是科学的基础——实验,为此自然需要一个关于引力实验本身的理论,这样就可从实验结果依据实验理论来评判引力理论。Dicke 从 1960s 开始所从事的实验研究使人们对等效原理的理解逐渐深刻,并意识到应把等效原理摆在引力理论的约束这一地位上。

Galileo 性 $m_I=m_G$ 可准确地称为\textbf{弱}(weak)\textbf{等效原理},简称 WEP。Einstein 将力学实验推至一切物理实验的等效原理则称 \textbf{Einstein 等效原理},简称 EEP。亦有学者称之\textbf{强}(strong)\textbf{等效原理},简称 SEP。
不过,Will\cite{Will18} 选择区分 EEP 和 SEP。他将 EEP 表述为:

    1. \textit{WEP 成立};

    2.\textbf{局域 Lorentz 不变性}:\textit{任何局域非引力实验的结果与观者 4-速无关};
    
    3.\textbf{局域位置不变性}:\textit{任何局域非引力实验的结果与其时空点本身无关}。

\noindent 注意局域一词仍指空间点。之后又在 SEP 的表述中强调\textbf{自引力系统}\footnote{自引力系统(self-gravitating system)指各部分由引力聚集或能产生引力场的系统,大到恒星、行星,小到日常所见的物体皆于考虑之列。}与测试粒子一样遵守 WEP,且将局域 Lorentz 和位置不变性中的非引力实验换为任意实验。可见这种要求似乎更强,它不仅考虑系统所处的外引力场,还考虑系统内物体所激发的自引力场,即引力的主被动方面皆考虑。
它们对引力理论选择的具体影响如何?目前任何引力理论都满足 WEP,因为验证很早就已开始且精度较高,这件近乎铁打的事实使人很难寻找一个不符 WEP 的理论。可论证若 EEP 成立,则引力一定是时空几何效应,严格满足一个度规理论的基本性质:

1.\textit{时空上能定义度规};

2.\textit{自由的测试粒子的世界线满足该度规下的测地线方程};

3.\textit{局域惯性系中非引力物理定律满足狭义相对论}。

\noindent 前文已从 WEP 窥探到 2 的成立。又可证\textbf{Schiff 猜想}(conjecture):凡完备且自洽的满足 WEP 的理论皆必须满足 EEP。当然,引力的度规理论是很多的,除了广义相对论,还有 Brans-Dicke 理论。

 由于目前已知的度规理论除广义相对论均不满足 SEP,故有人提出:若 SEP 成立,广义相对论可能是唯一选择。关于这一命题的讨论目前还不够严格,故只能算猜想。
目前学术界对等效原理的分类未达成共识,Ohanian\cite{Ohanian} 就批评用自引力系统强行区分 WEP 和 EEP。对此无法说得更通俗了,只有在我们学会后续知识后才能有更深理解。

上述所有论证细节和具体实验均放入

WEP 目前最精确的检验是 21 世纪于太空中完成的,得到铝、铂间 $m_G/m_I$ 的差异小于 $10^{-14}$,且数据在地球上空各处基本一致;关于 EEP,对量子非引力实验的描述需用波函数,而波函数至少散布在一个区域内,这样强引力场的潮汐效应会显现在波函数中,但可选择对弱引力场中的小区域波函数进行实验,已有数据显示这种量子系统可满足;SEP 目前最精确的检验利用了地月测距(lunar-laser-ranging),测得两天体间 $m_G/m_I$ 的差异不大于 $5.5\times 10^{-13}$,引力结合能的贡献差异不大于 $1.3\times 10^{-3}$。


\section{几何}


\section{后 Newton 近似}

\section{引力波}

\section{宇宙学}