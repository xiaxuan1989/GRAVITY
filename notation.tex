\chapter*{记号及单位制}
\addcontentsline{toc}{chapter}{\nameref{notation}} 

用分量指代物理量本身,书写将大大简化,且往往足以体现物理性质。比如,用 $E^i$ 可代表某个参考系所测电场 $\bm E=(E^1,E^2,E^3)=(E_x,E_y,E_z)$,用 $x^\mu$ 可代表四个时空坐标 $x^0=ct,x^1=x,x^2=y,x^3=z$。右上角的数字只是\textbf{指标}(index)而非乘幂。不必担心,很多情况下指标不会与乘幂相混淆。实在冲突时可根据上下文判断或额外注释。规定,时空指标一般取 $\mu,\nu$ 等希腊字母,遍及 $0,1,2,3$;空间指标取 $i,j$ 等拉丁字母,遍及 $1,2,3$。现代物理学总和指标打交道,初学者可能会感到复杂但又需要习惯。

物理学的代数运算常涉及对某指标的求和。注意到,单个因式中指标一般至多出现两次:出现两次的指标是求和哑元,称为\textbf{哑}(dummy)\textbf{标};留下的指标只出现一次,称\textbf{自由}(free)\textbf{指标}。比如,$4$ 维列向量 $X$ 的仿射变换可用方阵 $A$ 和常列向量 $Y$ 表为 $X'=AX+X_0$,分量表达为
\[X'^\mu=\sum_{\nu=0}^3 A^{\mu}{}_{\nu}X^\nu+ Y^\mu,\]
其中为了区分行列指标而将矩阵元的指标错开。在因式 $A^{\mu}{}_{\nu}X^\nu$ 中,$\mu$ 是自由指标,$\nu$ 是哑标。不与自由指标冲突的前提下,哑标可任意替换,比如换为 $A^{\mu}{}_{\sigma}X^\sigma$。多个因式之间可以出现重复哑标,因为它们是加法关系,比如用 $(A+B)$ 描述的线性变换:
\[X'^\mu=\sum_{\nu=0}^3 (A^{\mu}{}_{\nu}+B^{\mu}{}_{\nu})X^\nu=\sum_{\nu=0}^3 A^{\mu}{}_{\nu}X^\nu+\sum_{\nu=0}^3 B^{\mu}{}_{\nu}X^\nu.\]
$A^{\mu}{}_{\nu}X^\nu,B^{\mu}{}_{\nu}X^\nu$ 都出现 $\nu$。替换哑元后也可不重复,写作 $A^{\mu}{}_{\nu}X^\nu,B^{\mu}{}_{\sigma}X^\sigma$。单个因式可以出现多对哑标,比如,用 $\delta_{ij}$ 表示单位阵元,则 Descartes 系中线元的勾股定理是
\[\d\ell^2=\d x^2+\d y^2+\d z^2=\sum_{i,j=1}^3\delta_{ij}\d x^i\d x^j=\sum_{i=1}^3\sum_{j=1}^3\delta_{ij}\d x^i\d x^j,\]
当然,$\d x^2$ 的“2”表示平方而非指标。由乘法的交换、结合及分配律,求和顺序是无所谓的。
可以看到,\textbf{上标}(upper/raised indices)和\textbf{下标}(lower indices)位置总保持一致,这称为\textbf{指标平衡}。
如上性质说明,可省略求和号而只关心通项,毫不影响代数计算。由此 Einstein 提出\textbf{求和约定}:若某个指标一上一下成对出现,则要对该指标求和。也称对该指标\textbf{缩并}(contraction)。比如 $\d\ell^2=\delta_{ij}\d x^i\d x^j,X'^\mu=A^{\mu}{}_{\nu}X^\nu+ Y^\mu$。

物理学还涉及大量的坐标变换。将 4 维坐标系 $\{x\},\{x'\}$ 之间的光滑坐标变换记作
$x'^\mu=h^\mu(x^0,x^1,x^2,x^3)$,
通常混用符号地简写为 $x'^\mu=x'^\mu(x)$,链式法则给出
\[
    \d x'^\mu=\pdv{x'^\mu}{x^\nu}\d x^\nu,\quad \pdv{x'^\mu} =\pdv{x^\nu}{x'^\mu} \pdv{x^\nu}. 
\]
其中分母的上标看作分子的下标,$({\partial x'^\mu}/{\partial x^\nu})$ 就是分析学里的 \textbf{Jacobi 矩阵}。函数 $f$ 的全微分为
\[
    \d f=\pdv{f}{x^\mu}\d x^\mu=\del_\mu f\d x^\mu,
\]
因此梯度 $\del_\mu f$ 的变换是
\[
    \del'_\mu f=\pdv{f}{x'^\mu}=\pdv{x^\nu}{x'^\mu}\pdv{f}{x^\nu}=\pdv{x^\nu}{x'^\mu}\del_\nu f.
\]
注意对于 Jacobi 矩阵元,一般不用简记符 $\del_\mu$。

Jacobi 矩阵的互逆性:
\[\pdv{x^\mu}{\xi^\alpha}\pdv{\xi^\alpha}{x^\nu}=\pdv{\xi^\mu}{x^\kappa} \pdv{x^\kappa}{\xi^\nu}=\delta^\mu_\nu,\]
其中 $\delta^i_j$ 也是单位阵元。

部分量子场论教材选择统一使用下指标而只强调求和约定“总共重复两次”,更有甚者不区分指标顺序。这在讨论更深刻问题时出现缺陷,建议只在特殊情况采取简化。比如 $\delta^\mu_\nu$ 是考虑到 $I$ 是对称矩阵。

物理学是关乎实验测量的学科,讨论物理量的单位是极为重要的。在\textbf{国际制}(SI)中,我们认为时间、空间乃不同物理量,自然不会谈及“1s 等于多少 m”这种看似奇怪的问题。但在相对论中,这个问题很有意义,因为时空乃统一的概念。考虑令 $c=1$ 使 $x^0=t$,国际制下 $c$ 近似为 $3\times 10^8$ m/s,故这件事相当于
\[
1\,\mathrm s=3\times 10^8\,\mathrm m.
\]
这不会与国际制有任何冲突,因为在国际制中不会谈及“1s 等于多少 m”这种问题。这是国际制中一个可以利用的自由度。引力理论常涉及引力常量,把它们的数值都取为 1,即
\[
c=G=1,
\]
可简化大量公式的书写,这就是\textbf{几何单位制}(system of geometrized units),简称\textbf{几何制}(SG)。在国际制中 $G$ 近似等于 $6.67\times 10^{-11}\,\mathrm{m}^3/(\mathrm{s}^2\cdot\mathrm{kg})$,因此该操作相当于
\[
1\,\mathrm{kg}=\frac{6.67\times 10^{-11}}{(3\times 10^8)^2} \,\mathrm{m},
\]
可见这利用的是国际制的另一自由度。几何制便统一了长度、质量和时间的量纲。不涉及引力的量子理论经常使用如下的\textbf{自然单位制}(system of natural units),简称\textbf{自然制}(SN):
\[
c=\hbar=1,
\]
这里 $\hbar$ 在国际制中近似等于 $6.58\times 10^{-22}\,\mathrm{MeV}\cdot\mathrm s$。然而在计算物理量的数值时几何制就不再方便,因此我们要研究物理量、物理公式如何从几何制转换到其它单位制。一般来讲,所研究物理量在当前单位制是带上量纲的。比如,几何制下粒子物理常用 eV 作为质量单位,因为质能方程在几何制下统一了质量和能量量纲。欲将质量单位从几何制还原为国际制,无非是在问 eV 等于多少 kg。首先都转化到常用单位来,即
\[
1\,\mathrm{eV}=1.6 \times 10^{-19}\,\mathrm{C}\cdot\mathrm{V}=1.6\times 10^{-19}\,\mathrm{J}=1.6\times 10^{-19}\,\mathrm{kg}\cdot\frac{\mathrm{m}^2}{\mathrm{s}^2},
\]
代入几何制定义有
\[
1\,\mathrm{eV}=\frac{1.6\times 10^{-19}}{(3\times 10^8)^2}\,\mathrm{kg}.
\]
若所研究物理量在当前单位制只是一个无量纲数,则有可能它在其它单位制中有无量纲。比如定义直接告诉我们 $c$ 在几何制中无量纲。“真正无量纲”的量在任何单位制下都无量纲,因此不受单位转换的影响,如精细结构常数
\[
\alpha=\frac{e^2}{4\pi\epsilon_0\hbar c}=\frac{1}{137}.
\]
代入自然制定义得 $e^2/{4\pi\epsilon_0}=1/137$,这说明兼容自然制和精细结构常数的同时仍有自由度可利用。令
\[
c=\hbar=\epsilon_0=1,
\]
因而 $e=\sqrt{4\pi/137}$。由 $c=1/\sqrt{\epsilon_0\mu_0}$ 得 $\mu_0=1$。如上规定就称为 \textbf{Heaviside-Lorentz 单位制},简称 \textbf{HL 制}(SHL),其下不用再写出 Maxwell 方程中的常数。几何制和自然制还可共同构成 \textbf{Planck 制}(SP),这在量子引力理论中用得多。广义相对论常一齐使用几何制和 \textbf{Gauss 单位制},以便于引力和电磁学理论的书写:
\[
c=G=4\pi\epsilon_0=1,
\]
统称\textbf{几何 Gauss 制}。其中 $\epsilon_{0}$ 是真空介电常数(vacuum permittivity)。
进而根据光速 $c=1/\sqrt{\mu_0\epsilon_0}$ 可知真空磁导率(vacuum permeability)为 $\mu_{0}=4\pi$。 

当研究物理公式在不同单位制的转换时,最便捷的做法不是如上这样分析单位,而是分析量纲,即根据量纲补充缺失的常数。比如,在几何制下的 Schwarzschild 度规为
\[
    \d s^2=-\left(1-\frac{2M}{r}\right)\d t^2+\left(1-\frac{2M}{r}\right)^{-1}\d r^2+r^2\d\Omega^2,
\]
欲转换为国际制。注意几何制的定义是 $c=G=1$,使时间 $[T]$、长度 $[L]$、质量 $[M]$ 为相同量纲,这也是与国际制仅有的差距。我们只需在国际制下分析量纲,然后补上 $c,G$ 即可。$c$ 的国际制量纲为 $[T]^{-1}[L]$。$G$ 的国际制量纲为 $[T]^{-2}[L]^3[M]^{-1}$。研究系数 $g_{11}$,它在国际制无量纲。$\frac{M}{r}G^{\alpha}c^{\beta}$ 的量纲为 $[T]^{-2\alpha-\beta}[L]^{3\alpha+\beta-1}[M]^{1-\alpha}$。因此
\[
    -2\alpha-\beta=3\alpha+\beta-1=1-\alpha=0\implies \alpha=1,\beta=-2.
\]
$g_{00}$ 同理,故可得
\[
    \d s^2=-\left(1-\frac{2GM}{c^2 r}\right)c^2\d t^2+\left(1-\frac{2GM}{c^2 r}\right)^{-1}\d r^2+r^2\d\Omega^2.
\]
用相同方法可得到
\[
    R_{\mu\nu}-\frac 12 g_{\mu\nu} R=\frac{8\pi G}{c^4} T_{\mu\nu},\quad U^\mu U_\mu = - c^2,\quad \curl\bm B=\mu_0\epsilon_0\del_t\bm E+\mu_0\bm j, \quad \cdots
\]
方法皆完全相同。其中注意 $[\epsilon_0]=[T]^4[L]^{-3}[M]^{-1}[I]^2$($[I]$ 为电流量纲),而从几何 Gauss 制到国际制的过程是补充 $c,G,4\pi\epsilon_0$。再通常将 $c$ 去掉而引入 $\mu_0=1/c^2\epsilon_0$。

另外,有必要说明这样一种思想:有别于传统认知,物理公式应当视作“数”的公式而非“量”的公式。
简单来说,正如定义 $x^0=ct$ 一样,定义新单位制下某个物理量为旧单位制中的对应物理量乘以相应常数,这样物理方程中就再也不会出现该常数,因为它全部被收进了新定义里,比如,将 $G$ 收进场源质量里,并不影响其它任何物理量的计算;重新定义 $t'=ct$,其单位同距离单位彻底一致,时间和距离“共享”同一单位,因而 $c=1$ 是真正的“数”。不过从单位转换的角度看,这些数据并不需要收进任何一个物理量中,而是直接收进单位的相对关系里。
\label{notation}