%%%%%%%%%%%%%%%%%%%%%%%%%%%%%%%%%%%%%%%%%%%%%%%%%%%%%%%%%%
% 页面设置
%%%%%%%%%%%%%%%%%%%%%%%%%%%%%%%%%%%%%%%%%%%%%%%%%%%%%%%%%%
\usepackage[
    b5paper,
    bindingoffset=.425in,
    left=.5in,
    right=.5in,
    top=.8125in,
    bottom=.9375in,
]{geometry}
\usepackage{wrapfig}                % 环绕图形
\usepackage[toc]{multitoc}          % 多级目录
\usepackage{mdwlist}                % 紧凑列表
\usepackage{subfig}                 % 子图
\usepackage{perpage}                % 每页编号
\MakePerPage{footnote}              % 脚注


%%%%%%%%%%%%%%%%%%%%%%%%%%%%%%%%%%%%%%%%%%%%%%%%%%%%%%%%%%
% 字体设置
%%%%%%%%%%%%%%%%%%%%%%%%%%%%%%%%%%%%%%%%%%%%%%%%%%%%%%%%%%
\usepackage{
    amsmath, % 使用 equations
    amsthm,
    amsfonts,
    amssymb,
    slashed,
    url,
    graphicx
}

% 英文字体
\usepackage[no-math]{fontspec}
\setmainfont{TeXGyreTermesX}[
    UprightFont = *-Regular ,
    BoldFont = *-Bold ,
    ItalicFont = *-Italic ,
    BoldItalicFont = *-BoldItalic ,
    Extension = .otf ,
    Scale = 1.0]
\setsansfont{texgyreheros}[
    UprightFont = *-regular ,
    BoldFont = *-bold ,
    ItalicFont = *-italic ,
    BoldItalicFont = *-bolditalic ,
    Extension = .otf ,
    Scale = 0.9]

% 中文字体
\usepackage{ctex}
\xeCJKsetup{AutoFakeBold=true}      % 自动伪粗体
\setCJKmainfont[
    BoldFont = FandolSong-Bold,     % 粗体
    ItalicFont = FandolKai-Regular, % 斜体
    Mapping = fullwidth-stop        % 全角句点
]{FandolSong-Regular}

% 数学字体
\usepackage{mathrsfs}               % 花体
\usepackage{anyfontsize}            % 允许任意字体大小
% 保存原字体设置
\let\oldencodingdefault\encodingdefault
\let\oldrmdefault\rmdefault
\let\oldsfdefault\sfdefault
\let\oldttdefault\ttdefault
% 切换到 T1 编码并指定 newtx 的正文字体系列
\def\encodingdefault{T1}
\renewcommand{\rmdefault}{ntxtlf}
\renewcommand{\sfdefault}{qhv}
\renewcommand{\ttdefault}{ntxtt}
\usepackage{newtxmath}              % newtx
% 恢复原字体设置
\let\encodingdefault\oldencodingdefault
\let\rmdefault\oldrmdefault
\let\sfdefault\oldsfdefault
\let\ttdefault\oldttdefault
\let\Bbbk\relax                     % 清除可能冲突的 \Bbbk
\usepackage{esint}                  % 积分符号
% 重定义大型 ∑ 和 ∏ 运算符
\DeclareSymbolFont{CMlargesymbols}{OMX}{cmex}{m}{n}
\let\sumop\relax \let\prodop\relax
\DeclareMathSymbol{\sumop}{\mathop}{CMlargesymbols}{"50}
\DeclareMathSymbol{\prodop}{\mathop}{CMlargesymbols}{"51}


%%%%%%%%%%%%%%%%%%%%%%%%%%%%%%%%%%%%%%%%%%%%%%%%%%%%%%%%%%
% 引用设置
%%%%%%%%%%%%%%%%%%%%%%%%%%%%%%%%%%%%%%%%%%%%%%%%%%%%%%%%%%
\usepackage[
    backend=biber,      % biber 作为后端,支持 UTF-8、复杂引用等
    sorting=nyt,        % 按照名字、年份、标题排序
    hyperref=auto,      % 自动集成 hyperref 包,处理链接跳转
    backref=true,       % 标注引用它的页面
    backrefstyle=three, % 显示为 “2, 5, 8”
]{biblatex}             % 也可用 style=authoryear 等
\addbibresource{refs.bib} % 参考文献
\usepackage[
    hidelinks,          % 隐藏颜色
    bookmarksnumbered,  % PDF 书签显示章节编号
    hyperindex          % 索引也带有超链接
    %linktocpage, % 目录链接到页码
]{hyperref} %超链接
\usepackage{bookmark} % 处理超链接的书签

%\usepackage{imakeidx}                   % 索引
%\makeindex[options= -s index.ist]       % 初始化索引,以便使用 \index 添加条目
%\newcommand{\idx}[2]{{\index{#2@#1}}}   % 中文注音


%=================== 颜色相关 ===================
\usepackage[svgnames,dvipsnames]{xcolor}  % 丰富的颜色名支持,自定义颜色

%=================== 数学相关 ===================
%\usepackage{isomath}                       % ISO 规范
\usepackage{physics}                       % 物理
\usepackage{mathtools,nccmath}             % amsmath的扩展和改进,增强数学排版能力

%=================== 图形与绘图 ===================
\usepackage{tikz}
\usetikzlibrary{backgrounds}              % TikZ背景库
\usetikzlibrary{arrows,shapes,positioning,shadows,trees,mindmap} % 各类常用TikZ库
\usetikzlibrary{tikzmark}                  % 用于标记TikZ节点
\usetikzlibrary{calc,math}                 % 计算与数学辅助库
\usetikzlibrary{decorations.markings}     % 绘图装饰(箭头等)
%\usetikzlibrary{3d,perspective}          % 3D视角
\usepackage{tikz-cd}                       % 绘制交换图(commutative diagrams)
\usepackage{tikz-3dplot}                   % 3D坐标绘图辅助
\usepackage{pgfplots}                      % 基于TikZ的绘图,支持函数绘制
\pgfplotsset{compat = newest}              % 兼容最新版本pgfplots
%\usepgfplotslibrary{colormaps}           % 颜色映射库
\usepgflibrary{shapes.geometric}           % 几何形状库

\usepackage[edges]{forest}                  % 画树结构(改进版TikZ树)
\usetikzlibrary{arrows.meta}                % TikZ箭头元库
\colorlet{linecol}{black!75}                 % 定义颜色变量(线颜色)
\usepackage{xkcdcolors}                      % XKCD配色方案

\usetikzlibrary{patterns}                    % 纹理和图案库
\tikzset{>={Stealth[inset=0pt,angle=20:10pt]}} % 设置箭头样式

\usepackage[all]{xy}                         % xy-pic,用于交换图和复杂图形绘制
\usepackage[inline]{asymptote}               % Asymptote 3D绘图,支持内联绘制

\usepackage{wallpaper}                       % 背景图片设置

%=================== 版面与排版 ===================
\usepackage{fix-cm}                         % 允许任意缩放计算机现代字体大小
\usepackage{textpos}                        % 绝对定位文本框,方便布局
\usepackage{appendix}                       % 附录管理
\usepackage{caption}                        % 自定义图表标题样式
%\usepackage{setspace}                      % 行间距调整

%=================== 表格排版 ===================
\usepackage{array}                          % 增强的表格列类型支持
\usepackage{ragged2e}                       % 允许表格内文字左右对齐的命令
\newcolumntype{P}[1]{>{\RaggedRight\hspace{0pt}}p{#1}} % 左对齐可换行列类型P
\newcolumntype{X}[1]{>{\RaggedRight\hspace*{0pt}}p{#1}}% 左对齐可换行列类型X

%=================== 其他 ===================
\usepackage{comment}                        % 多行注释环境
\usepackage{xspace}                         % 自定义命令后智能处理空格
\usepackage{diagbox}                        % 表格中斜线分割单元格
\usepackage{tcolorbox}                      % 彩色框,常用于定理、示例、注释框等

%%%%%%%%%%%%%%%%%%%%%%%%%%%%%%%%%%%%%%%%%%%%%%%%%%%%%%%%%%
% 自定义 macros
%%%%%%%%%%%%%%%%%%%%%%%%%%%%%%%%%%%%%%%%%%%%%%%%%%%%%%%%%%
\definecolor{plop}{HTML}{4D7186}    % 石板蓝 + 青灰色
\definecolor{cds}{HTML}{5F7C7D}     % 浅军绿色
\definecolor{rice}{HTML}{FFFBE8}    % 米色

%================== 元信息 ==================
\title{引力}                                  % 书名
\newcommand{\booksubtitle}{GRAVITY}          % 副标题
\newcommand{\booklicense}{CC0 4.0 License}  % 版权信息
\author{夏草}                                % 作者
\newcommand{\authorsubtitle}{California, USA}% 作者附加信息

\makeatletter
\newcommand{\booktitle}{\@title}             % 书名快捷命令
\newcommand{\bookauthor}{\@author}           % 作者快捷命令
\makeatother


%================== 名称重定义 ==================
\renewcommand{\contentsname}{目录}
\renewcommand{\proofname}{证}
%================== 编号规则 ====================
\numberwithin{equation}{section} % 公式按章节
\numberwithin{figure}{chapter}  % 图序按章节
%================== 定理环境 ====================
\newtheorem{eg}{\indent 例}[chapter]
\newtheorem{theorem}{定理}[section]
\newtheorem{definition}{定义}[chapter]
\newtheorem*{remark}{注}
%================== 环境定义 ==================
\newenvironment{solution}{\begin{proof}[解]}{\end{proof}}


%================== 数学符号 =================
\newcommand{\pr}{\mathrm{Pr}}
\newcommand{\Tset}{\mathcal{T}}

\newcommand{\cate}[1]{\ensuremath{\mathsf{#1}}}	% Font series for categories

\def\bm{\vb*}
\def\d{\dd}
\def\oo{\,\vdot\,}

\def\D{\mathsf{D}}
\def\c{\mathrm{C}}
\def\C{\mathbb{C}}
\def\a{\mathrm{A}}
\def\R{\mathbb{R}}
\def\N{\mathbb{N}}
\def\Z{\mathbb{Z}}
\def\Q{\mathbb{Q}}
\def\l{\ell}
\def\eqto{\Leftrightarrow}
\def\mt{\varnothing}
\def\e{\mathrm{e}}
\def\i{\mathrm{i}}
\def\pl{\mathrel{/\mskip-4mu/}}

\def\sgn{\operatorname{sgn}}
\def\diag{\operatorname{diag}}
\def\S{\mathrm{S}}
\def\A{\mathrm{A}}
\def\alt{\operatorname{alt}}
\def\H{\mathcal{H}}
\def\I{\mathscr{I}}

\def\del{\partial}


%%%%%%%%%%%%%%%%%%%%%%%%%%%%%%%%%%%%%%%%%%%%%%%%%%%%%%%%%%
% tikz 公式标注
%%%%%%%%%%%%%%%%%%%%%%%%%%%%%%%%%%%%%%%%%%%%%%%%%%%%%%%%%%
% Commands for Highlighting text -- non tikz method
\newcommand{\highlight}[2]{\colorbox{#1!17}{$\displaystyle #2$}}
%\newcommand{\highlight}[2]{\colorbox{#1!17}{$#2$}}
\newcommand{\highlightdark}[2]{\colorbox{#1!47}{$\displaystyle #2$}}

% Commands for Highlighting text -- non tikz method
\renewcommand{\highlight}[2]{\colorbox{#1!17}{#2}}
\renewcommand{\highlightdark}[2]{\colorbox{#1!47}{#2}}

\newcommand{\hl}[3][a]{\tikzmarknode{#1}{\highlight{#2}{$\displaystyle #3$}}}

\newcommand{\hn}[5][a]{
\begin{tikzpicture}[overlay,remember picture,>=stealth,nodes={align=left,inner ysep=1pt},<-]
    \ifthenelse{#2>0}{\def\x{north}}{\def\x{south}}
    \ifthenelse{#3>0}{
    \def\y{west}
    \def\z{east}}{
    \def\y{east}
    \def\z{west}}
    
    \path (#1.\x) ++ (0,#2em) node[anchor=south \y,color=#4!67] (a#1){\textit{#5}};
    \draw [color=#4!57](#1.\x) |- ([xshift=-0.3ex,color=#4]a#1.south \z);
\end{tikzpicture}
}

\newcommand{\hc}[5][a]{
\begin{tikzpicture}[overlay,remember picture,>=stealth,nodes={align=left,inner ysep=1pt},<-]
    \ifthenelse{#2>0}{\def\x{north}}{\def\x{south}}
    \ifthenelse{#3>0}{
    \def\y{west}
    \def\z{east}}{
    \def\y{east}
    \def\z{west}}
    \path (#1.\x) ++ (#3em,#2em) node[anchor=south \y,color=#4!67] (a#1){\textit{#5}};
    \draw [color=#4!57](#1.\x) |- ([xshift=-0.3ex,color=#4]a#1.south \z);
\end{tikzpicture}
}