\documentclass[border=3pt,tikz]{standalone}
\usepackage{ctex}
\usepackage[utf8]{inputenc}
\usepackage{amsmath} % for \text
\usepackage{etoolbox} % ifthen
\usepackage[outline]{contour} % glow around text
\usetikzlibrary{calc} % for adding up coordinates
\usetikzlibrary{decorations.markings,decorations.pathmorphing}
\usetikzlibrary{angles,quotes} % for pic (angle labels)
\usetikzlibrary{arrows.meta} % for arrow size
\usepackage{xfp} % higher precision (16 digits?)
\contourlength{1.1pt}

\tikzset{>=latex} % for LaTeX arrow head
\colorlet{myred}{red!85!black}
\colorlet{myblue}{blue!80!black}
\colorlet{mygreen}{green!80!black}
\colorlet{mydarkred}{red!55!black}
\colorlet{mylightred}{red!85!black!12}
\colorlet{myfieldred}{mydarkred!5} % for S' background
\colorlet{mydarkblue}{blue!50!black}
\colorlet{mylightblue}{blue!50!black!30}
\colorlet{mylightblue2}{myblue!10}
\colorlet{mypurple}{blue!40!red!80!black}
\colorlet{mydarkgreen}{green!50!black}
\colorlet{mydarkpurple}{blue!40!red!50!black}
\colorlet{myorange}{orange!40!yellow!95!black}
\colorlet{mydarkorange}{orange!40!yellow!85!black}
\colorlet{mybrown}{brown!20!orange!90!black}
\colorlet{mydarkbrown}{brown!20!orange!55!black}
\tikzstyle{world line}=[myblue!40,line width=0.3]
\tikzstyle{world line t}=[mypurple!50!myblue!40,line width=0.3]
\tikzstyle{world line'}=[mydarkred!40,line width=0.3]
\tikzstyle{mysmallarr}=[-{Latex[length=3,width=2]},thin]
\tikzstyle{mydashed}=[dash pattern=on 3 off 3]
\tikzstyle{rod}=[mydarkbrown,draw=mydarkbrown,double=mybrown,double distance=2pt,
                 line width=0.2,line cap=round,shorten >=1pt,shorten <=1pt]
\tikzstyle{vector}=[->,line width=1,line cap=round]
\tikzstyle{vector'}=[vector,shorten >=1.2]
\tikzstyle{particle}=[mygreen,line width=0.9]
\tikzstyle{photon}=[-{Latex[length=5,width=4]},myorange,line width=0.8,decorate,
                    decoration={snake,amplitude=1.0,segment length=5,post length=5}]

\def\tick#1#2{\draw[thick] (#1) ++ (#2:0.06) --++ (#2-180:0.12)}
\def\tickp#1#2{\draw[thick,mydarkred] (#1) ++ (#2:0.06) --++ (#2-180:0.12)}
\def\Nsamples{100} % number samples in plot

\begin{document}
\begin{tikzpicture}[scale=1.8]
  \message{Invariant hyperboloids^^J}
  
  % SETTINGS
  \def\xmin{2.2}
  \def\xmax{3.1}
  \def\ymin{2.2}
  \def\ymax{2.6}
  \def\xmaxp{2.85} % maximum of rotated axis
  \def\Nlines{4} % number of world lines (at constant x/t)
  \pgfmathsetmacro\ang{atan(0.52)} % angle between x and x' axes
  \pgfmathsetmacro\d{0.64*\xmax/\Nlines} % grid size
  \pgfmathsetmacro\D{\d/cos(\ang)/sqrt(1-tan(\ang)^2)} % grid size, boosted
  \pgfmathsetmacro\dextra{(\Nlines+1)*\d} % extra line
  \pgfmathsetmacro\st{3*\d} % spacetime interval
  \pgfmathsetmacro\sx{4*\d} % spacetime interval
  \pgfmathsetmacro\sr{sqrt(\sx^2-\st^2)} % spacetime interval sr^2 = st^2 - sx^2 < 0
  \pgfmathsetmacro\Ax{3*\D*sin(\ang)} % x coordinate of event A
  \pgfmathsetmacro\Ay{3*\D*cos(\ang)} % y coordinate of event A
  \pgfmathsetmacro\Bx{4*\D*cos(\ang)} % x coordinate of event B
  \pgfmathsetmacro\By{4*\D*sin(\ang)} % y coordinate of event B
  \pgfmathsetmacro\Cx{\Ay+\By} % x coordinate of event C' %(\Bx+tan(\ang)*\Ay)/sqrt(1-tan(\ang)^2)
  \coordinate (O)  at (0,0);
  \coordinate (X)  at (\xmax+0.2,0);
  \coordinate (T)  at (0,\ymax+0.2);
  \coordinate (X') at (\ang:\xmaxp+0.2);
  \coordinate (T') at (90-\ang:\xmaxp+0.2);
  \coordinate (A)  at (0,\st);        % event A
  \coordinate (A') at (90-\ang:3*\D); % event A', boosted A
  \coordinate (B)  at (\sx,0);        % event A
  \coordinate (B') at (\ang:4*\D);    % event A', boosted A
  \coordinate (C)  at (4*\d,3*\d);    % event C
  
  % WORLD LINE GRID
  \message{  Making world lines...^^J}
  \foreach \i [evaluate={\x=\i*\d;}] in {1,...,\Nlines}{
    \message{  Running i/N=\i/\Nlines, x=\x...^^J}
    \draw[world line]   (-\x,-\ymin) -- (-\x,\ymax);
    \draw[world line]   ( \x,-\ymin) -- ( \x,\ymax);
    \draw[world line t] (-\xmin,-\x) -- (\xmax,-\x);
    \draw[world line t] (-\xmin, \x) -- (\xmax, \x);
  }
  \draw[world line]    (\dextra,-\ymin)  -- (\dextra,\ymax);
  \draw[world line]    (\dextra+\d,-\ymin) -- (\dextra+\d,\ymax);
  %\draw[world line'] (-\xmin,-\dextra) -- (\xmax,-\dextra);
  \draw[world line'] (-\xmin,\dextra) -- (\xmax,\dextra);
  
  % BOOSTED WORLD LINE GRID
  \message{  Making world lines for boosted frame...^^J}
  \fill[mydarkred,opacity=0.05]
    (O) --++ (\ang:\xmaxp) --++ (90-\ang:\xmaxp) --++ (\ang:-\xmaxp) -- cycle;
  \fill[mydarkred,opacity=0.05]
    (O) --++ (\ang:\D-\xmaxp) --++ (90-\ang:\D-\xmaxp) --++ (\ang:\xmaxp-\D) -- cycle;
  \foreach \i [evaluate={\x=\i*\D;}] in {1,...,\Nlines}{
    \message{  Running i/N=\i/\Nlines, x=\x...^^J};
    \ifnumcomp{\i}{<}{\Nlines}{
      \draw[world line'] (\ang:-\x) --++ (90-\ang:\D-\xmaxp);
      \draw[world line'] (90-\ang:-\x) --++ (\ang:\D-\xmaxp);
    }{}
    \draw[world line'] (\ang:\x) --++ (90-\ang:\xmaxp);
    \draw[world line'] (90-\ang:\x) --++ (\ang:\xmaxp);
  }
  
  % AXES
  \draw[->,thick] (0,-\ymin) -- (T) node[left=-1] {$t$};
  \draw[->,thick] (-\xmin,0) -- (X) node[below=0] {$x$};
  \draw[->,thick,mydarkred] (90-\ang:\D-\xmaxp) -- (T')
    node[right=2,above=-1] {$t'$};
  \draw[->,thick,mydarkred] (\ang:\D-\xmaxp) -- (X') node[below=2,right=-3] {$x'$};
  
  % LIGHTCONE
  \draw[myorange,thick]
    (-1.1*\xmin,1.1*\xmin) -- (1.1*\ymin,-1.1*\ymin)
    (-\xmaxp,-\xmaxp) -- (1.2*\xmaxp,1.2*\xmaxp);
  
  % AREA HYPERBOLIC SECTORS
  \fill[mygreen!90!black,opacity=0.15,thick,samples=\Nsamples,smooth,variable=\x,domain=0:\Ax]
    plot(\x,{sqrt((\st)^2+(\x)^2)}) -- (O) -- (A);
  \fill[mydarkblue!90!black,opacity=0.15,thick,samples=\Nsamples,smooth,variable=\y,domain=0:\By]
    plot({sqrt((\sx)^2+(\y)^2)},\y) -- (O) -- (B);
  \node[mydarkgreen,scale=0.92] at (89.5-\ang/2:1.97*\d)
    {\contour{mygreen!90!black!15}{$\dfrac{t_1^2\theta}{2}$}};
  \node[mydarkblue,scale=0.92] at (\ang/2:3.1*\d)
    {\contour{mydarkblue!90!black!15}{$\dfrac{x_1^2\theta}{2}$}};
  
  % SPACELIKE HYPERBOLOIDS
  \draw[mygreen,thick,samples=\Nsamples,smooth,variable=\x,domain=-\xmin:1.05*\xmax]
    plot(\x,{sqrt((\st)^2+(\x)^2)});
  \draw[mydarkgreen,very thick,samples=\Nsamples,variable=\x,domain=0:\Ax,
        decoration={markings,mark=at position 0.58 with {\arrow{latex}}},postaction={decorate}]
    plot(\x,{sqrt((\st)^2+(\x)^2)});
  \node[mydarkgreen,right=1,above right=0] at (-\xmin,\ymax)
    {类时校准线};
  
  % TIMELIKE HYPERBOLOIDS
  \draw[myred,very thick,samples=\Nsamples,variable=\y,domain=\st:\Cx,
        decoration={markings,mark=at position 0.58 with {\arrow{latex}}},postaction={decorate}]
    plot({sqrt(\sr^2+(\y)^2)},\y);
  \draw[myblue,thick,samples=\Nsamples,smooth,variable=\y,domain=-1.05*\ymin:0.95*\ymax]
    plot({sqrt(\sx^2+(\y)^2)},\y);
  \draw[mydarkblue,very thick,samples=\Nsamples,variable=\y,domain=0:\By,
        decoration={markings,mark=at position 0.58 with {\arrow{latex}}},postaction={decorate}]
    plot({sqrt(\sx^2+(\y)^2)},\y);
  \node[mydarkblue,right=0] at (0.7*\xmax,-0.25*\xmax)
    {\contour{white}{类空校准线}};
  
  % TICKS
  \draw[mydarkgreen,dashed] ({\Ax},0) -- (A') -- (0,{\Ay});
  \draw[mydarkblue,dashed] ({\Bx},0) -- (B') -- (0,{\By});
  \tick{0,\st}{0} node[mydarkgreen,right=4,below left=-2.5] {$t_1$};
  \tick{\sx,0}{90} node[mydarkblue,below=1,below left=-3] {$x_1$};
  \tick{0,\Ay}{0} node[mydarkgreen,above=1,left=-2]
    {\contour{white}{$t_1\cosh\theta$}};
  \tick{\Ax,0}{90} node[mydarkgreen,right=4,below=-4]
    {\contour{white}{$t_1\sinh\theta$}};
  \tick{\Bx,0}{90} node[mydarkblue,right=9,below=-4]
    {\contour{white}{$x_1\cosh\theta$}};
  \tick{0,\By}{0} node[mydarkblue,below=1,left=-2]
    {\contour{white}{$x_1\sinh\theta$}};
  
  % EVENTS
  \fill[mydarkgreen] (A)  circle(0.03); % event A
  \fill[mydarkgreen] (A') circle(0.03); % event A'
  \fill[mydarkblue]  (B)  circle(0.03); % event B
  \fill[mydarkblue]  (B') circle(0.03); % event B'
  \fill[mydarkred]   (C)  circle(0.03); % event C
  \fill[mydarkred] (\ang:4*\D)++(90-\ang:3*\D) coordinate (C') circle(0.03); % event C'
  %\node[mydarkred,above=2,right=6] at (C') {$\left\{\begin{aligned}
  %  ct' &= ct\cosh\theta -  x\sinh\theta \\
  %   x' &=  x\cosh\theta - ct\sinh\theta
  %\end{aligned}\right.$};
  
\end{tikzpicture}
\end{document}