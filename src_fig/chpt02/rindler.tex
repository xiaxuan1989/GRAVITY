
\documentclass[border=3pt,tikz]{standalone}
\usepackage{ctex}
\usepackage[utf8]{inputenc}
\usepackage{tikz}
\usepackage{amsmath} % for \text
\usepackage{mathrsfs} % for \mathscr
\usepackage{xfp} % higher precision (16 digits?)
\usepackage[outline]{contour} % glow around text
\usetikzlibrary{decorations.markings,decorations.pathmorphing}
\usetikzlibrary{arrows.meta} % for arrow size
\contourlength{1.2pt}

\newcommand{\calI}{\mathscr{I}} %\mathcal
\tikzset{>=latex} % for LaTeX arrow head
\colorlet{myred}{red!80!black}
\colorlet{myblue}{blue!80!black}
\colorlet{mygreen}{green!80!black}
\colorlet{mydarkred}{red!50!black}
\colorlet{mydarkblue}{blue!50!black}
\colorlet{mylightblue}{mydarkblue!6}
\colorlet{mypurple}{blue!40!red!80!black}
\colorlet{mydarkpurple}{blue!40!red!50!black}
\colorlet{mylightpurple}{mydarkpurple!80!red!6}
\colorlet{myorange}{orange!40!yellow!95!black}
\tikzstyle{cone}=[mydarkblue,line width=0.2,top color=blue!60!black!30,
                  bottom color=blue!60!black!50!red!30,shading angle=60,fill opacity=0.9]
\tikzstyle{cone back}=[mydarkblue,line width=0.1,dash pattern=on 1pt off 1pt]
\tikzstyle{world line}=[myblue!60,line width=0.4,shorten <=-2mm,shorten >=-2mm]
\tikzstyle{world line t}=[mypurple!60,line width=0.4,shorten <=-2mm,shorten >=-2mm]
\tikzstyle{particle}=[mygreen,line width=0.5]
\tikzstyle{photon}=[-{Latex[length=4,width=3]},myorange,line width=0.4,decorate,
                    decoration={snake,amplitude=0.9,segment length=4,post length=3.8}]
\tikzstyle{singularity}=[myred,line width=0.6,decorate,
                         decoration={zigzag,amplitude=1.8,segment length=5}]
\tikzset{declare function={% Kruskal-Szekeres coordinates
  sing(\x)        = {\fpeval{sqrt(\x*\x+1)}};%
  rstar(\c)       = {\fpeval{(\c/2-1)*exp(\c/2)}};%
  kruskalu(\x,\c) = {\fpeval{sqrt(\x*\x+(\c/2-1)*exp(\c/2))}};%
  kruskalv(\x,\c) = {\fpeval{sqrt(\x*\x-(\c/2-1)*exp(\c/2))}};%
}}
\def\tick#1#2{\draw[thick] (#1) ++ (#2:0.04) --++ (#2-180:0.08)}
\def\Nsamples{50} % number samples in plot



% LIGHTCONE
\def\R{0.10} % size lightcone
\def\e{0.08} % vertical scale
\def\ang{45} % angle light cone
\def\angb{acos(sqrt(\e)*sin(\ang))} % angle ellipse center to point of tangency
\def\a{\R*sin(\ang)*sqrt(1-\e*sin(\ang)^2)/(1-\e*sin(\ang)^2)} % vertical radius
\def\b{\R*sqrt(\e)*sin(\ang)*cos(\ang)/(1-\e*sin(\ang)^2)} % horizontal radius
\def\coneback#1{ % light cone part to be drawn behind world lines
  \draw[cone back] % dashed line back
    (#1)++(-45:\R) arc({90-\angb}:{90+\angb}:{\a} and {\b});
  \draw[cone,shading angle=-60] % top edge & inside
    (#1)++(0,{\R*cos(\ang)/(1-\e*sin(\ang)^2)}) ellipse({\a} and {\b});
}
\def\conefront#1{ % light cone part to be drawn over world lines
  \draw[cone] % light cone outside
    (#1) --++ (45:\R) arc({\angb-90}:{-90-\angb}:{\a} and {\b})
     --++ (-45:2*\R) arc({90-\angb}:{-270+\angb}:{\a} and {\b}) -- cycle;
}

\begin{document}


% KRUSKAL DIAGRAM - equidistant world lines
\begin{tikzpicture}[scale=1.8]
  \message{Kruskal diagram^^J}
  
  \def\xmax{2}
  \def\Nlines{4} % number of world lines (at constant r/t)
  \coordinate (O)  at ( 0, 0); % center I: origin (r,t) = (0,0)
  \coordinate (SW) at (-\xmax,-\xmax); % horizon r=2M, t=-infty, region I
  \coordinate (SE) at ( \xmax,-\xmax); % horizon r=2M, t=-infty, region I
  \coordinate (NW) at (-\xmax, \xmax); % horizon r=2M, t=+infty, region IV
  \coordinate (NE) at ( \xmax, \xmax); % horizon r=2M, t=+infty, region IV
  
  % REGIONS FILLS
  \fill[mylightpurple] (NW) -- (O) -- (NE) -- cycle; % region I
  \fill[mylightblue]   (NE) -- (O) -- (SE) -- cycle; % region II
  \fill[mylightpurple] (SW) -- (O) -- (SE) -- cycle; % region III
  \fill[mylightblue]   (NW) -- (O) -- (SW) -- cycle; % region IV
  
  \begin{scope}
    \clip (-\xmax,-\xmax) rectangle (\xmax,\xmax);
    
    % CONSTANT T WORLD LINES
    \message{Making world constant-r lines (region I, III)...^^J}
    \foreach \i [evaluate={\c=2*\i/\Nlines; \a=tanh(\c/2); \b=1/tanh(\c/2);
                           \y=min(\xmax,\b/sqrt(\b*\b-1)); \x=\y/\b; }] in {1,...,\Nlines}{
      \message{  Running i/N=\i/\Nlines, (x,y)=(\x,\y) ...^^J}
      \draw[mypurple!60,line width=0.4,shorten <=-2mm] (-\xmax,{-\a*\xmax}) -- (0,0);
      \draw[mypurple!60,line width=0.4,shorten >=-2mm] (0,0) -- (-\xmax,{\a*\xmax});
    }
    
    % CONSTANT R WORLD LINES
    \message{Making world constant-r lines (region I, III)...^^J}
    \foreach \i [evaluate={\c=2+\i/\Nlines; \vmax=kruskalv(\xmax,\c);}] in {1,...,\Nlines}{ %; \ct=tan(90*\c)
      \message{  Running i/N=\i/\Nlines, c=\c...^^J}
      \draw[world line,samples=\Nsamples,smooth,variable=\y,domain=-\vmax:\vmax]
        plot({-kruskalu(\y,\c)},\y);
    }
    
    \foreach \i in {1,...,\Nlines}{
      \draw[myorange!60,thick] (\xmax,{\xmax-\i*\xmax/3}) -- ({\i*\xmax/6},{-\i*\xmax/6});
      \draw[myorange!60,thick] (\xmax,{-\xmax+\i*\xmax/3}) -- ({\i*\xmax/6},{\i*\xmax/6});
    }
    
  \end{scope}
  
  
  
  % LABELS
  \pgfmathsetmacro\rl{-0.26*\xmax} % r left
  \pgfmathsetmacro\rr{0.35*\xmax} % r right
  
  \node[world line t,left] at (-\xmax,{\xmax*tanh(1/\Nlines)}) {$t=\text{常数}$};
  \node[world line,above=2,left] at (-\xmax,{kruskalv(\xmax,(3))}) {$r=\text{常数}$};
  \node[world line t,above=1,left] at (-1.05*\xmax,0) {$t=0$};
  \node[world line t,above=1,right] at (1.1*\xmax,0) {$t=0$};
  \node[below=10,left,rotate=45,mydarkblue] at (-.65*\xmax,-.65*\xmax)
    {\small $t=+\infty$};
    \node[below=10,right,rotate=-45,mydarkblue] at (.65*\xmax,-.65*\xmax)
    {\small $t=-\infty$};
    
    \node[above=10,right,rotate=45,mydarkblue] at (.65*\xmax,.65*\xmax)
    {\small $t=+\infty$};
    \node[above=10,left,rotate=-45,mydarkblue] at (-.65*\xmax,.65*\xmax)
    {\small $t=-\infty$};
    
    \node[above=10,right,rotate=45,mydarkblue] at (.02*\xmax,.02*\xmax)
    {\small $x=0=U,u=+\infty$};
    \node[below=10,right,rotate=-45,mydarkblue] at (.02*\xmax,-.02*\xmax)
    {\small $x=0=V,v=-\infty$};
  
  \node[above=2,left,mydarkblue] at (-\xmax,\xmax)
    {$U$};
  \node[above=2,right,mydarkblue] at (\xmax,\xmax)
    {$V$};
  
  % AXES
  \draw[->,thick] (-\xmax-0.1,0) -- (\xmax+0.2,0) node[left=1,below] {$X$};
  \draw[->,thick] (0,-\xmax-0.1) -- (0,\xmax+0.2) node[left=-1] {$T$};
  
  \draw[->,thick,mydarkblue] (SE) -- (O) -- (NW);
  \draw[->,thick,mydarkblue] (SW) -- (O) -- (NE);
  
 
  
  
\end{tikzpicture}



\end{document}